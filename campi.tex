\setcounter{section}{10} % per scegliere la lettera giusta
\section{Teoria dei Campi e di Galois}

\subsection{Richiami}
\begin{multicols}{2}

\begin{definition}
	Data un'estensione di campi $ K \subseteq L $ (o anche $ \faktor{L}{K} $), un elemento $ \alpha \in L $ si dice \emph{algebrico} su $ K $ quando 
	\[ \exists f \in K[\,x\,] \quad \text{tale che} \quad f(\alpha) = 0 \]
	Un elemento si dice \emph{trascendente} se non è algebrico.
\end{definition}

\begin{definition}
	Un'estensione di campi $ K \subseteq L $ si dice \emph{algebrica} quando ogni elemento di $ L $ è algebrico su $ K $
\end{definition}

\begin{definition}
	L'\emph{indice di un'estensione} $$  [L:K]  $$ è la dimensione di $ L $ visto come $ K $ spazio vettoriale.
\end{definition}

\begin{definition}
	Diciamo che un'estensione $ K \subseteq L $ è \emph{semplice} se esiste un elemento $ \alpha \in L $ tale che
	\[ L = K(\alpha) = \left\{ \frac{f(\alpha)}{g(\alpha)} \mid f, g \in K[\,x\,] \quad g(x) \neq 0  \right\} \]
\end{definition}

Possiamo costruire l'estensione semplice a partire dall'anello dei polinomi su K, applicandovi l'omomorfismo di valutazione in $ \alpha $
\fun{\varphi_\alpha}{K[\,x\,]}{K[\,\alpha\,] \subseteq L}{f(x)}{f(\alpha)}
l'immagine $ K[\alpha] $ è un sottoanello di un campo ed è quindi un dominio, di cui possiamo costruire il campo dei quozienti
\[ \Q(K[\,\alpha\,]) = K(\alpha) \]
Possiamo però mostrare che, se $ \alpha $ è algebrico, questa operazione non è necessaria. Infatti, il nucleo dell'omomorfismo di valutazione è costituito da tutti quei polinomi che si annullano in $ \alpha $, i quali costituiscono un ideale principale (perché $ K[\,x\,] $ è \textsc{ed}, dunque \textsc{pid}):
\[ \ker\varphi_\alpha = (\mu(x)) \]
Abbiamo già osservato che  l'immagine è un dominio, dunque $ (\mu(x)) $ è un'ideale primo (\ref{idealequoziente}). Abbiamo
\[ K[\,\alpha\,] \cong \frac{K[\,x\,]}{(\mu(x))} \]
Ma, poiché ci troviamo in un \textsc{pid}, $ (\mu(x)) $ è anche un ideale massimale (\ref{primossemassimale}) e dunque $ K[\,\alpha\,] $ è un campo, che coincide con il proprio campo dei quozienti:
\[ K[\,\alpha\,] = \Q(K[\,\alpha\,]) = K(\alpha) \]
In questo caso, dobbiamo avere anche che $ \mu(x) $ è un polinomio irriducibile, possiamo quindi scegliere un rappresentate privilegiato tra i generatori del nucleo:

\begin{definition}
	Chiamiamo \emph{polinomio minimo} di $ \alpha $ l'unico generatore monico di $ \ker\varphi_\alpha $, che indicheremo con $ \mu_\alpha $:
	\[ \ker\varphi_\alpha = (\mu_\alpha(x)) \]
\end{definition}

\begin{remark}
	Nel caso di estensioni algebriche semplici \[ [K(\alpha) : K ] = \deg \mu_\alpha \] questo perché $ K[\,\alpha\,] $ è un $ K $-spazio vettoriale di base $$  1,\, \alpha,\, \dots,\, \alpha^{\deg \mu_\alpha-1}  $$
\end{remark}

\begin{definition}
	Dato un insieme $ S \subseteq L $  definiamo
	\[ K(S) = \bigcap_{K, S \subseteq F \subseteq L} F  \]
\end{definition}

\begin{definition}
	Chiamiamo \emph{Torre di Estensioni} una catena di inclusioni tra campi
	\[ K \subseteq F \subseteq L \qquad\text{o anche}\qquad \begin{tikzcd}[row sep=0.25cm]
	L \arrow [d,-] \\
	F \arrow [d,-] \\
	K
	\end{tikzcd} \]
\end{definition}

Elenchiamo ora una serie di proprietà delle torri di estensioni:

\begin{prop}\label{1}
	$ L/K $ è finita $ \Leftrightarrow $ $ L/F $ e $ F/K $ sono finite
	\[ \text{e in tal caso } [L : K] = [L : F][F : K] \]
\end{prop}
\begin{proof}
	è un conto sulle dimensioni come spazi vettoriali.
\end{proof}

\begin{prop}\label{2}
	$ L/K $ è finita $ \Rightarrow $ $ L/K $ è algebrica.
\end{prop}
\begin{proof}
	Diciamo $ [L:K] = n $. Preso un elemento $ \alpha \in L $, le sue potenze
	\[ 1,\, \alpha,\, \alpha^2,\, \alpha^3 ,\,\dots,\, \alpha^n \]
	sono troppe per essere linearmente indipendenti, quindi esistono $ a_0,\, \dots,\, a_n \in K $ tali che
	\[ a_0 + a_1 \alpha + \dots + a_n \alpha^n = 0 \]
	che possiamo leggere come l'annullarsi del polinomio
	\[ a_0 + a_1 x + \dots + a_n x^n \in K[\,x\,] \]
	nel nostro elemento $ \alpha $.
\end{proof}
\begin{remark}
	Il viceversa è falso! L'estensione $ [\bar{\Q} : \Q] $ ha dimensione infinita, perché devono starci, per dire, tutte le radici dell'unità.
\end{remark}

Tuttavia, aggiungendo un'ipotesi riusciamo a invertire il risultato:

\begin{prop}\label{3}
	$ L/K $ algebrica e finitamente generata $ \Rightarrow $ finita.
\end{prop}
\begin{proof}
	Aggiungendo un elemento per volta,
	\[\begin{tikzcd}[column sep=0.25cm]
	K \arrow [r,-] &
	K(\alpha_1) \arrow[r,-] &
	K(\alpha_1, \alpha_2) \arrow[r,-] &
	\cdots \arrow[r,-] &
	K(\alpha_1, \cdots, \alpha_n) = L
	\end{tikzcd} \]
	 l'estensione ottenuta è sempre finita per il lemma \ref{1}.
\end{proof}
\columnbreak
\begin{prop}\label{4}
	$ L/K $ algebrica $ \Leftrightarrow $ $ L/F $ e $ F/K $ algebriche.
\end{prop}
\begin{proof}
	$ \Rightarrow $. Un polinomio a coefficienti in $ K $ ha, a maggior ragione, coefficienti in $ F $, quindi $ L/F $ è automaticamente algebrica. Analogamente, ogni elemento in $ F $ è anche un elemento di $ L $, quindi è algebrico su $ K $.
	
	$ \Leftarrow $. Preso un elemento $ \alpha $ di $ L $, troviamo un polinomio a coefficienti in $ F $
	\[ \mu = a_0 + a_1 x + \dots + a_n x^n \]
	su cui si annulla. A questo punto, possiamo costruire la sottoestensione
	\[ F_0 = K(a_0,\, \dots,\, a_n) \subseteq F \]
	che, essendo finitamente generata, ed algebrica, ci permette di invocare il lemma \ref{3} per concludere che è finita! Infine, possiamo aggiungere ad $ F_0 $ la radice di $ \mu $ da cui siamo partiti, producendo un'altra estensione finita
	\[\begin{tikzcd}[column sep=0.25cm]
	K \arrow [r,-] &
	F_0 \arrow[r,-] &
	F_0(\alpha) \arrow[r,-] &
	L
	\end{tikzcd} \]
	e dunque algebrica. Così, ogni $ \alpha \in L $ è algebrico su $ K $.
\end{proof}

\begin{definition}
	Un campo $ \Omega $ è detto \emph{algebricamente chiuso} quando ogni polinomio $ f \in \Omega[\, x\,] $ ammette almeno una radice in $ \Omega $.
\end{definition}

\begin{definition}
	$ \overline{\Omega} $ è detta \emph{chiusura algebrica} di $ \Omega $ se:
	\begin{enumerate}
		\item $ \overline{\Omega} $ è algebricamente chiuso.
		\item $ \faktor{\overline{\Omega}}{\Omega} $ è algebrica.
	\end{enumerate}
\end{definition}

\begin{theorem}[$ \exists ! \overline{K} $]
	Ogni campo $ K $ ammette una chiusura algebrica $ \overline{K} $ e ogni due sue chiusure $ \overline{K}, \overline{\overline{K}} $ sono $ K $-isomorfe, ovvero esiste un isomorfismo
	\[ \varphi : \overline{K} \to \overline{\overline{K}} \]
	tale che $ \varphi_{|K} = id $.
\end{theorem}


\end{multicols}


\subsection{Separabilità}
\begin{multicols}{2}
	
	\begin{theorem}[di immersione]\label{est}
		Sia $ \faktor{K(\alpha)}{K} $ un'estensione algebrica semplice. Ogni immersione
		\[ \varphi: K \hookrightarrow \overline{K} \]
		ammette esattamente tante estensioni a un omomorfismo
		\[ \tilde{\varphi}: K(\alpha) \hookrightarrow \overline{K} \]
		 quante le radici di $ \varphi(\mu_\alpha) $.
	\end{theorem}
	\begin{proof}
		Consideriamo l'estensione algebrica come quoziente
		\[ K(\alpha) \cong \frac{K[\, x \,]}{(\mu_\alpha)} \]
		e definiamo l'estensione a partire da $ K[\, x \,] $ in modo che
		\begin{align*}
		\Phi\colon &K[\, x \,] \to \bar{K} \\
		& k \mapsto \varphi(k) \\
		& x \mapsto \beta 
		\end{align*}
		imponendo che il nucleo corrisponda con l'ideale generato dal polinomio minimo di $ \alpha $
		\[ (\mu_\alpha) = \ker{\Phi} = \{ p \in K[\, x \,] \mid \varphi(p)(\beta) = 0 \} \]
		otteniamo una mappa iniettiva, come cercato.
		Dobbiamo pertanto scegliere $ \beta $ tra le radici di $ \varphi(\mu_\alpha) $; facendolo siamo sicuri che
		\[ (\mu_\alpha) \subseteq \ker{\Phi} \]
		ed essendo $ (\mu_\alpha) $ massimale e l'immagine non banale (perché stiamo quantomeno immergendo il campo), otteniamo
		\[ (\mu_\alpha) = \ker{\Phi} .\]
	\end{proof}
	\begin{remark}
		Il teorema si generalizza facilmente per induzione ad estensioni finite. Vale inoltre un risultato analogo per estensioni infinite, che garantisce l'esistenza dell'estensione. La dimostrazione è appena più tecnica; è un'applicazione del Lemma di Zorn sull'insieme delle sottoestensioni per cui riusciamo ad estendere l'immersione. 
	\end{remark}


	\begin{definition}[Separabile]
		Un polinomio $ f \in K[\,x\,] $ si dice \emph{separabile} se ha tutte le radici distinte in $ \bar{K} $.
		
		Un'estensione algebrica $ \faktor{L}{K} $ si dice \emph{separabile} se, comunque scegliamo $\alpha \in L  $, il corrispondente polinomio minimo $ \mu_\alpha $ è separabile.
	\end{definition}

\begin{theorem}[Criterio della derivata]
	Se siamo in caratteristica $ 0 $ o su un campo finito, allora tutti i polinomi irriducibili sono separabili.
\end{theorem}
\begin{proof}
	Consideriamo $ f $ assieme alla sua derivata formale $ f' $. \`{E} chiaro, per come è definita la derivata, che $ f $ ha radici multiple se e solo se ha radici in comune con $ f' $. Consideriamo
	$$  \textsc{mcd}(f, f')  $$
	che, poiché $ f $ è irriducibile, può essere solo $ 1 $ o $ f $.
	
	Osserviamo che l'\textsc{mcd} di due polinomi appartiene all'anello di definizione di questi, dunque, se sono coprimi, trovandoci in un \textsc{ed}
	\[ \exists g, h  \qquad fg +f'h = 1 \]
	lo sono a maggior ragione nell'anello costruito sulla chiusura algebrica. Così, se $  \textsc{mcd}(f, f') = 1  $, abbiamo finito. Altrimenti dobbiamo avere che $ f'=0 $, ma è chiaro che, in caratteristica zero, questo non può succedere e, se ci mettiamo in $ \mathbb{F}_p[\,x\,] $, possiamo mostrare che
	\[ f' = 0 \quad\Rightarrow\quad f = g^p \]
	semplicemente scrivendo esplicitamente il polinomio e osservando che tutti i coefficienti che hanno una speranza di essere non-nulli sono quelli dei termini $ x^{pk} $.
\end{proof}

\begin{theorem}[delle immersioni separabili]\label{immsep}
	Sia $ \faktor{K(\alpha)}{K} $ un'estensione algebrica, di grado $ n $. Se è separabile, ogni immersione
	\[ \varphi: K \hookrightarrow \bar{K} \]
	ammette esattamente n estensioni.
\end{theorem}
\begin{proof}
	Il grado di $ \varphi\mu_\alpha $ è $ n $, che ammette esattamente $ n $ radici distinte. Il teorema \ref{est} conclude.
\end{proof}
\begin{remark}
	Il teorema precedente si generalizza per induzione al caso di estensioni finite.
\end{remark}

\begin{definition}
	In un'estensione algebrica $ \faktor{L}{K} $ si dicono \emph{coniugati} di un elemento $ \alpha \in L $, tutte le soluzioni del polinomio minimo $ \mu_\alpha $.
\end{definition}

\begin{remark}
	Le estensioni dell'immersione del teorema di sopra (\ref{immsep}) sono proprio gli omomorfismi di coniugio $ \varphi_i $, che mandano $ \alpha_1 \mapsto \alpha_i $.
\end{remark}

\begin{theorem}[dell'Elemento primitivo]\label{primitivo}
	Ogni estensione separabile finita è semplice.
\end{theorem}
\begin{proof}
	Se $ K $ è un campo finito, il suo gruppo moltiplicativo è ciclico, dunque ogni generatore è primitivo.
	
	Rimane il caso in cui $ K $ è infinito.
	Prendiamo $ E/K $ estensione separabile e finita, dove possiamo assumere che
	\[ E = K(\alpha, \beta) \]
	Consideriamo gli $ n = [E:K] $ omomorfismi di coniugio che ci garantisce il teorema \ref{immsep}
	\[ \varphi_1,\, \dots,\, \varphi_n : E \to \bar{K}  \]
	e costruiamo il polinomio a coefficienti in $ \bar{K}[\,x\,] $
	\[ F(x) = \prod_{i < j}{\left(\varphi_i(\alpha + x \beta) - \varphi_j(\alpha + x \beta)\right)}. \]
	Osserviamo che questo polinomio non è identicamente nullo
	\[ F \equiv 0 \LR \exists i \neq j \text{ tale che }\; \begin{cases}
	\varphi_i(\alpha) = \varphi_j(\alpha) \\
	\varphi_i(\beta) = \varphi_j(\beta)
	\end{cases} \]
	dunque, essendo di grado finito in un campo infinito, ammette una \textquotedblleft non radice" $ t \in K $ tale che
	$  F(t) \neq 0  $, chiamiamo 
	\[ \gamma = \alpha + t\beta \]
	Vogliamo mostrare che $ K(\gamma) = K(\alpha, \beta) $. Il polinomio minimo $ \mu_\gamma $ ha tra le radici almeno
	\[ \varphi_1{(\gamma)}, \, \dots, \, \varphi_n{(\gamma)} \]
	che sono tutti distinti
	\[ \varphi_i(\gamma) \neq \varphi_j(\gamma) \quad\forall i \neq j \]
	altrimenti $ \gamma $ sarebbe soluzione del polinomio $ F $. Possiamo allora affermare che
	\[ [K(\gamma):K] = \deg{\mu_\gamma} \geq n = [K(\alpha, \beta):K], \]
	e che, avendo $K(\gamma) \subseteq E $ per costruzione, $ K(\gamma) = E $.
\end{proof}

\begin{remark}
	La separabilità si distribuisce in tutte le direzioni lungo le torri di estensioni.
\end{remark}
\end{multicols}

\subsection{Normalità}
\begin{multicols}{2}
Introduciamo ora un'altra utile proprietà di alcune estensioni

\begin{definition}[Normalità]
	Un'estensione algebrica $ \faktor{F}{K} $ si dice \emph{normale} se, preso comunque un $ K $-omomorfismo
	\[ \varphi \colon F \to \bar{K} \qquad\text{tale che } \varphi_{|K} = id \]
	questo rispetta l'estensione: $ \varphi(F) = F $.
\end{definition}

\textbf{Esempi.}
\begin{enumerate}
	\item Tutte le estensioni di grado $ 2 $ sono normali.
	\item $ {\Q(\sqrt[3]{2})}/{\Q} $ non è normale.
\end{enumerate}

\begin{theorem}[normale]
	In un'estensione $ \faktor{F}{K} $ algebrica e normale, ogni polinomio $ f \in K[\,x\,] $ irriducibile e con almeno una radice in $ F $, ha tutte le radici in $ F $.
\end{theorem}
\begin{proof}
	Consideriamo tutte le radici che $ f $ ammette in $ \bar{K} $: $ \alpha_1, \dots, \alpha_n $. Possiamo assumere senza perdita di generalità che $ \alpha_1 \in F $. Definiamo i $ K $-omomorfismi $ \varphi_1,\, \dots,\, \varphi_n $ in modo che
	\fun{\varphi_i}{K(\alpha_1)}{\bar{K}}{\alpha_1}{\alpha_i}
	Possiamo estenderli grazie al teorema \ref{est} a $ K $-omomorfismi
	\[ \tilde{\varphi}_i \colon F \to K \] 
	ma $ \faktor{F}{K} $ è normale, pertanto 
	$$  \tilde{\varphi}_i(F) = F \qquad\forall \; i \in \{1, \dots, n\} $$
	dunque $ \alpha_i \in F $ per ogni $ i $, come volevamo.
	
\end{proof}

\begin{theorem}[Caratterizzazione della normalità]\label{carnorm}
	Sia $ \faktor{F}{K} $ un'estensione (algebrica) finita. Questa è normale se e solo corrisponde al campo di spezzamento di una famiglia di polinomi in $ K[\,x\,] $.
\end{theorem}
\begin{proof}
	$ \Rightarrow$. La finitezza ci dice che l'estensione è anche finitamente generata
	 \[ F = K(\alpha_1, \dots, \alpha_n ) \]
	ci basta dunque prendere la famiglia di polinomi $ \{\mu_{\alpha_i} \}_{i} $ e, per il teorema precedente, questi polinomi si spezzano in $ F $.
	
	$ \Leftarrow $. Supponiamo che $ F $ sia il campo di spezzamento della famiglia di polinomi $ \{f_i\}_{i \in I} $, con radici $ \{\alpha_i,j\}_{i,j} $. Abbiamo che 
	\[F = K(\{\alpha_{i, j}\})\]
	 Preso un $ K $-automorfismo
	 \[ \varphi \colon F \to \bar{K} \]
	 abbiamo già osservato che questo deve necessariamente permutare i coniugati, dunque sicuramente
	 \[ \varphi(F) \subseteq F \]
	 ma, visto che $ \varphi $ è iniettivo, $ \varphi(F) $ e $ F $ hanno lo stesso grado e pertanto sono la stessa estensione di $ K $.
\end{proof}

 Consideriamo ora il problema del comportamento della normalità attraverso le torri di estensioni
\[\begin{tikzcd}[row sep=0.25cm]
L \arrow [d,-] \\
F \arrow [d,-] \\
K
\end{tikzcd} \]

e chiediamoci:
\begin{itemize}
	\item Se L/K è normale, lo è anche L/F?
	
	Sì! Se $ \varphi $ è un $ F $-omomorfismo è anche un $ K $-omomorfismo, che già rispettava l'estensione $$  \varphi(L)=L  $$
	
	\item Se L/K è normale, lo è anche F/K? No!
	\[\begin{tikzcd}[column sep=0.25cm]
	\Q \arrow [r,-] &
	\Q(\sqrt[4]{2}) \arrow [r,-] &
	\Q(\sqrt[4]{2}, i)
	\end{tikzcd} \]
	
	L'ultimo campo a destra è il luogo in cui $ x^4-2 $ si spezza, pertanto è normale. Possiamo però costruire un $ K $-omomorfismo che manda $ \sqrt[4]{2} $ in $ i\sqrt[4]{2} $.
	
	\item Se L/F e F/K sono normali, lo è anche L/K? No!
	\[\begin{tikzcd}[column sep=0.25cm]
	\Q \arrow [r,-] &
	\Q(\sqrt{2}) \arrow [r,-] &
	\Q(\sqrt[4]{2})
	\end{tikzcd} \]
	
	Abbiamo già osservato che le estensioni di grado due sono normali, purtroppo $ \Q(\sqrt[4]{2}) $ non spezza il polinomio minimo $ x^4-2 $.
\end{itemize}


\end{multicols}

\subsection{Teoria di Galois}
\begin{multicols}{2}
	
	\begin{definition}
		Un'estensione $ \faktor{E}{K} $ algebrica si dice \emph{di Galois} se è sia normale che separabile.
	\end{definition}
	
	(Da qui in avanti svilupperemo la Teoria di Galois per estensioni \emph{finite}, pertanto quando diremo che un'estensione ha questa proprietà, sottintenderemo la finitezza.)
	
	
	In questo caso il gruppo degli automorfismi 
	\[ \text{Aut}_K(E) = \{ \varphi : E \to E \mid \varphi_K = id \} \]
	per normalità è costituito dalle estensioni delle immersioni di $ K $ nella sua chiusura algebrica
	\[ \text{Aut}_K(E) = \{ \varphi : E \to \bar{K} \mid \varphi_K = id  \} \]
	che, per separabilità, sono tante quante il grado dell'estensione
	\[ |\text{Aut}_K(E)| = [E : K] \]
	e viene chiamato \emph{gruppo di Galois} dell'estensione
	\[ \Gal{\faktor{E}{K}} := \text{Aut}_K(E) \]
	
	Possiamo anche parlare del gruppo di Galois di un polinomio, riferendoci indirettamente al gruppo di Galois dell'estensione corrispondente a quella del suo campo di spezzamento, le due definizioni sono equivalenti per la proposizione \ref{carnorm}.
	
	Pensarla in quest'ottica ci permette di procede con una piacevole osservazione:
	\begin{remark}
		Il gruppo di Galois è composto da tutti e soli quegli omomorfismi che coniugano le radici di $ f $, pertanto
		\[ \Gal{f} \hookrightarrow \S_n \]
		Inoltre, se $ f $ è irriducibile, l'azione del gruppo su queste radici è \emph{transitiva}, nel senso che appartengono tutte alla stessa orbita.
	\end{remark}
	
	\textbf{Esempi.}
	\begin{enumerate}
		\item Se $ \faktor{E}{K} $ è di grado $ 2 $ e separabile, allora è di Galois e
		\[ \Gal{\faktor{E}{K}} \cong \Z_2 \]
		\item $ \Gal{x^3 -2} \cong \S_3 $
		\item $ \Gal{\Q(\zeta_7)/\Q} \cong \Z_6 $.
		\item $ \Gal{\Q(\zeta_7+ \zeta_7^{-1})/\Q} \cong \Z_3 $
	\end{enumerate}
	
	\begin{remark}
		Come si comporta la proprietà di Galois sulle torri? La separabilità si estende in tutte le direzioni, pertanto si comporta tanto male quanto la normalità. In particolare, l'unica proposizione che vale è:	 
	\end{remark}
\begin{prop}
	Data una torre 
	\[\begin{tikzcd}[column sep=0.30cm]
	K \arrow [r,-] &
	F \arrow [r,-] &
	L
	\end{tikzcd} \] se $ L/K $ è di Galois, allora anche $ L/F $ è di Galois.
\end{prop}
	\columnbreak
	Introduciamo ora due lemmi fondamentali.
	\begin{prop}[del gruppo logaritmo]\label{cosetta1}
		Siano $ E/K $ un'estensione di Galois e $ H < \Gal{E/K} $, allora
		\[  E^H = K \LR H = \Gal{E/K}  \]
	\end{prop}
	\begin{proof} Dimostriamo un'implicazione alla volta. \\
		$ \Leftarrow $. Gli elementi di $ H $ fissano $ K $ per definizione, è quindi chiara la prima inclusione
		\[ K \subseteq E^{\Gal{E/K}}. \]
		Inoltre presa un'estensione semplice $ K(\alpha) $ non banale
		\[ K \subset K(\alpha) \subseteq E \]
		esiste un omomorfismo che manda $ \alpha $ in un suo coniugato
		\[ \varphi: K(\alpha) \to E \quad\text{ tale che } \varphi(\alpha)\neq \alpha \]
		che possiamo estendere a un'immersione (\ref{est})
		\[ \tilde{\varphi}: E \to \bar{K} \quad\text{ che fissa $ E $.} \]
		Abbiamo pertanto scovato un automorfismo $ \tilde{\varphi}_{|E} \in \Gal{E/K} $ che non fissa $ \alpha $, che ci costringe ad accettare che
		\[ E^{\Gal{E/K}} \subseteq K. \]
		
		$ \Rightarrow $. Per il teorema dell'elemento primitivo \ref{primitivo} possiamo assumere che
		\[ E = K(\alpha). \]
		Consideriamo il polinomio che ha per radici tutti i suoi coniugati
		\[ f(x) = \prod_{\sigma \in H}{(x - \sigma(\alpha))} \]
		e osserviamo che per ogni $ \gamma \in H $ si ha
		\[ \gamma f = \prod_{\sigma \in H}{(x - \gamma\sigma(\alpha))} = \prod_{\gamma\sigma \in H}{(x - \gamma\sigma(\alpha))} = f \]
		dunque tutti i suoi coefficienti devono essere fissati da ogni elemento di $ H $, che pertanto vive in $ E^H[\, x\,] = K[\, x\,] $.
		Avendo $ \alpha $ come radice, cade nell'ideale $ (\mu_\alpha) $, quindi
		\[ |\Gal{E/K}| = \deg \mu_\alpha \leq \deg f = |H| \]
		da cui la tesi.
	\end{proof}
	
	\begin{prop}[del gruppo all'esponente]
		Siano $ E/K $ un'estensione di Galois, $ H < \Gal{E/K} $ e $ \sigma \in \Gal{E/K} $, allora
		\[ \sigma (E^H) = E^{\sigma H \sigma^{-1}} \]
	\end{prop}
	\begin{proof}
		Ci basta scrivere le definizioni
		\[ E^{\sigma H \sigma^{-1}} = \{ \alpha \in E \mid \sigma\tau\sigma^{-1}(\alpha) = \alpha \quad\forall \tau \in H \} \]
		ossia, chiamando $ \beta = \sigma^{-1}(\alpha) $,
		\[ = \{ \sigma(\beta) \in E \mid \tau(\beta) = \beta \quad\forall \tau \in H \} \]
		ma $ \sigma $ è un automorfismo di $ E $, pertanto \textquotedblleft$ \sigma(\beta) \in E $" mi elenca tutti gli elementi di $ E $, così che
		\[ = \sigma(E^H). \]
	\end{proof}
	
	\begin{theorem}[di corrispondenza]\label{corrgall}
		Sia $ \faktor{L}{K} $ un'estensione di Galois finita. La mappa
		\fun{\alpha}
		{\{F \text{ campo} \mid K \subseteq F \subseteq L \}}
		{\left\{ H < \Gal{\faktor{L}{K}} \right\}}
		{F}
		{\Gal{L/F}}
		è biettiva con inversa
		\[ \beta: H \mapsto L^H = \{ \alpha \in L \mid \sigma(\alpha) = \alpha \;\forall \sigma \in H \} \]
		Inoltre
		\[ H \lhd\; \Gal{L/K} \LR L^H/K \text{ è di Galois} \]
		e in questo caso
		\[ \Gal{L^H/K} \cong \dfrac{\Gal{L/K}}{\Gal{L/L^H}} \]
		
	\end{theorem}
	\begin{proof}
	Applicando la definizione
	\[ \alpha(\beta(H)) = \alpha(L^H) = \Gal{L/L^H} \]
	e usando il lemma \ref{cosetta1} appena dimostrato, abbiamo che
	\[ (L)^H = L^H \RR \Gal{L/L^H} = H \]
	
	Analogamente
	\[ \beta(\alpha(F)) = \beta(\Gal{L/F}) = L^{\Gal{L/F}} = F \]
	dove abbiamo il lemma nel senso più naturale.
	
	
	Per la seconda parte, applichiamo il secondo lemma:
	\begin{align*}
		H \lhd\; \Gal{L/K} &\LR \sigma H \sigma^{-1} = H & \forall\sigma \in \Gal{L/K} \\
		& \LR \sigma( L^H) = L^H & \forall\sigma \in \Gal{L/K}
	\end{align*}
	ovvero se e solo se $ L^H $ viene rispettato da tutti i $ K $-automorfismi di $ L $, ossia se e soltanto se è a sua volta un'estensione di Galois. In tal caso, ci concentriamo sulla funzione di restrizione
	\fun{\texttt{res}}{\Gal{L/K}}{\Gal{L^H/K}}{\sigma}{\sigma_{|H}}
	che è un omomorfismo di nucleo
	\[ \ker \texttt{res} = \Gal{L/L^H} \]
	dunque il Primo Teorema di Omomorfismo ci serve la tesi.
	\end{proof}
	
	
\end{multicols}

\subsection{Relazioni tra estensioni e gruppi di Galois}
\begin{multicols}{2}
	\begin{prop}
		Valgono i seguenti fatti:
		\begin{enumerate}
			\item  $ H < K \LR L^H \supseteq L^K $
			\item $ L^{H \cap K} = L^H L^K $
			\item $ L^{\gen{H, K}} = L^H \cap L^K $
		\end{enumerate}
		
	\end{prop}
	
	\begin{theorem}[dell'estensione a caso]\label{galrandom}
		Se $ L/K $ è un'estensione di Galois e $ F/K $ è un'estensione algebrica, allora $ LF/F $ è di Galois. 
		
		\[\begin{tikzcd}[row sep=0.25cm, column sep=0.3cm]
		 & LF \arrow[dl,-] \arrow[dr,-, dashed] &  \\
		L \arrow[ddr,-] \arrow[dr,-] &  & F \arrow [ddl,-] \\
		& L\cap F \arrow[ur,-] \arrow[d,-]  & \\
		& K &
		\end{tikzcd} \]
		
		Inoltre
		\[ \Gal{LF/F} \cong \Gal{L / L\cap F} \]
	\end{theorem}
	
	\begin{proof}
		Come sempre, la separabilità non ha davvero bisogno di essere controllata, ci concentriamo quindi sulla normalità. Prendiamo un $ F $-automorfismo $ \varphi $ di $ LF $. Per la normalità dell'estensione $$  \varphi(L) = L.  $$ Sapendo per ipotesi che $ \varphi_{|F} = id $: 
		\[ \varphi(LF) = \varphi(L)\varphi(F) = LF \]
		come volevamo. Vogliamo ora far vedere che l'omomorfismo di restrizione
		\fun{\texttt{res}}{\Gal{LF/F}}{\Gal{L/K}}{\varphi}{\varphi_{|L}}
		è iniettivo; infatti un $ F $-omomorfismo che è agisce come l'identità su $ L $ è necessariamente banale!
		\[ \ker \texttt{res} = \{ \varphi \in \Gal{LF/F} \mid \varphi_{|L} = id \} = \{ id \} \]
		Dunque possiamo identificare il gruppo di Galois appena individuato con un sottogruppo di $ \Gal{L/K} $ e pertanto, per il teorema di corrispondenza \ref{corrgall} possiamo associarvi un'estensione di $ K $:
		\[ L^{\texttt{res}(\Gal{LF/F})} = L \cap F \]
		è chiaro che se un elemento appartiene ad entrambi i campi, vi rimane, perché i $ \varphi $ fissano tutti gli elementi di $ F $ per definizione! Inoltre, se un elemento $ \alpha $ vive in $ L $ ma non in $ F $, non appartiene sicuramente al campo di partenza $ K \subseteq F $, quindi ha almeno un coniugato in cui viene mandato da almeno un elemento di $ \Gal{LF/F} $. O meglio
		\[ \alpha \in L^{\texttt{res}(\Gal{LF/F})} \quad\Rightarrow\quad \alpha \in LF^{\Gal{LF/F}} = F \]
		per il lemma \ref{cosetta1}. Dobbiamo pertanto concludere che
		\[ \texttt{res}(\Gal{LF/F}) \cong \Gal{L/L\cap F} \]
	\end{proof}
	\columnbreak
	\begin{definition}
		Diciamo che due estensioni $ L_1/K $ e $ L_2/K $ sono \emph{linearmente disgiunte} quando
		\[ L_1 \cap L_2 = K \]
	\end{definition}

	\begin{remark}
		In generale si ha
		\[\begin{tikzcd}[row sep=0.25cm, column sep=0.3cm]
		& L_1L_2 \arrow[dl,-] \arrow[dr,-] &  \\
		L_1 \arrow[ddr,-, "m" left] \arrow[dr,-] &  & L_2 \arrow [ddl,-, "n"] \\
		& L_1\cap L_2 \arrow[ur,-] \arrow[d,-, "c"]  & \\
		& K &
		\end{tikzcd} \]
		e visto che $ c \mid (m, n) $, se le due estensioni hanno grado coprimo allora devono essere linearmente disgiunte!
	\end{remark}
	\begin{remark}
		$ L_1 $ e $ L_2 $, estensioni di Galois, sono linearmente disgiunte se e solo se \[ [L_1L_2: K] = [L_1: K][L_2: K] \]
	\end{remark}
	
	\begin{theorem}[dei prodotti di Galois]
		Siano $ L_1/K $ e $ L_2/K $ estensioni di Galois, allora l'estensione $ L_1L_2/K $ è di Galois. 
		
		\[\begin{tikzcd}[row sep=0.25cm, column sep=0.2cm]
		& L_1L_2 \arrow[dl,-] \arrow[dr,-] \arrow[dd, -, dashed] &  \\
		L_1 \arrow[dr,-] &  & L_2 \arrow [dl,-] \\
		& K &
		\end{tikzcd} \]
		
		Inoltre \[ \Gal{L_1L_2/K} \hookrightarrow \Gal{L_1/K} \times \Gal{L_2/K} \]
		e questa immersione è un isomorfismo se e solo se le estensioni di partenza sono linearmente disgiunte.
	\end{theorem}
	
	\begin{proof}
		$ L_1L_2/K $ è di Galois: infatti $$  \varphi(L_1L_2) = \varphi(L_1)\varphi(L_2) = L_1L_2  $$ perché le due estensioni sono normali.
		
		Consideriamo ora l'omomorfismo di restrizione
		\fun{\texttt{res}}{\Gal{L_1L_2/K}}{\Gal{L_1/K} \times \Gal{L_2/K}}{\varphi}{(\varphi_{|L_1}, \varphi_{|L_2})}
		
		che è iniettivo, perché se un omomorfismo agisce in modo identico su tutti i generatori del campo, lo farà su tutto il campo.
	
	Prendiamo i due sottogruppi $ H_1, H_2 \lhd \Gal{L_1L_2 /K} $ tali che
	\[ (L_1L_2)^{H_i} = L_i \]
	Se supponiamo le due estensioni linearmente disgiunte, allora
	\[ H_1 \cong \Gal{L_1L_2/L_1} \cong \Gal{L_2/L_1 \cap L_2} \cong \Gal{L_2/K}  \]
	usando in mezzo la proposizione \ref{galrandom}. Inoltre
	\[ H_1 \cap H_2 = \{ id \} \qquad\text{e}\qquad H_1H_2 = \Gal{L_1L_2 /K} \]
	quindi possiamo applicare il teorema di struttura \ref{struttura}
	\[ H_1 \times H_2 \cong \Gal{L_1L_2/K} \]
	il che è equivalente alla suriettività di $ \texttt{res} $.
	
	Viceversa, l'isomorfismo ci dice che
	$$  \gen{H_1, H_2} = \Gal{L_1L_2/K}  $$
	e pertanto
	$$ L_1 \cap L_2 = L_1L_2^{H_1} \cap L_1L_2^{H_2} = L_1L_2^{\gen{H_1, H_2}} = K. $$

\end{proof}
	
	\subsubsection{Gruppo di Galois di un ciclotomico}
	
	\begin{theorem}[dei ciclotomici]
		Siano $ n $ un intero positivo e $ \xi $ una radice $ n $-esima primitiva dell'unità, allora
		\[ \Gal{\mu_{\xi}} \cong \Z_n^\times \]
	\end{theorem}
	
	\begin{proof} Iniziamo osservando che tutti i coniugati di $ \xi $ sono radici di
		\[ f = x^n -1 = \mu_\xi (x)  g(x)\]
		
		
		Tutte le radici di $ \mu_\xi $ hanno ordine moltiplicativo $ n $ in $ \mathbb{C}^\times $: per definizione sono i coniugati di $ \xi $, dunque c'è un isomorfismo che li scambia.\\
		
		Mostriamo che tutte le radici primitive sono coniugate di $ \xi $; anzi, più in generale che per ogni primo $ p $ che non divide $ n $, $ \xi^p $ non è una radice di $ g $, mentre chiaramente è una radice di $ f $. Se, per assurdo, avessimo che
		\[ g(\xi^p) = 0 \]
		allora avremmo che $ \mu_\xi \mid g(x^p) $. Questo deve valere anche modulo $ p $, dove
		
		\[ \overline{g(x^p)} = \left(\overline{g(x)}\right)^p \]
		dunque $ \overline{\mu_\xi} $ e $ \overline{g} $ hanno almeno una radice in comune, mentre $ \overline{f} $ non ha radici doppie:
		\[ \overline{f'} =  nx^{n-1} \neq 0 \quad\Rightarrow\quad (\overline{f}, \overline{f'}) = 1.   \]
		
		Possiamo così concludere che $ \mu_\xi $ ha come radici tutte e sole le radici $ n $-esime primitive dell'unità, pertanto
		\[ \deg\mu_\xi = \phi(n) = |\Gal{\mu_\xi}| \]
		
		Infine, verifichiamo che la mappa
		\fun{\Phi}{\Z_n^\times}{\Gal{\mu_\xi}}{i}{\varphi_i: \xi \mapsto \xi^{i}}
		è un isomorfismo.
	\end{proof}
	
	
	
	\subsubsection{Gruppo di Galois di un campo finito}
	\begin{theorem}[Frobenius]
	Dato un primo $ p $ e un intero positivo $ n $
	\[ \Gal{\mathbb{F}_{p^n} /\mathbb{F}_p} \cong \gen{\Phi} \]
	dove $ \Phi $ è l'automorfismo di Frobenius: \fun{\Phi}{\mathbb{F}_{p^n}}{\mathbb{F}_{p^n}}{x}{x^p}
	\end{theorem}
	
	\begin{proof}
		La mappa $ \Phi $ è un omomorfismo perché
		\begin{align*}
		(a + b)^ p &= a^p + b^p \\
		(ab)^p &= a^p b^p
		\end{align*}
		e, fissando l'unità, è non banale. Pertanto dev'essere un automorfismo! Dunque abbiamo, quantomeno,
		\[ \gen{\Phi} < \Gal{\mathbb{F}_{p^n} /\mathbb{F}_p} \]
		Inoltre
		\[ |\gen{\Phi}| = n = |\Gal{\mathbb{F}_{p^n} /\mathbb{F}_p}| \]
		perché se vale
		\[ x^{p^m} = x \qquad\forall x \in \mathbb{F}_{p^n} \]
		dobbiamo avere che il polinomio $ x^{p^m} - x $ ha almeno $ p^n $ radici distinte, quindi ha grado $ p^m \geq p^n $. Inoltre, per costruzione,
		\[ x^{p^n} = x \qquad\forall x \in \mathbb{F}_{p^n} \]
		da cui la tesi.
	\end{proof}
	
	\begin{remark}
		Se volessimo gruppi di galois di estensioni più elaborate, ci basta ricordare che
		\[ \Gal{\mathbb{F}_{p^{dn}} /\mathbb{F}_{p^d} } \;<\; \Gal{\mathbb{F}_{p^{dn}} /\mathbb{F}_p} \cong \gen{\Phi} \]
		dunque
		\[ \Gal{\mathbb{F}_{p^{dn}} /\mathbb{F}_{p^d}} \cong \gen{\Phi^d} \]
	\end{remark}
\end{multicols}

\subsection{Esistenza e unicità della chiusura algebrica}
\begin{multicols}{2}
	\setcounter{theorem}{4}
	\begin{theorem}[$ \exists ! \overline{K} $]
		Ogni campo $ K $ ammette una chiusura algebrica $ \overline{K} $ e ogni due sue chiusure $ \overline{K}, \overline{\overline{K}} $ sono $ K $-isomorfe, ovvero esiste un isomorfismo
		\[ \varphi : \overline{K} \to \overline{\overline{K}} \]
		tale che $ \varphi_{|K} = id $.
	\end{theorem}

\begin{proof}
	Concentriamoci sull'esistenza. Prendiamo l'insieme di tutti i polinomi non costanti a coefficienti nel campo in questione e indicizziamoli opportunamente
	$$  \{ p \in K[\,x\,] \mid \deg p > 0 \} = \{p_\lambda \mid \lambda \in \Lambda\};  $$
	costruiamo ora l'anello dei polinomi su $ K $ con tante incognite quante i polinomi di $ K[\,x\,] $: detto
	\[ X = \{ x_\lambda \}_{\lambda \in \Lambda}, \]
	consideriamo l'anello
	\[ K[X] \]
	e in questo anello l'ideale
	\[ I = (p_{\lambda}(x_{\lambda}))_{\lambda \in \Lambda} \subseteq K[X]. \]
	
	Osserviamo che $ I $ è un ideale proprio, vale a dire
	\[ I \neq (1). \]
	Se così non fosse avremmo una serie di indici $ \{1, \dots, n\} \subseteq \Lambda $ e coefficienti $ f_1, \dots, f_n \in K[X] $ tali che
	\[ \sum f_i p_i(x_i) = 1, \]
	ma, essendo i $ p_i $ in questione finiti, possiamo costruire l'estensione 
	$$  L = K(\alpha_1, \alpha_2, \dots, \alpha_n),  $$
	dove $ \alpha_i $ è una radice di $ p_i $.
	Da questa estensione esce un bellissimo omomorfismo di valutazione
	\begin{align*}
	\psi \colon K[X] &\to L \\
	k &\mapsto k \\
	x_\lambda &\mapsto \begin{cases}
	0 \text{ se } \lambda \notin \{1, \dots, n\}\\
	\alpha_i \text{ se } \lambda = i \in \{1, \dots, n\}
	\end{cases} 
	\end{align*}
	che, annullando tutti i polinomi considerati, ci restituisce
	\[ 1 = \psi(1) = \sum \psi(f_i)\psi(p_i(x_i)) = 0, \]
	ovvero, una contraddizione!
	
	Pertanto $ I $ è proprio, dunque è contenuto in un qualche ideale massimale $ M $ (\ref{idmax}):
	\[ I \subseteq M \subseteq K[X]. \]
	Ci è quindi concesso porre
	\[ E_1 = \faktor{K[X]}{M}, \]
	che è
	\begin{enumerate}
		\item un campo, perché quoziente di un anello per un ideale massimale;
		\item contiene $ K $: essendo l'omomorfismo
		\[ K \hookrightarrow K[X] \to E_1 \]
		non banale perché
		\[ 1 \mapsto 1 \mapsto 1 + M \neq M \]
		dev'essere iniettivo;
		\item in cui ogni $ p_\lambda \in K[\,x\,] $ ha almeno una radice, infatti
		\[ p_\lambda(\overline{x_\lambda}) = \overline{p_\lambda(x_\lambda)} = 0 \]
	\end{enumerate}
	
	Non è però facile dimostrare che $ E_1 $ contiene anche le radici dei suoi polinomi, possiamo ovviare al problema costruendo in modo analogo $$ E_1 \subseteq E_2 \subseteq E_3 \subseteq E_4 \subseteq \dots  $$
	ogni volta infilandoci le radici di tutti i polinomi a coefficienti nel campo precedente. Consideriamo infine il grande campo
	\[ \Omega = \bigcup_{n \in \mathbb{N}} E_n. \]
	Quest'oggetto sarà algebricamente chiuso: ogni polinomio di $ \Omega[\,x\,] $ ha finiti monomi e pertanto vive in $ E_{m} $, per un qualche indice $ m $, dunque ha radici in $ E_{m+1} \subseteq \Omega $.
	
	Per concludere, possiamo considerare 
	\[ \boxed{\overline{K} = \{ \alpha \in \Omega \mid \alpha \text{ è algebrico su } K \}} \]
	e osservare che
	\begin{enumerate}
		\item è un campo: infatti presi $ \alpha, \beta \in \overline{K} $, abbiamo che
		\[ \alpha + \beta,\, \alpha\beta, \, \alpha^{-1} \in K(\alpha, \beta) \subseteq \overline{K} \]
		\item contiene $ K $: perché stava in $ \Omega $ e tutti i suoi elementi sono banalmente algebrici;
		\item è un'estensione algebrica di $ K $, per definizione;
		\item è algebricamente chiuso: preso un qualunque polinomio
		\[ p(x) \in \overline{K}[\, x \,] \]
		questo deve vivere in un $ E_m $, ovvero i coefficienti $ c_0, \dots c_n $ vivono in $ E_m $: esiste un'estensione \emph{algebrica}
		\[ L = K(c_0, \dots, c_n) \subseteq E_m  \]
		quindi $ p \in L[\,x\,] \subseteq E_m[\,x\,]  $ deve avere almeno una radice $ \alpha $ in $ E_{m+1} $, con cui possiamo estendere $ L $ a $ L(\alpha) $. Per la transitività delle estensioni algebriche sulla torre
		\[\begin{tikzcd}[column sep=0.25cm]
		K \arrow [r,-] &
		L \arrow [r,-] &
		L(\alpha)
		\end{tikzcd} \]
		scopriamo che $ \alpha \in \Omega $ è algebrico su $ K $ e pertanto appartiene a $ \overline{K} $
		
	\end{enumerate}
che dunque è una chiusura algebrica di $ K $.

Dimostriamo che è unica. Supponiamo di averne un'altra, $\overline{\overline{K}} $. Possiamo immergere il campo $ K $ in una delle sue chiusure
\[ \sigma :  K \hookrightarrow \overline{K} \]
ed estendere l'immersione all'altra chiusura, grazie alla versione infinita del teorema (\ref{est}), dato che $ \overline{\overline{K}}/K $ è algebrica:
\[ \tilde{\sigma} : \overline{\overline{K}} \to \overline{K} \]
Essendo però $ \tilde{\sigma}\left(\overline{\overline{K}}\right) $ algebricamente chiusa
\[ p(\alpha) = 0 \text{ in } \overline{\overline{K}} \quad\Rightarrow\quad \tilde{\sigma}p(\tilde{\sigma}\alpha) = 0 \text{ in } {\overline{K}} \]
ogni elemento di $ \overline{K} $, essendo algebrico su $ K $, è anche un elemento della seconda chiusura $ \overline{\overline{K}} $; pertanto $ \tilde{\sigma} $ è un isomorfismo.

\end{proof}
	
	
	
\end{multicols}


