\setcounter{section}{6} % per scegliere la lettera giusta
\section{Teoria dei Gruppi}
\subsection{Automorfismi e Azioni}
\begin{multicols}{2}
\begin{theorem}
	Se G è un gruppo, $ \left(\Aut{G}, \circ\right) $ è un gruppo.
\end{theorem}

\textbf{Esempi}.
\begin{enumerate}
	\item $ \Aut{\mathbb{Z}} \cong \{\pm id\} \cong \mathbb{Z}_2 $
	\item $ \Aut{\mathbb{Z}_n} \cong \mathbb{Z}_n^\times $
	\item $ \Aut{\mathbb{Q}} \cong \mathbb{Q}^\times $
	\item $ \Aut{\mathbb{R}} \cong \; ? $
\end{enumerate} 

\begin{definition}[Gruppo degli automorfismi interni]
	Sia $ \Int{G} = \{ \varphi_g \mid g \in G \} $ l'insieme di tutti gli automorfismi interni, i.e. degli automorfismi di coniugio: \[ \varphi_g(x) = gxg^{-1} \quad\forall x \in G \]
\end{definition}
\begin{remark}
	è immediato osservare che $ \Int{G} \lhd \Aut{G} $.
\end{remark}
\begin{theorem}[degli Automorfismi Interni]
	\[ \Int{G} \cong \faktor{G}{Z(G)} \]
\end{theorem}
\begin{proof}
	La funzione \fun{\Phi}{G}{\Int{G}}{g}{\varphi_g} è un omomorfismo con kernel $ Z(G) $. La tesi segue dal Primo Teorema di Omomorfismo.
\end{proof}

\begin{remark}
	\[ H \lhd G \;\Leftrightarrow\; \varphi_g\left(H\right) = H \quad \forall \varphi_g \in \Int{G} \]
\end{remark}

\begin{definition}[Sottogruppo caratteristico]
	Un sottogruppo $ H < G $ si dice caratteristico se è invariante per tutto $ \Aut{G} $, i.e. $$ \varphi\left(H\right) = H \quad \forall \varphi \in \Aut{G} $$
\end{definition}
\begin{remark}
	Un sottogruppo caratteristico è anche normale, ma non è vero il viceversa: basta considerare $ \langle (0, 1) \rangle \lhd \mathbb{Z}_2 \times \mathbb{Z}_2 $
\end{remark}

\begin{definition}[Azione]
	Si dice azione di un gruppo $ G $ su un insieme $ X $ un omomorfismo $ \varphi $ tale che \begin{align*}
	\varphi \colon G &\to \mathcal{S}\left(X\right) \\
	g &\mapsto \varphi_g(x) = g \cdot x.
	\end{align*}
\end{definition}

\textbf{Esempio.} Siano $ G = C = \{ z \in \mathbb{C} \mid |z| = 1 \} $, $ X = \mathbb{R}^2 $ e sia $ \varphi $ l'azione: \begin{align*}
\varphi \colon C &\to \mathcal{S}\left(\mathbb{R}^2\right) \\
z &\mapsto \mathcal{R}(O, \arg z)
\end{align*}

\begin{remark}
	Un'azione induce naturalmente una relazione di equivalenza su $ X $: $ x \sim y \Leftrightarrow \exists g \in G \; t.c.\; g \cdot x = y $. Viene quindi spontaneo prendere in considerazione gli elementi della partizione così ottenuta. 
\end{remark}
\begin{definition}[Orbita]
	Si dice orbita di un elemento $ x \in X $ l'insieme di tutti gli elementi che posso essere raggiunti da $ x $ tramite l'azione: \[ \Orb{x} = \{ g \cdot x \mid \forall g \in G \} \]
\end{definition}
\begin{remark}
	Detto $ R $ un insieme di rappresentanti delle varie orbite, per il partizionamento prima considerato: \[ X = \bigcup_{x \in R} \Orb{x} \;\Rightarrow\; |X| = \sum_{x \in R} |\Orb{x}| \]
\end{remark}

\begin{definition}[Stabilizzatore]
	Si dice stabilizzatore di un elemento $ x \in X $ l'insieme di tutti gli elementi di $ G $ che agiscono in modo banale su $ x $: \[ \Stab{x} = \{ g \in G \mid g \cdot x = x \} \]
\end{definition}
\begin{remark}
	Lo stabilizzatore è un sottogruppo $$ \Stab{x} < G $$ ma non è necessariamente normale. $$  Z_{\mathcal{A}_{14}}(3 \; 1 \; 4) < \mathcal{A}_{14}$$
\end{remark}

\begin{theorem}[Relazione Orbite-Stabilizzatori]
	\[ |G| = |\Orb{x}||\Stab{x}| \]
\end{theorem}
\begin{proof}
	La funzione $ f $ così definita
	\begin{align*}
	f \colon \{ g \Stab{x} \mid g \in G \} &\to \{\Orb{x} \mid x \in X\} \\
	g\Stab{x} &\mapsto g \cdot x
	\end{align*}
	è biunivoca, infatti:
	\begin{align*}
		g \cdot x = h \cdot x & \Leftrightarrow \varphi_g(x) = \varphi_h(x) \\
		& \Leftrightarrow \varphi_h^{-1} \varphi_g (x) = x \\
		& \Leftrightarrow \varphi_{h^{-1}g} (x) = x \\
		& \Leftrightarrow h^{-1}g \cdot x = x \\
		& \Leftrightarrow h^{-1}g \in \Stab{x} \\
		& \Leftrightarrow g \in h\Stab{x} \\
		& \Leftrightarrow g\Stab{x} = h\Stab{x} \\
	\end{align*}
\end{proof}
\begin{remark}
	Dall'osservazione precedente \[ |X| = \sum_{x \in R} \frac{|G|}{|\Stab{x}|} \]
\end{remark}

\textbf{Esempi.}
\begin{enumerate}
	\item $ [G = C,\; X = \mathbb{R}^2 ] $ e l'azione dell'ultimo esempio. Questa ruota ogni punto attorno all'origine, pertanto le orbite sono circonferenze centrate nell'origine e gli stabilizzatori sono tutti banali, tranne quello dell'origine che coincide con $ G $.
	\item $ [G = \mathbb{R},\; X = \mathbb{R}^2 ] $ e l'azione che trasforma $ r \in \mathbb{R} $ nella traslazione orizzontale di lunghezza $ r $. Le orbite sono le rette parallele alla traslazione e gli stabilizzatori sono tutti banali.
	\item $ [G,\; X = G ] $ e l'azione sia la mappa che manda un elemento $ g $ nel coniugio per questo $ \varphi_g(x) = gxg^{-1} $. L'orbita di un elemento contiene tutti i coniugati di questo ed è detta \emph{classe di coniugio} di $ x $ $ (\mathcal{C}_x) $. Lo stabilizzatore di $ x $ contiene tutti e soli gli elementi tali che $ xg = gx $, ovvero il sottogruppo di tutti gli elementi che commutano con $ x $, è detto \emph{centralizzatore} di $ x $ ($ Z_G(x) $).
	\item $ [G,\; X = \{ H \mid H < G \} ] $ e l'azione di coniugio. Le orbite non sono particolarmente interessanti, mentre lo stabilizzatore di un sottogruppo è detto \emph{Normalizzatore} di $ H $, $ N(H) $ ed è il più grande sottogruppo di $ G $ in cui $ H $ è normale.
\end{enumerate}
\begin{remark}
	$ H \lhd G \;\Leftrightarrow\; N(H) = G $
\end{remark}
\begin{remark}[euristica]
	Le azioni più comuni sono quelle naturali: il coniugio, la moltiplicazione a sinistra e, talvolta, la moltiplicazione a destra per l'inverso.
\end{remark}
\end{multicols}

\subsection{Formula delle Classi e Cauchy}
\begin{multicols}{2}

\begin{theorem}[Formula delle Classi]
	Per ogni gruppo finito vale
	\[ |G| = |Z(G)| + \sum_{\substack{x \in R'}} \frac{|G|}{|Z_G(x)|} \]
\end{theorem}
\begin{proof}
	Riprendiamo la partizione di $ X $ in orbite, ma separando quelle banali da quelle non
	
	\[ |X| = \sum_{\substack{x \in R \\ \Orb{x} = \{x\} }} 1 + \sum_{\substack{x \in R \\ \Orb{x} \neq \{x\} }} \frac{|G|}{|\Stab{x}|} \]
	
	Osserviamo cosa succede nel caso dell'azione di coniugo da un gruppo in se (l'esempio 3 di sopra). L'orbita di $ x $ è banale se e solo se $ gxg^{-1} = x $, $ \forall g \in G $, ovvero nel caso in cui $ x $ commuti con tutti gli elementi di $ G $ (stia nel centro). Dunque $ R = Z(G) $ mentre gli elementi rimanenti rappresentano centralizzatori distinti.
\end{proof}

\begin{definition}[$ p $-gruppo]
	Si dice $ p $-gruppo un gruppo finito $ G $ di ordine potenza di un primo $ p $: $ |G| = p^n $.
\end{definition}

\textbf{Esempi.}
\begin{enumerate}
	\item \textbf{Un $ p $-gruppo $ G $ ha centro non banale.} Tutti i centralizzatori degli elementi di $ R' $ hanno dimensione $ p^k $ per un intero $ 0 \leq k < n $, dunque
	\[ p \mid \frac{|G|}{|Z_G(x)|} \forall x \in R' \]
	pertanto, per la formula delle classi,
	\[ p \mid |G| - \sum_{\substack{x \in R'}} \frac{|G|}{|Z_G(x)|} = |Z(G)| \]
	che quindi, contenendo $ e $, deve avere almeno $ p $ elementi.
	
	\item \textbf{I gruppi di ordine $ p^2 $ sono abeliani.} Il centro di $ G $ avrà, per quanto appena dimostrato, ordine $ p $ o $ p^2 $. Nel secondo caso abbiamo finito. Nel primo \[  \left|\faktor{G}{Z(G)}\right| = p \] dunque il quoziente è ciclico. Presi due elementi qualunque $ x, y \in G $ possiamo esprimerli come $ x = g^h a $ e $ y = g^k b $, dove $ g $ è il generatore del quoziente e $ a, b \in Z(G) $. Allora, sfruttando la commutatività degli elementi del centro \[ xy = ( a) (g^k b) = g^{h+k} ab = g^{k+h} ba = (g^k b) (g^h a) = yx \]
	ricaviamo la commutativa per tutti gli elementi del gruppo.
	
	\item Una possibile dimostrazione del Teorema di Cauchy:
\end{enumerate}

\begin{theorem}[di Cauchy] \label{cauchy}
	Per ogni fattore primo $ p $ di $ |G| $ esiste un elemento $ g $ di $ G $ di ordine $ p $.
\end{theorem}
\begin{proof}[Dimostrazione Classica]
	Sia $ |G|= pn $, procediamo per induzione su $ n $. \\
	Se $ n = 1 $, $ G $ è ciclico, quindi ha un generatore di ordine $ p $. \\
	Supponiamo ora che tutti i gruppi di ordine $ kp  \quad\forall k < m$ abbiamo un elemento di ordine $ p $. Se $ |G|=pm $ ci sono due casi:
	\begin{enumerate}
		\item Esiste un sottogruppo proprio $ H $ di ordine multiplo di $ p $, da cui ricadiamo nell'ipotesi induttiva.
		\item Se nessun sottogruppo di $ G $ ha ordine divisibile per $ p $, allora
		\[ p \mid \frac{|G|}{|Z_G(x)|} \forall x \in R' \] perché i $ Z_G(x) < G $. Per la formula delle classi
		\[ p \mid |G| - \sum_{\substack{x \in R'}} \frac{|G|}{|Z_G(x)|} = |Z(G)| \]
		ma abbiamo supposto che i sottogruppi propri non abbiamo ordine multiplo di $ p $, dunque il centro deve coincidere con l'intero gruppo, che risulta pertanto commutativo.
	\end{enumerate}
\end{proof}
\begin{proof}[Dimostrazione Magica]
	Sia $$ X = \{(x_1, \dots, x_p) \in G^p \mid x_1\cdots x_n = 1 \} $$ questo insieme ha esattamente $ |G|^{p-1} $ elementi, infatti scelti i primi $ (p-1) $ l'ultimo è univocamente determinato come il suo unico inverso. Se una $ p $-upla non è composta da un solo elemento ripetuto, allora possiamo ciclare i suoi termini per ottenere altre $ (p-1) $ $ p $-uple in $ X $. Dunque, detto $ n $ il numero di $ g $ tali che $ g^p = 1 $ \[ p \mid |G|^{p-1} - n \;\Rightarrow\; p \mid n \] e poiché $ e^p = e $ ci sono almeno $ p $ elementi di ordine $ p $.
\end{proof}



\textbf{Esercizio.} Classificare i gruppi $ G $ di ordine $ 6 $.

Per Cauchy esistono $ x, y \in G $ di ordine, rispettivamente, 2 e 3.
\begin{itemize}
	\item Se $ G $ è abeliano, ord$ (xy) = 6 $, quindi $ G $ è ciclico e pertanto isomorfo a $ \mathbb{Z}_6 $.
	\item Se non lo è, costruiamo un isomorfismo esplicito...
\end{itemize}

\begin{theorem}[Cayley]\label{Cayley}
	Possiamo immergere ogni gruppo $ G $ in $ \mathcal{S}(G) $.
\end{theorem}
\begin{proof}
	L'azione di moltiplicazione a sinistra è fedele
	 \fun{\Phi}{G}{\mathcal{S}(G)}{g}{\varphi_g(x) = gx}
	ovvero, iniettiva.
\end{proof}

\begin{definition}[Sottogruppo generato]
	Sia $ S \subset G $ un \emph{sottoinsieme} di $ G $. Chiamiamo $ \langle S \rangle $ il più sottogruppo contenente $ S $, sottogruppo generato da $ S $. \[ \langle S \rangle = \bigcap_{\substack{H \leq G \\ S \subseteq H }} H  \]
\end{definition}
\begin{theorem}[Caratterizzazione dei sottogruppi generati]
	\[ \langle S \rangle = \{ s_1 \cdots s_k \mid k \in \mathbb{N}, \; s_i \in S \cup S^{-1} \} \]
\end{theorem}
\begin{proof}
	Chiamiamo $ X $ il magico insieme nel RHS. Chiaramente $ S \subseteq X $ e pertanto $ X $, che è facile verificare essere un gruppo, è parte della famiglia sotto intersezione: $ X \subseteq \bigcap \mathcal{F} $. Inoltre se $S \subseteq H < G $ sicuramente in $ H $ compaiono tutte le $ k $-uple di $ X $ e quindi $ X < H $ per ogni sottogruppo di $ \mathcal{F} $. Dunque $ X \subseteq \bigcap\mathcal{F} $.
\end{proof}

\textbf{Esempi.}
\begin{enumerate}
	\item $ \langle S \rangle $ è abeliano se e solo se tutti gli elementi di $ S $ commutano fra loro.
	\item $ \langle S \rangle $ è normale se e solo se ogni ogni elemento di $ S $ rimane in $ \langle S \rangle $ per coniugio.
	\item $ \langle S \rangle $ è caratteristico se e solo se ogni elemento di $ S $ viene mandato in $ \langle S \rangle $ da ogni automorfismo di $ G $.
	\item $ G' = \langle ghg^{-1}h^{-1} \mid g, h \in G \rangle $ è detto \emph{Gruppo dei Commuatatori o Gruppo Derviato di $ G $}. Questo gruppo gode di alcune proprietà fondamentali
	
	\begin{enumerate}
		\item $ G' = \{ e \} \;\Leftrightarrow\; G $ abeliano.
		\item $ G' $ è caratteristico e pertanto normale in $ G $.
		\item \textbf{Dato $ H \lhd G $, il quoziente $ \faktor{G}{H} $ è abeliano se e solo se $ G' < H $.}
	
		
	\end{enumerate}

\begin{proof}
	La verifica delle proprietà $ (a) $ e $ (b) $ è banale. Rimane l'ultima $ (c) $:
	\begin{align*}
	\faktor{G}{H} \;\text{abeliano} & \Leftrightarrow\; xHyH = yHxH &\forall x, y \in G \\
	& \Leftrightarrow\; xyH = yxH &\forall x, y \in G\\
	& \Leftrightarrow\; x^{-1}y^{-1}xy \in H &\forall x, y \in G\\
	& \Leftrightarrow\; g' \in H &\forall g' \in G'
	\end{align*}
\end{proof}
\begin{definition}
	$ \faktor{G}{G'} $ è detto l'abelianizzato di $ G $, perché è sempre abeliano!
\end{definition}
\end{enumerate}
\end{multicols}

\subsection{Gruppi Diedrali  $ D_n $}
\begin{multicols}{2}
\begin{definition}[Gruppo Diedrale]
	Sia $ D_n $ il gruppo delle isometrie dell'$ n $-agono regolare.
\end{definition}
\begin{theorem}[Caratterizzazione di $ D_n $]
	Si ha \[ D_n = \langle \rho, \sigma \mid \rho ^n = e, \sigma^2 = e, \sigma \rho \sigma = \rho^{-1} \rangle \]
\end{theorem}
\begin{proof}
	Tutti gli elementi sopra definiti possiamo ridurli a un elemento della forma $ \rho^k $ o $ \sigma \rho^k $ per un qualche $ 0 \leq k < n $. Questo perché così sono fatti i generatori e ogni operazione permessa (composizione e inversione) tra due generatori può immediatamente essere ridotta a questa forma attraverso le cancellazioni imposte. Inoltre possiamo immergere $ D_n $ in un sottogruppo di $ \mathbf{O}_2(\mathbb{R}) $ di ordine $ 2n $ attraverso un omomorfismo suriettivo: \begin{align*}
	\Phi \colon D_n &\to \mathbf{O}_2(\mathbb{R}) \\
	\sigma &\mapsto \left(\begin{matrix}
	-1 & 0 \\
	0 & 1
	\end{matrix}\right) \\
	\rho &\mapsto \left(\begin{matrix}
	\cos\frac{2\pi}{n} & \sin\frac{2\pi}{n} \\
	-\sin\frac{2\pi}{n} & \cos\frac{2\pi}{n}
	\end{matrix}\right) 
	\end{align*}
	Pertanto ognuno dei rappresentati sopra individua un'effettiva trasformazione distinta.
\end{proof}

\begin{remark}
	Conosciamo già un gruppo diedrale: $ D_3 \cong \mathcal{S}_3 $.
\end{remark}
\begin{remark}
	Il sottogruppo $ C_n $ delle rotazioni, generato da $ \rho $, è ovviamente ciclico e, avendo indice 2, è anche normale in $ D_n $. $$ \langle \rho \rangle = C_n \lhd D_n $$
\end{remark}

\begin{theorem}[Ordine degli elementi di $ D_n $]
	Sappiamo che
	\begin{itemize}
		\item tutte le simmetrie hanno ordine 2.
		\item ci sono $ \varphi(m) $ rotazioni di ordine $ m $, per ogni $ m \mid n $.
	\end{itemize}
\end{theorem}
\begin{proof}
	La seconda parte è immediata conseguenza della ciclicità del sottogruppo delle rotazioni.
	L'ordine delle riflessioni possiamo calcolarlo esplicitamente notando che $ \left(\sigma\rho^k\right)\left(\sigma\rho^k\right) = \left(\sigma\rho^k\sigma\right)\rho^k = \rho^{-k}\rho^{k} = e $ grazie alla terza proprietà imposta nella caratterizzazione.
\end{proof}

\begin{theorem}[Sottogruppi di $ D_n $]
	I sottogruppi $ H < D_n $ rientrano in una di queste due categorie:
	\begin{itemize}
		\item $ H < C_n $: di cui ne abbiamo esattamente uno per ogni ordine divisore di $ n $.
		\item $ H = (H \cap C_n) \sqcup \tau(H\cap C_n) $: di cui ce ne sono $ d $ di ordine $ \frac{2n}{d} $ per ogni $ d \mid n $.
	\end{itemize}
\end{theorem}
\begin{proof}
	Se $ H < C_n $ il risultato viene da Aritmetica. Se $ H \nless C_n $, $ H $ contiene almeno una rotazione $ \tau = \sigma\rho^i $. Consideriamo l'omomorfismo $ f $ che fa commutare il diagramma
	\[ \begin{tikzcd}
	D_n \arrow{r}{\Phi} \arrow[swap, dashed]{dr}{f} & \mathbf{O}_2(\mathbb{R}) \arrow{d}{det} \\
	& \{\pm 1\} \cong \mathbb{Z}_2	\end{tikzcd}
	\]
	Restringiamo l'omomorfismo trovato ad $ H $
	\[ \begin{tikzcd}
	D_n \arrow{r}{\Phi} \arrow[swap]{dr}{f} & \mathbf{O}_2(\mathbb{R}) \arrow{d}{det} \\
	H \arrow[dashed]{r}{} \arrow[swap, hook]{u}{} & \mathbb{Z}_2	\end{tikzcd}
	\]
	Visto che $ \ker f = C_n \lhd D_n $, il sottogruppo $ H $ viene scomposto in nucleo e laterale
	\[ H = f^{-1}(0) \sqcup f^{-1}(1) = (H \cap C_n) \sqcup \tau (H\cap C_n) \]
	Infine, dato che il nucleo è contenuto nel sottogruppo ciclico $ (H \cap C_n) < C_n $, possiamo pensarlo come il sottogruppo generato da una potenza della rotazione elementare \[ H \cap C_n = \langle \rho^d \colon d \mid n \rangle \] Il suo unico laterale sarà allora composto dagli $ d $ elementi della forma  \begin{align*}\tau(H\cap C_n) &=  \{\tau\rho^d, \tau\rho^{2d}, \dots, \tau\rho^{n-d} \} \\ &= \{\sigma\rho^{d+i}, \sigma\rho^{2d+i}, \dots, \sigma\rho^{n-d +i} \}\end{align*}  che è facile convincersi dipendere solamente dalla classe di $ i \mod d $.
\end{proof}

\textbf{Esercizi.}
\begin{enumerate}
	\item Quali sottogruppi di $ D_n $ sono normali?
	\item Quali sottogruppi di $ D_n $ sono caratteristici?
	\item Quali sono i quozienti di $ D_n $?
	\item $ (\star) $ Chi è $ \Aut{D_n} $?
\end{enumerate}

\end{multicols}

\subsection{Gruppi di Permutazioni $ \mathcal{S}_n $}
\begin{multicols}{2}
\begin{definition}[Gruppi di Permutazioni]
	Dato un insieme $ X $, chiamiamo
	\[  \mathcal{S}(X) = \{ f: X \to X \mid f  \text{ è bigettiva}  \}  \]
	con l'operazione di composizione, il gruppo delle permutazioni di $ X $. Se l'insieme è finito $ |X| = n $, allora
	\[  \mathcal{S}(X) \cong S(\{ 1 ,2, \dots, n \})  \]
	lo chiamiamo $ \mathcal{S}_n $.
\end{definition}

\begin{theorem}
	Ogni permutazione $ \sigma \in \mathcal{S}_n $ si scrive in modo unico come prodotto di cicli disgiunti.
\end{theorem}
\begin{remark}
	Cicli disgiunti commutano.
\end{remark}

%\begin{remark} Questa cosa non vuol dire niente!
%	I cicli sono orbite dell'applicazione di immersione. 
%\end{remark}

\begin{remark}
	$ \mathcal{S}_n $ è generato dai suoi cicli.
\end{remark}

\textbf{Esercizi.}
\begin{enumerate}
	\item Quanti $ k $-cicli ci sono in $ \mathcal{S}_n $?
	\item Come conto gli elementi con una composizione fissata in un $ \mathcal{S}_n $ dato? Per esempio, come calcolo le permutazioni del tipo $ 3+3+2+2+2 $ in $ \mathcal{S}_{10} $?
	\item L'ordine di $ \sigma $ è il minimo comune multiplo delle lunghezze dei suoi $ k $-cicli.
\end{enumerate}

\begin{remark}
	$ \mathcal{S}_n $ è generato dalle sue trasposizioni.
\end{remark}
\begin{remark}
	La decomposizione in trasposizioni non è unica. Ma la parità del numeri di trasposizioni lo è:
\end{remark}
\begin{theorem}[delle trasposizioni]
	La parità del numero di trasposizioni della scomposizione di una qualunque permutazione $ \sigma \in \mathcal{S}_n $ non dipende dalla scomposizione.
\end{theorem}
\begin{proof}
	Consideriamo
	\fun{\text{sgn}}{\mathcal{S}_n}{\mathbb{Z}^\times = \{\pm 1\}}{\sigma}{\prod_{1 \leq i < j \leq n}\frac{\sigma(i)-\sigma(j)}{i-j}}
	questo è un omomorfismo di gruppi. Infatti:
	\begin{enumerate}
		\item è ben definito, ovvero $ |\text{sgn}(\sigma)| = 1 $: tutte le differenze che compaiono a denominatore compaiono anche a numeratore, poiché $ \sigma $ è una permutazione, magari con ordine o segno, differente.
		\item Si comporta bene con la composizione
		\begin{align*}
			\text{sgn}(\sigma \circ \tau) &= \prod_{i < j}\frac{\sigma(\tau(i))-\sigma(\tau(j))}{i-j} \\
			& = \prod_{i < j}\frac{\sigma(\tau(i))-\sigma(\tau(j))}{\tau(i)-\tau(j)}\cdot \frac{\tau(i)-\tau(j)}{i-j} \\
			& = \prod_{i < j}\frac{\sigma(\tau(i))-\sigma(\tau(j))}{\tau(i)-\tau(j)}\cdot\prod_{i < j} \frac{\tau(i)-\tau(j)}{i-j}\\
			& = \text{sgn}(\sigma)\cdot \text{sgn}(\tau)
		\end{align*}
	\end{enumerate}
	Per concludere, osserviamo che tutte le trasposizioni hanno segno negativo. 
\end{proof}
\begin{definition}[Gruppo Alterno]
	Chiamiamo $ \mathcal{A}_n $ o \emph{gruppo alterno} il sottogruppo delle permutazioni pari
	\[ \ker(\text{sgn}) = \mathcal{A}_n \lhd \mathcal{S}_n \]
\end{definition}
\begin{remark}
	Ogni sottogruppo $ H < \mathcal{S}_n $ è contenuto interamente in $ \mathcal{A}_n $ o viene spezzato a metà dal gruppo alterno. Questo perché possiamo restringere il determinante al solo $ H $, che dunque può spezzarsi in nucleo e laterale, o immergersi interamente nel nucleo.
\end{remark}

\begin{theorem}[delle permutazioni coniugate]
	Due permutazioni $ \sigma, \tau \in \mathcal{S}_n $ sono coniugate se e solo se hanno lo stesso tipo di decomposizione in cicli.
\end{theorem}
\begin{proof}
	E' più che sufficiente osservare che dato un ciclo $$  \sigma = (a_1 \; \cdots \; a_k)  $$ e una permutazione tale che $ \tau : a_i \mapsto b_i $ si ha
	\[\tau\sigma\tau^{-1} = (\tau(a_1) \; \cdots \; \tau(a_k)) = (b_1 \; \cdots \; b_k) \]
\end{proof}

\begin{remark}
	Notiamo che il centralizzatore di $ \sigma $ coincide con lo stabilizzatore dell'azione di coniugio di $ \mathcal{S}_n $ in se. Dunque
	\[ |Z(\sigma)| = \frac{n!}{|\mathcal{C}(x)|} \]
\end{remark}
\textbf{Esercizio.} Data una permutazione $ \sigma $ trovare $ |N(\langle \sigma \rangle)| $.

Osserviamo che
\[ N(\gen{\sigma}) = \{ \tau \mid \tau\sigma\tau^{-1} = \sigma^k \} \]
dunque il normalizzatore contiene il centralizzatore di $ \sigma $ e, visto che il coniugio preserva la scomposizione in cicli, che possiamo prendere solo i $ k $ coprimi coll'ordine di $ \sigma $. Inoltre prese due permutazioni $ \tau_1, \tau_2 \in N(\gen{\sigma})$ che generano lo stesso $ \sigma^k $, abbiamo che
\[ \tau_1 \sigma \tau_1 ^{-1} = \tau_2 \sigma \tau_2 ^{-1} \quad\Leftrightarrow\quad (\tau_2^{-1}\tau_1) \sigma (\tau_2^{-1}\tau_1) ^{-1} = \sigma \]
Dunque $ \tau_2^{-1}\tau_1 \in Z(\sigma) $, ovvero $ \tau_1 \in \tau_2 Z(\sigma) $. Pertanto il normalizzatore dev'essere composto da tutti i laterali del centralizzatore indotti da permutazioni che mi danno $ \sigma^k $ dello stesso tipo di $ \sigma $.
Ovvero
\[ N(\gen{\sigma}) = \bigcup_{(i, ord(\sigma))= 1} \tau_i Z(\sigma) \]
e pertanto
\[ |N(\gen{\sigma})| = |Z(\sigma)| \cdot \phi(\text{ord}(\sigma)) \]

\begin{theorem}
	Il gruppo alterno $ \mathcal{A}_n $ è semplice per $ n \geq 5 $.
\end{theorem}
\begin{proof}
	Iniziamo a dimostrare che $ \mathcal{A}_5 $ è semplice, prima o poi concluderemo per induzione. Le classi di coniugio di questo gruppo hanno dimensione 1, 12, 12, 15, 20. Un gruppo normale dovrebbe essere unione di un certo numero di queste orbite, ma sommando le loro cardinalità non ottengo mai divisori di 60.
\end{proof}

\begin{theorem}[del ribelle]
	Per ogni $ n \geq 3 $ i gruppi di permutazioni sono i propri automorfismi $$  \Aut{\S_n} \cong \S_n  $$ ma non per $ \S_6 $, che è un ribelle.
\end{theorem}
\begin{proof}
	Per cominciare, osserviamo che
	\[ \Int{\S_n} \cong \faktor{\S_n}{Z(S_n)} \cong \S_n \]
	dunque $ \S_n \lhd \Aut{\S_n} $, pertanto ci basta dimostrare che tutti gli automorfismi sono interni. Sfruttiamo il fatto che gli automorfismi mandano classi di coniugio in classi di coniugio e che preservano l'ordine degli elementi per mostrare che
	\begin{itemize}
		\item  Ogni automorfismo rispetta la classe delle trasposizioni. Guardiamo come sono fatte le classi di coniugio degli elementi di ordine 2. Sia $ C_k $ la classe delle permutazioni composte da $ k $ trasposizioni disgiunte, abbiamo che
		\[ |C_k|  = \frac{1}{2^k \cdot k!}\cdot \frac{n!}{(n-2k)!} \]
		Imporre $ |C_1| = |C_k| $, equivale a risolvere
		\[ 2^{k-1} = \frac{(n-2)!}{(n-k)!}{n-k \choose k } \]
		\begin{itemize}
			\item Per $ \boxed{k = 2} $ abbiamo
			\[ 4 = (n-2)(n-3) \]
			che non ha soluzione, perché i due fattori a destra sono interi consecutivi, quindi uno di loro sarà dispari.
			\item Per $ \boxed{k = 3} $ abbiamo
			\[ 4 = (n-2) {n-3 \choose 3} \]
			che ammette $ n = 6 $ come unica soluzione (il caso famigerato!), visto che $ n-2 \mid 4 $ limita la ricerca delle soluzioni ai soli $ n = 3, 4, 6 $.
			\item Infine, per $ \boxed{k > 3} $ notiamo che il fattore 
			\[ \frac{(n-2)!}{(n-k)!} \]
			contiene almeno un primo dispari.
		\end{itemize}
		Pertanto ogni automorfismo rispetta $ C_1 $, perché le altre classi sono troppo grandi. 
		\item Osserviamo che, preso $ \varphi \in \Aut{\S_n} $, si ha
		$$  \varphi(1, i) = (a_1, a_i)  $$
		con tutti gli $ a_i $ distinti.
		\item Dunque $ \varphi $ coincide, in realtà, con il coniugio per la permutazione
		$$  \sigma : i \mapsto a_i  $$
	\end{itemize}
\end{proof}

\end{multicols}
\subsection{Prodotti diretti}

\begin{multicols}{2}



\begin{theorem}[di Struttura]{\label{struttura}}
	Sia $ G $ un gruppo e $ H, K < G $ due sottogruppi. Se
	\begin{enumerate}
		\item $ H \lhd\, G $   e   $ K \lhd\, G $
		\item $ HK = G $
		\item $ H \cup K = \{e\} $
	\end{enumerate}
allora $$  G \cong H \times K  $$
\end{theorem}
\begin{proof}
	Mostriamo innanzitutto che $ hkh^{-1}k^{-1} $ appartiene ad entrambi i sottogruppi, infatti:
	\[ H  \ni h(kh^{-1}k^{-1}) = h(kh^{-1}k^{-1}) = (hkh^{-1})k^{-1} \in K \]
	dunque, per la seconda ipotesi, $$ hkh^{-1}k^{-1} = e $$ quindi gli elementi di un sottogruppo commutano con quelli dell'altro $ hk = kh $.
	
	Consideriamo ora l'isomorfismo \fun{\Phi}{H\times K}{G}{(h,g)}{hg}
	e verifichiamo che
	\begin{enumerate}
		\item è ben definito.
		\item è un omomorfismo: infatti \[ \Phi(hh', kk') = hh'kk' = hkh'k' = \Phi(h, k)\Phi(h',k') \]
		\item è suriettivo per la prima ipotesi.
		\item è iniettivo per la seconda, infatti \[ \ker\Phi = \{ (h, k) \mid hk = e \} = \{ (e, e) \} \]
	\end{enumerate}
\end{proof}

\begin{remark}
	Nel prodotto diretto i fattori commutano.
\end{remark}

\textbf{Proprietà di $ G = H \times K $.}
\begin{enumerate}
	\item $ Z(G) = Z(H) \times Z(K) $.
	\item $ \Int{G} \cong \Int{H} \times \Int{K} $.
	\item $ \Aut{H} \times \Aut{K} < \Aut{G} $.
\end{enumerate}
\begin{theorem}[degli automorfismi prodotto]
	Si ha $$  \Aut{H} \times \Aut{K} < \Aut{H \times K}  $$ e sono isomorfi se e solo se $ H $ e $ K $ sono caratteristici.
\end{theorem}
\begin{proof}
	Consideriamo l'omomorfismo \fun{\Phi}{\Aut{H} \times \Aut{K}}{\Aut{H \times K}}{(f, g)}{\varphi_{fg}: (h, k) \mapsto \left(f(h), g(k)\right)}
	e verifichiamo che
	\begin{itemize}
		\item è bene definito, ovvero $ \varphi $ è un automorfismo. Immediata conseguenza del fatto che $ f $ e $ g $ sono a loro volta automorfismi.
		\item è un omomorfismo.  \begin{align*}
			\Phi(ff',gg') &= \left( f(f'(h)), g(g'(k)) \right)\\& = \varphi_{fg}(\varphi_{f'g'}(h, k)) \\&= \Phi(f, g)\Phi(f',g') 
		\end{align*}
		\item è iniettivo.
		\[ \ker\Phi = \{ (id, id) \} \]
		altrimenti c'è almeno un elemento di uno dei due gruppi che non va in se stesso.
		
		\item è suriettivo se e solo se $ H\times\{e_K\} $ e $ \{e_H\}\times K $ sono caratteristici in $ H \times K $.
		
		$ \Rightarrow $. Se $ \Phi $ è suriettivo, allora tutti gli automorfismi di $ H \times K $ sono della forma di cui sopra e pertanto $ \varphi_{fg} $ agisce sugli elementi di $ H $ come $ \varphi_{fg|H} = f \in \Aut{H} $.
		
		$ \Leftarrow $. Viceversa, supponiamo $ H $ e $ K $ caratteristici, preso un automorfismo $ \varphi \in \Aut{H \times K}$ consideriamo le sue restrizioni ai due sottogruppi caratteristici.
		\[ f = \Pi_H \left( \varphi_{|H\times\{e_K\}} \right) \qquad g = \Pi_K \left( \varphi_{|\{e_H\}\times K} \right) \]
		Notiamo che $ f \in \Aut{H} $.
		\begin{itemize}
			\item $ f $ è iniettiva. Se $ f(h) = f(h') $ allora \[\Pi_H \left( \varphi(h, e_K) \right) = \Pi_H \left( \varphi(h', e_K) \right) \]
			poiché $ H\times\{e_K\} $ è caratteristico \[ \varphi(h, e_K) = (a, e_K) \qquad   \varphi(h', e_K) = (b, e_K) \]
			ma necessariamente $ a = f(h) $ e $ b = f(h') $, pertanto
			\[ \varphi(a, e_K) = (f(h), e_K) = (f(h'), e_K) = \varphi(b, e_K) \]
			e, visto che $ \varphi $ è iniettivo, $ h = h' $.
			\item $ f $ è suriettiva. Fissiamo un qualunque $ h \in H $. Essendo $ H\times\{e_K\} $ caratteristico, necessariamente la controimmagine di $ (h, e_K) $ è un suo elemento \[ \varphi^{-1}(h, e_K) = (h', e_K) \]
			dunque \[ f(h') =\Pi_H \left( \varphi(h', e_K) \right) = \Pi_H \left( h, e_K \right) = h \]
		\end{itemize}
		Infine osserviamo che $ \Phi(f, g) = \varphi $. Infatti
		\begin{align*} \varphi_{fg}(h, k) &= \left(f(h), g(k)\right)\\& = \left(\Pi_H \left( \varphi(h, e_K) \right), \Pi_K \left( \varphi(e_H, k) \right)\right) \\& =
		\left(\Pi_H \left( \varphi(h, k) \right), \Pi_K \left( \varphi(h, k) \right)\right) \\& =
		 \varphi(h, k) \end{align*}
		 dove la terza uguaglianza segue da
		 \begin{align*}
		 \Pi_H \left( \varphi(h, e_K) \right) &= \Pi_H \left( \varphi(h, e_K) \right)\Pi_H(\varphi(e_H, k)) \\
		 &= \Pi_H \left( \varphi(h, e_K)\varphi(e_H, k)\right) \\&= \Pi_H \left( \varphi(h, k)\right)
		 \end{align*}
	\end{itemize}
\end{proof}

\textbf{Esercizio.} Trovare $ \Aut{\mathbb{Z}_{20} \times \mathbb{Z}_{2} } $.
\end{multicols}

\subsection{Classificazione dei Gruppi di ordine 8}
\[ \begin{tikzcd}
\boxed{\exists g \mid \text{ord}(g) = 8} \arrow{r}{\text{sì}} \arrow[swap]{d}{no} & \mathbb{Z}_8 \\
\boxed{g^2 = e \;\;\forall g \arrow{r}{\text{sì}}} \arrow[swap]{d}{no} & \left(\mathbb{Z}_2\right)^3 \\
\boxed{\varphi_h \mid h \notin C_4 \arrow{d}{\text{-id}}} \arrow{r}{\text{id}} & \mathbb{Z}_2 \times \mathbb{Z}_4  \\
\boxed{\text{ord}(h) = 2} \arrow{r}{\text{sì}} \arrow{d}[swap]{no}  & D_4 \\
Q_8
\end{tikzcd}
\]

\begin{multicols}{2}
	Prendiamo un gruppo $ G $ di ordine 8.
\begin{itemize}
	\item Se esiste un elemento di ordine 8 il gruppo è ciclico e pertanto isomorfo a $ \mathbb{Z}_8 $.
	\item Se $ G $ ha solo elementi di ordine 2, allora è isomorfo a $ \left(\mathbb{Z}_2\right)^3 $. Mostriamo un risultato appena più generale. 
\end{itemize}
\begin{theorem}[dei gruppi solipsisti]
	Se $ |G| $ ha solo elementi di ordine due ed è finito, allora $ G \cong \left(\mathbb{Z}_2\right)^n $.
\end{theorem}
\begin{proof}
	Osserviamo che $ a^2b^2 = e = (ab)^2 = abab $ e, moltiplicando per $ a $ a sinistra e per $ b $ a destra, otteniamo $ ab = ba $ per ogni $ a, b \in G $. Pertanto $ G $ è abeliano. Possiamo ora procedere per induzione sulla dimensione di $ G $. Se $ |G| = 2 $ il risultato è chiaro. Supponiamo ora che sia vero per tutti i gruppi di ordine $ < 2^n $ e supponiamo $ 2^ n \leq |G| < 2^{n+1} $. Quando prendiamo un insieme minimale di $ h < n $ generatori $ \langle g_1, \dots g_h \rangle $ di un sottogruppo di $ H < G $, questo sarà isomorfo a $ \left(\mathbb{Z}_2\right)^h $ per ipotesi induttiva. Prendiamo un elemento $ g \notin H$, abbiamo che $ H $ e $ \langle g \rangle \cong \mathbb{Z}_2 $ sono sottoinsiemi normali e con intersezione banale, pertanto il sottoinsieme \[ \langle g, g_1, \dots g_h \rangle \cong H \times\langle g \rangle \cong \left(\mathbb{Z}_2\right)^h \times \mathbb{Z}_2 \cong \left(\mathbb{Z}_2\right)^{h+1} \]
	per il teorema di struttura \ref{struttura}.
	Così facendo possiamo continuare ad aggiungere elementi fino a prenderli tutti.
\end{proof}
Se $ G $ non ha elementi di ordine $ 8 $ e non hanno tutti ordine $ 2 $, allora esiste un $ g \in G $ di ordine 4 e sia $ C_4 = \langle g \rangle $. Sia $ h \notin C_4 $ e consideriamo l'azione di coniugio di $ h $ su $ C_4 $
\fun{\varphi_g}{C_4}{C_4}{x}{hxh^{-1}}
ben definita perché $ C_4 $, avendo indice $ 2 $, è normale in $ G $. Poiché $ \Aut{\mathbb{Z}_4} \cong \mathbb{Z}_2 $, abbiamo solo due possibilità: $$  \varphi_g =  id^{\pm 1}  $$

\begin{itemize}
	\item $ \left[\varphi_g =  id, \, \text{ord}(h) =2\right] $. Dunque gli elementi di $ C_4 $ commutano con $ h $, l'intersezione tra $ C_4 $ e $ \langle h \rangle $ è banale e il loro prodotto genera $ G $ per ragioni di cardinalità, pertanto \[ G \cong  \langle h \rangle\times C_4  \cong  \mathbb{Z}_2\times\mathbb{Z}_4 \]
	\item $ \left[\varphi_g =  id, \, \text{ord}(h) =4\right] $. Possiamo considerare $ h^2 $ e ricondurci al caso precedente.
	\item $ \left[\varphi_g =  id^{-1}, \, \text{ord}(h) =2\right] $. Abbiamo che $ hgh = g^{-1} $, quindi per la nostra caratterizzazione dei gruppi diedrali \[ G \cong D_4 \]
	\item $ \left[\varphi_g =  id^{-1}, \, \text{ord}(h) =4\right] $. Anche ord$ (gh) = 4 $. Infatti
	\[ e = ghgh = ghgh^{-1}hh = hh \neq e \]
	Dunque abbiamo trovato l'ordine di tutti gli elementi, possiamo costruire un'isomorfismo esplicito con $ Q_8 $.
\end{itemize}
\begin{definition}[Quaternioni]
	Sia $ Q_8 $ l'insieme $ \{ \pm 1, \pm i, \pm j, \pm, k \} $ con l'operazione che soddisfa
	\[ i^2 = j^2 = k^2 = ijk = -1 \]
\end{definition}
\end{multicols}

\subsection{Lemmi vari}

\begin{multicols}{2}
	
\begin{theorem}[del piccolo indice primo]\label{ppp}
	Siano $ G $ un gruppo finito e $ H $ un sottogruppo che ha come indice il più piccolo primo $ p $ che divide $ G $, allora $ H \lhd G $.
\end{theorem}
\begin{proof}
	Consideriamo l'azione di $ G $ sull'insieme $ X $ dei laterali di $ H $ per moltiplicazione a sinistra \fun{\Phi}{G}{\mathcal{S}_p}{g}{\Pi_g : xH \mapsto gxH}
	Osserviamo che 
	\begin{align*}
		g \in \Stab{xH} &\Leftrightarrow gxH = xH \\
		&\Leftrightarrow x^{-1}gx \in H \\
		&\Leftrightarrow g \in xHx^{-1} \\
	\end{align*}
	dunque $ \Stab{xH} = xHx^{-1} $ è il sottogruppo coniugato di $ H $ rispetto ad $ x $. Possiamo ora riscrivere il nucleo come
	\[ \ker\Phi = \bigcap_{x \in G} xHx^{-1} < H \]
	e osservare che, per il Primo Teorema di Omomorfismo
	\[ \Phi ' : \faktor{G}{\ker\Phi} \rightarrow \mathcal{S}_p \]
	è iniettivo e pertanto
	\[ \left| \faktor{G}{\ker\Phi} \right| \mid |\mathcal{S}_p| = p! \]
	ma $ p $ era il più piccolo primo a dividere $ |G| $, quindi non potendo $ \ker\Phi $ coincidere con tutto il gruppo, dovrà essere proprio $ H $. Il che conclude la dimostrazione.
\end{proof}

\begin{remark}
	Sia $ G $ un gruppo abeliano. Sia \fun{\psi_n}{G}{G}{x}{x^n} preso un qualunque automorfismo $ \varphi \in \Aut{G} $ il seguente diagramma è commutativo
	\[ \begin{tikzcd}
	G \arrow{r}{\psi _n} \arrow[swap]{d}{\varphi} & G \arrow{d}{\varphi} \\
	G \arrow{r}{\psi _n} & G
	\end{tikzcd}
	\]
	quindi $ \ker\psi _n $ e $ \psi _n(G) $ sono caratteristici in $ G $. \\
	
\end{remark}
	
	\textbf{Esercizi.}
	\begin{enumerate}
		\item Trova $ \Aut{\mathbb{Z}\times\mathbb{Z}_n} $.
		\item Trova $ \Aut{\mathbb{Z}_2\times\mathbb{Z}_4\times\mathbb{Z}_4} $.
		\item Trova $ \Aut{Q_8 \times D_4} $.
		\item Sia $ G $ un gruppo abeliano finito. Se $ H \lhd G $ è ciclico e lo è anche il loro quoziente, allora anche $ G $ è ciclico.
	\end{enumerate}
\end{multicols}

\subsection{Prodotto Semidiretto}
\begin{multicols}{2}
	\begin{definition}[Prodotto semidiretto]
		Siano $ H, K $ due gruppi e $ \varphi: K \rightarrow \Aut{H} $ un'omomorfismo. Si dice prodotto semidiretto
		\[ H \rtimes_\varphi K \]
		l'insieme dato dal prodotto cartesiano, dotato dell'operazione
		\[ (h, k) \cdot (h', k') = (h \varphi_k(h'), kk') \]
	\end{definition}

\begin{remark}
	Il prodotto semidiretto è un gruppo.
\end{remark}
\begin{remark}
	Il prodotto diretto è un prodotto semidiretto in cui $ \varphi $ manda tutti gli elementi di $ K $ nell'identità su $ H $.
\end{remark}
\begin{remark}
	Sia $ \bar{H} = H \times {e_k} $. Si ha $$  \ker\Pi_K =  \bar{H} \lhd H \rtimes_\varphi K  $$ qualunque sia l'omomorfismo $ \varphi$. Infatti $ \bar{H} $ è il nucleo dell'omomorfismo di proiezione su $ K $.
\end{remark}
\begin{remark}
	Inoltre $ \bar{K}$ se e solo se il prodotto è diretto. $$  \bar{K} \lhd H \rtimes_\varphi K  \;\Leftrightarrow\; \rtimes = \times $$
\end{remark}

\begin{theorem}[di decomposizione]
	Siano $ G $ un gruppo e $ H, K < G $ sottogruppi. Se
	\begin{enumerate}
		\item $ H \lhd\, G $
		\item $ HK = G $
		\item $ H \cap G = \{e\} $
	\end{enumerate}
allora $$  G \cong H \rtimes_\varphi K  $$ dove $ \varphi $ manda $ k $ nella corrispondente azione di coniugio \fun{\varphi}{K}{\Aut{H}}{k}{\varphi_k : h \mapsto hkh^{-1}}
\end{theorem}
\begin{proof}
	Consideriamo \fun{\Phi}{H \rtimes_\varphi K}{G}{(h, k)}{hk}
	questo
	\begin{itemize}
		\item è un omomorfismo, perché \begin{align*}
			\Phi((h, k)(h', k')) &= \Phi(h\varphi_k(h'), kk') \\
			&= \Phi(hkh'k^{-1}, kk') \\
			&= hkh'k^{-1}kk' \\
			&= hkh'k'\\
			&= \Phi(h, k)\,\Phi(h', k')\\
		\end{align*}
		\item è iniettivo e suriettivo per le ipotesi, come nella decomposizione in prodotto diretto.
	\end{itemize}
dunque $ \Phi $ è un isomorfismo come desiderato.
\end{proof}

\textbf{Esempi.}
\begin{enumerate}
	\item $ \mathcal{S}_n \cong A_n \rtimes_\varphi \langle (1 \; 2) \rangle $, con $ \varphi $ di coniugio.
	\item $ D_n \cong \langle \rho \rangle \rtimes_\varphi \langle \sigma \rangle $, con $ \varphi $ di coniugio.
\end{enumerate}

\columnbreak
\textbf{Classificazione dei gruppi di ordine pq.} \\

Se $ p = q $, allora $ |G| = p^2 $, quindi $ G $ è abeliano. Allora necessariamente
\[ G \cong \mathbb{Z}_{p^2}  \qquad \text{oppure} \qquad G \cong \mathbb{Z}_{p}\times\mathbb{Z}_{p} \]
Se $ p < q $, allora ho due elementi $ x, y $ di ordine, rispettivamente, $ p $ e $ q $, che generano relativi gruppi ciclici. Il più grande dei quali sarà normale perché ha indice $ p $ (vedi \ref{ppp}):
$$  \mathbb{Z}_p < G \quad\text{e}\quad \Z_q \lhd G $$
Osserviamo inoltre che i due sottogruppi hanno intersezione banale e pertanto $ \Z_p\Z_q = G $ per ragioni di cardinalità. Quindi
\[ G \cong \Z_q \rtimes_{\varphi} \Z_p \]
dove
\fun{\varphi}{\Z_p}{\Aut{\Z_q}  \cong \mathbb{Z}_q^\times \cong \mathbf{GL}_1(\mathbb{F}_q)}{1}{\varphi_1: x \mapsto kx}

\textbf{Achtung!} Capire com'è fatta questa azione non è semplice. Prima di tutto, siamo passati in notazione additiva, perché i gruppi in questione sono abeliani e cilici. A questo punto è molto comodo pensare a $ \Z_q $ come spazio vettoriale unidimensionale e al suo gruppo di automorfismi come l'insieme delle "matrici" invertibili che vi agisce sopra, ovvero l'insieme degli elementi invertibili di $ \mathbb{F}_p $. Infine abbiamo definito l'omomorfismo solo per il generatore di $ \Z_p $, perché possiamo ottenere gli altri da
\[ y \mapsto \varphi_y : x \mapsto k^yx \]
ed è molto semplice convincersene: stiamo componendo applicazioni lineari, dunque moltiplicando fra loro le corrispondenti matrici.
Non tutte le scelte di $ k \in \Z_q $ sono accettabili però, per esempio se 
$$  p \nmid q-1 = |\Z_q^\times| $$
l'unico omomorfismo $ \varphi $ possibile è quello banale che ci induce
\[ \boxed{G \cong \Z_p \times \Z_q} \]

Se invece 
$ p \mid q -1 $
possiamo scegliere $ k $ come un qualunque generatore del solo sottogruppo di ordine $ p $ di $ \Z_q^\times \cong \Z_{q-1} $. Fissiamo un generatore $ g $, allora tutte le possibili azioni saranno della forma

\fun{\psi^m}{\Z_p}{\Aut{\Z_q}}{1}{\psi^m_{1}: x \mapsto g^mx}
al variare di $ m $ in $ \{1, \dots, p-1\} $.
Osserviamo dunque che

\[ \psi^m_1(x) = g^{m}(x) = \psi^1_m(x) \]


e possiamo dunque costruire la funzione
\fun{\Phi}{\Z_q \rtimes_{\psi^m} \Z_p}{\Z_q \rtimes_{\psi^1} \Z_p}{(x, y)}{(x, my)}
e verificare che è un'isomorfismo:
\begin{itemize}
	\item è un omomorfismo
	\begin{align*} \Phi(x,\, y)\,\Phi(x',\, y') &= (x,\, my)(x',\, my') \\&= (x + \psi_{my}^1(x'), \,my + my')
	\\&= (x + \psi_y^{m}(x'),\, my + my') \\&= \Phi(x + \psi^m_y(x'),\, y + y')  \end{align*} 
	\item è iniettivo: se $ \Phi(h, k) = (e, e) $, allora $ h = e $ e, poiché $ k^m $ è un automorfismo di $ \Z_p $, $ k = e $.
\end{itemize}
pertanto, in questo caso, esiste \emph{un unico} gruppo di ordine $ pq $ non abeliano. 
\[ \boxed{G \cong \Z_q \rtimes Z_p} \]



\end{multicols}

\subsection{Teorema di Sylow}
\begin{multicols}{2}
	\begin{definition}
		Chiamiamo $ p $-sylow ogni $ p $-sottogruppo di ordine massimo. Ovvero $ H < G $, dove $ |G| = p^mn $ con $ (m, n) = 1 $ e $ |H| = p^m $.
	\end{definition}
\begin{theorem}[di Sylow]\label{sylow}
	Sia G un gruppo finito di ordine $ |G| = p^nm  $, dove $ p $ è primo e $ m $ è un intero a lui coprimo: $ (p, m) = 1 $. Allora sappiamo che:
	\begin{itemize}
		\item[$ \exists $.] Per ogni $ 0 \leq \alpha \leq n $, esiste un sottogruppo $ H<G $ di ordine $|H| = p^\alpha $.
		\item[$ \subseteq $.] Ogni $ p $-sottogruppo è incluso in un $ p $-sylow.
		\item[$ \varphi_g $.] Due qualsiasi $ p $-sylow sono coniugati.
		\item[$ n_p $.] Il numero $ n_p $ di $ p $-sylow è congruo a $ 1 \mod{p}$.
	\end{itemize}
\end{theorem}
\begin{proof}
		Dimostriamo pure i punti in ordine
	\begin{itemize}

		\item[$ \exists $.] Fissiamo $ 0 \leq \alpha \leq n $. Sia $ \mathcal{M}_\alpha $ l'insieme di tutti i sottoinsiemi di $ G $ di cardinalità $ p^\alpha $ \[ \mathcal{M}_\alpha = \{ M < G \mid |M| = p^\alpha \} \] possiamo allora calcolarci
		\[ |\mathcal{M}_\alpha| = {p^nm \choose p^\alpha} = p^{n-\alpha}m \prod_{i = 1}^{p^\alpha - 1}\frac{p^nm-i}{p^\alpha -1} \]
		e poiché $ v_p({p^nm-i}) = v_p(i) =  v_p({p^\alpha -i}) $, allora
		$$  v_p\left(\prod_{i = 1}^{p^\alpha - 1}\frac{p^nm-i}{p^\alpha -1}\right) =  0  $$
		concludiamo che
		\[ p^{n-\alpha} \mid\mid \mathcal{M}_\alpha \]
		Consideriamo l'azione di $ G $ su $ \mathcal{M}_\alpha $ data dalla moltiplicazione a sinistra
		\fun{\Phi}{G}{\mathcal{S}(\mathcal{M}_\alpha)}{g}{\psi_g: M \mapsto gM}
		Vogliamo ora mostrare che esiste uno stabilizzatore della cardinalità giusta. Per la solita decomposizione in orbite abbiamo
		\[ |\mathcal{M}_\alpha| = \sum \frac{|G|}{|\Stab{M_i}|} \]
		Per le osservazione sulla cardinalità, deve esistere un'orbita di cardinalità non divisibile per $ p^{n-\alpha +1} $
		\[ \exists i \text{ tale che }  p^{n-\alpha +1}\nmid |\Orb{M_i}| \]
		 Il corrispondente stabilizzatore avrà pertanto cardinalità divisibile almeno per $ p^\alpha $.
		 Ma se fissiamo un elemento $ x \in M_i $ e consideriamo la funzione iniettiva
		 \fun{f}{\Stab{M_i}}{M_i}{y}{xy}
		 scopriamo che lo stabilizzatore non può avere una cardinalità maggiore dell'insieme che stabilizza
		 \[ p^\alpha \mid |\Stab{M_i}| \leq |M_i| = p^\alpha \]
		 ed è dunque il sottogruppo che cercavamo.
		
		\item[$ \subseteq $.] Sia $ H < G $ un $ p $-sottogruppo $ |H| = p^\alpha $ e $ S $ un $ p $-sylow. Consideriamo l'azione di $ H $ sull'insieme $ X $ delle classi laterali di $ S $ per moltiplicazione a sinistra
		\fun{F}{H}{\mathcal{S}(X)}{h}{\psi_h : gS \mapsto hgS}
		Per la decomposizione in orbite
		\[ m=[G:S]=|X| = \sum\frac{|H|}{|\Stab{gS}|} = \sum\frac{p^\alpha}{p^{e_i}} \]
		ma, non potendo $ p $ dividere $ m $, esiste un laterale $ \bar{g}S $ stabilizzato da tutto $ H $. Ovvero
		\[ h\bar{g}S = \bar{g}S \;\Leftrightarrow\; h \in \bar{g}S\bar{g}^{-1} \; \forall h \in H \]
		Dunque $ H \in \bar{g}S\bar{g}^{-1} $, che è il $ p $-sylow cercato.
		
		\item[$ \varphi_g $.] Siano $ A, B $ $ p $-sylow. Per il punto precedente  \[ \exists g \in G \text{ tale che } A < gBg^{-1} \]
		che hanno la stessa cardinalità e pertanto coincidono.
		\item[$ n_p $.] Consideriamo l'azione di coniugio di un $ p $-sylow $ S $ sull'insieme $ Y $ dei suoi coniugati
		\fun{\Psi}{S}{\mathcal{S}(Y)}{g}{\varphi_g: H \mapsto gHg^{-1}}
		Mostriamo che l'orbita di $ S $ è l'unica banale. Infatti se $ H \in Y $ ha orbita banale significa che è stabilizzato da $ S $, dunque che i due commutano e pertanto il loro prodotto è un sottogruppo di $ G $. \[ |HS| = \frac{|H||S|}{|H \cap S|} = \frac{p^{2n}}{|H \cap S|} \mid p^nm \] Necessariamente $ |H \cap S| = p^n $ e dunque $ H = S $. Per una formula ancora mai usata
		\[ n_p = |Y| = \Orb{S} + \sum_{H \neq S}\frac{|S|}{\Stab{H}} \equiv 1 \mod{p} \]
		Per concludere è sufficiente osservare che se $ \Stab{H} \lneq S $ allora l'orbita corrispondente ha cardinalità divisibile per $ p $.
	\end{itemize}
\end{proof}

\begin{remark}
	$ n_p $ è l'indice del normalizzatore di un $ p$-sylow in $ G $. Infatti, estendendo l'azione di coniugio a tutto il gruppo
	\[ n_p = |\Orb{P}| = \frac{|G|}{|\Stab{P}|} = \frac{|G|}{|N(P)|} = [G : N(P)] \]
\end{remark}

\begin{theorem}
	Ogni gruppo $ G $ abeliano finito è prodotto diretto dei suoi $ p $-sylow.
\end{theorem}
\begin{proof}
	Usiamo la nozione additiva e sia $ G = p^nm $ come al solito. Per ogni divisore $ d$ dell'ordine del gruppo sia
	\[ G_d = \ker\psi_d = \{ g \in G \mid dg = 0 \} \]
	Ci è sufficiente mostrare che
	\[ G \cong G_{p^n} \times G_m \]
	Osserviamo innanzitutto che $ G_{p^n} $ è un p-sylow. Dev'essere un $ p $-gruppo perché se $ |G_{p^n}| $ fosse divisibile per un primo $ q $, allora per il Teorema di Cauhcy \ref{cauchy} conterrebbe almeno un elemento di ordine $ q $, contro la sua definizione. A questo punto, dovendo contenere l'unico $ p $-sylow di $ G $  (il coniugio è banale negli abeliani) non può che esserlo.
	Verifichiamo che
	\begin{enumerate}
		\item I due sottogruppi sono normali in $ G $, perché è abeliano.
		\item La loro intersezione è banale, perché tutti gli elementi del $ p $-sylow hanno ordine divisibile per un primo che non divide l'ordine $ m $ dell'altro sottogruppo.
		\item La loro somma è $ G $. Infatti per Bezout esistono interi $ a, b $ tali che
		\[ ap^n + bm = 1 \]
		che moltiplicato per un qualunque elemento di $ g \in G $ diventa
		\[ a(gp^n) + b(gm) = g \]
		Osserviamo che $ gp^n \in G_{m} $, poiché $$  m (gp^n) = (mp^n) g = |G| g = 0  $$
		Analogamente $ gm \in G_{p^n} $ e pertanto la somma dei due sottogruppi contiene $ G $.
	\end{enumerate}
\end{proof}

\textbf{Esercizi.}
\begin{enumerate}
	\item Chi è il 2-sylow di $ \mathcal{S}_4 $?
	\item Chi sono i gruppi di ordine $ 12 $?
\end{enumerate}
\bigskip
\textbf{Classificazione dei gruppi di ordine 12.}

Quanti possono essere i $ p $-sylow?
I 3-sylow sono necessariamente di ordine $ 3 $, pertanto ciclici, e possono essere $ n_3 = 1, 4 $. I 2-sylow sono di ordine $ 4 $, quindi isomorfi a $ \Z_2 \times \Z_2 $ o $ \Z_4 $, e sono $ n_2 = 1, 3 $. Se $ P_3 $ non è normale, allora ne ho 4 copie con intersezione banale e rimane spazio solo per un $ P_2 $, che sarà normale.

Quindi uno tra un 2-sylow e un 3-sylow dev'essere normale, inoltre sono ciclici e avranno intersezione banale e il loro prodotto ha necessariamente cardinalità $ 12 $. Quindi abbiamo scoperto che $ G $ è isomorfo al prodotto semidiretto tra un sylow e l'altro. Analizziamo le varie possibilità

\begin{itemize}
	\item $ \Z_4 \rtimes_\varphi \Z_3 $. Abbiamo
	\[ \varphi : \Z_3 \to \Aut{\Z_4} \cong \Z_2 \]
	che è dunque necessariamente banale e otteniamo
	\[ \boxed{G \cong \Z_4 \times \Z_3} \]
	\item $ \Z_2 \times \Z_2 \rtimes_\varphi \Z_3 $. Abbiamo
	\[ \varphi : \Z_3 \to \Aut{\Z_2\times\Z_2} \cong \mathcal{S}_3 \]
	che avrà immagine nel sottogruppo di ordine $ 3 $, abbiamo quindi l'automorfismo banale, da cui
	\[  \boxed{G \cong \Z_2 \times \Z_2 \times \Z_3} \]
	e quelli associati a $ \sigma $ e $ \sigma^2 $, che vogliamo mostrare indurre lo stesso prodotto. Infatti, scelto un $ \phi $ non banale, possiamo far agire $ G $ sull'insieme dei suoi 3-sylow per coniugio: sia \fun{\Phi}{G}{\mathcal{S}({\text{3-sylow di G}})\cong \mathcal{S}_4}{g}{\varphi_g : H \mapsto gHg^{-1}}
	Osserviamo che $ N(P_3) = P_3 $, per la formula delle classi. Allora $$  \ker\Phi = \bigcap \Stab{H} = \bigcap N(H) = \bigcap H = \{ e \}  $$ dunque $ \Phi $ è iniettivo e mappa $ G $ in un sottogruppo di ordine 12 di $ \mathcal{S}_4 $. Ma l'unico sottogruppo di questa dimensione è $ \mathcal{A}_4 $, quindi entrambi i gruppi generati dal prodotto non diretto sono isomorfi a questo sottogruppo.
	\[ \boxed{G \cong \mathcal{A}_4} \]
	\item $ \Z_3 \rtimes_\varphi \Z_4 $. Abbiamo
	\[ \varphi : \Z_4 \to \Aut{\Z_3} \cong \Z_2 \]
	che dunque può essere solo $ \pm id $. Il caso banale ci restituisce un prodotto diretto, già considerato, l'altro è un gruppo buffo
	\[ \boxed{G \cong \Z_3 \rtimes_{-id} \Z_4} \]
	\item $ \Z_3 \rtimes_\varphi \Z_2 \times \Z_2  $. Abbiamo
	\[ \varphi : \Z_2 \times \Z_2  \to \Aut{\Z_3} \cong \Z_2 \]
	e abbiamo, oltre all'omomorfismo banale, 3 modi di proiettare $ \Z_2 \times \Z_2  $ su un suo fattore. A meno di isomorfismi di $ \Z_2 \times \Z_2  $, $ \Z_3 $ commuta con uno dei fattori e agisce con $ -id $ sull'altro quindi
	\[ \boxed{G \cong \Z_3\times \Z_2 \rtimes_{-id} \Z_2 \cong D_6} \]
\end{itemize}




\end{multicols}

\subsection{Automorfismi di un gruppo buffo}
\begin{multicols}{2}
	Vogliamo scoprire chi è $ \Aut{Q_8 \times D_4} $. Per far questo, possiamo scomporre un qualunque automorfismo $ \varphi $ nelle sue restrizioni ai due termini del prodotto e proiettarli sulle due componenti. Il seguente diagramma magico è molto esplicativo
	\[ \begin{tikzcd}
	Q_8 \arrow[swap, hook]{dr}{} & & & Q_8 \\
	D_4 \arrow[swap, hook]{r}{} & G \arrow{r}{\varphi} & G \arrow[swap]{ur}{\Pi_Q}\arrow[swap]{r}{\Pi_D} & D_4
	\end{tikzcd}\]
	Dunque possiamo scomporre l'automorfismo nei quattro omomorfismi
	\[ \varphi= \left(\begin{matrix}
	\alpha & \beta \\
	\gamma & \delta \\
	\end{matrix}\right) \]
	dove 
	\begin{align*}
		\alpha: Q_8 \rightarrow Q_8 \qquad & \qquad \beta: D_4 \rightarrow Q_8 \\
		\gamma: Q_8 \rightarrow D_4 \qquad & \qquad \delta: D_4 \rightarrow D_4
	\end{align*}
	 Iniziamo ad analizzare i possibili omomorfismi.
	 \begin{itemize}
	 	\item [$ \beta $.] Consideriamo le possibili immagini per dimensione, tra i sottogruppi dei quaternioni:
	 	\begin{itemize}
	 		\item [$\{e\}  $.][$ \checkmark $] Ovviamente abbiamo un omomorfismo banale.
	 		\item [$ \mathbb{Z}_2 $.][$ \checkmark $] Il nucleo dev'essere un sottogruppo di indice 2 e il diedrale ne ha tre: $ \langle \rho \rangle, \langle \rho^2, \sigma \rangle, \langle \rho^2, \sigma\rho \rangle $.
	 		\item [$ \mathbb{Z}_4 $.] L'unico sottogruppo di indice $ 4 $ del diedrale è $ \langle \rho \rangle \cong \mathbb{Z}_4 $ ed è il nucleo di un omomorfismo che uccide i termini di ordine $ 4 $.
	 		\item [$ Q_8 $.] Non è possibile, sarebbe un isomorfismo!
	 	\end{itemize}
	 	Tutti questi omomorfismi preservano necessariamente i centralizzatori, perché l'unico sottogruppo dei quaternioni di ordine $ 2 $ è il centro. Dunque sembrano accettabili tutti gli omomorfismi
	 	\[ \beta: D_4 \to Z(Q_8) \]
	 	
	 	\item [$ \gamma $.] Consideriamo le possibili immagini, per dimensione:
	 	\begin{itemize}
	 		\item [$\{e\}  $.][$ \checkmark $] Ovviamente abbiamo un omomorfismo banale.
	 		\item [$ \mathbb{Z}_2 $.][$ \checkmark $] Il nucleo dev'essere un sottogruppo di indice 2 e i quaterionioni ne hanno tre: $ \langle i \rangle, \langle j \rangle, \langle k \rangle $.
	 		\item [$ \mathbb{Z}_4 $.] L'unico sottogruppo di indice $ 4 $ dei quaternioni è $ \{\pm 1\} $ ed è il nucleo di un omomorfismo che uccide i termini di ordine $ 4 $.
	 		\item [$ \mathbb{Z}_2\times\mathbb{Z}_2 $.] Possiamo mandare i quaternioni in $ \left(\mathbb{Z}_2\right)^3 $ usando i tre omomorfismi con immagine $ \mathbb{Z}_2 $, questo omomorfismo non sarà suriettivo, altrimenti sarebbe un isomorfismo, e ha almeno $ 4 $ elementi nell'immagine, visto che gli omomorfismi di sopra sono distinti. Quindi, permutando le componenti opportunamente, otteniamo $ 6 $ omomorfismi.
	 		\item [$ D_4 $.] Non è possibile, sarebbe un isomorfismo!
	 	\end{itemize}
	 	possiamo però escludere alcuni omomorfismi osservando che l'automorfismo $ \varphi $ deve preservare i centralizzatori. Infatti osservando il magico diagramma
	 	\begin{align*}
	 		Q_8 &\hookrightarrow Q_8 \times D_4 &\rightarrow Q_8 \times D_4 \\
	 		i &\mapsto (i, e) &\mapsto (\alpha(i), \gamma(i))
	 	\end{align*}
		 scopriamo che $ Z(i, e) \cong \mathbb{Z}_4 \times D_4 $. Possiamo ora cercare di capire cosa dovrebbe essere $ Z(\alpha(i))\times Z(\gamma(i)) $, per esempio elencando i possibili prodotti di sottogruppi di ordine $ 32 $
		 \begin{itemize}
		 	\item [$ Q_8 \times \left(\mathbb{Z}_2\right)^2 $.]
		 	Che però ha solo $ 25 $ elementi di ordine $ 2 $.
		 	\item [$ Q_8 \times \mathbb{Z}_4 $.]
		 	Che ha sol $ 11 $ elementi di ordine $ 2 $.
		 	\item [$ \mathbb{Z}_4 \times \left(\mathbb{Z}_2\right)^2 $.] Che però è abeliano.
		 	\item [$ \mathbb{Z}_4 \times \mathbb{Z}_4 $.] Che è abeliano.
		 	\item [$ \mathbb{Z}_4 \times D_4$.][$ \checkmark $] Che sicuramente è il gruppo che cerchiamo.
		 \end{itemize}
	 quindi necessariamente il centralizzatore di $ Z(\gamma(i)) \cong D_4 $ e pertanto $ \gamma(i) $ è un elemento del centro di $ D_4 $, che ha solo due elementi. Quindi gli omomorfismi $ \gamma $ accettabili sono solo quello banale e i tre che hanno immagine in $ \mathbb{Z}_2 $. Dunque sembrano accettabili tutti gli omomorfismi
	 \[ \beta: Q_8 \to Z(D_4) \]
	 
	 \item [$ \alpha $.] Dev'essere un isomorfismo. Se non fosse un'isomorfismo l'immagine non potrebbe avere dimensione $ 4 $, perché come già visto i sottogruppi di indice adatto eliminano gli elementi di ordine $ 4 $, e non potrebbe avere dimensione più piccola, perché altrimenti il primo termine dell'immagine di $ \varphi $ apparterrebbe sempre al centro di $ G $.
	 
	 \item [$ \delta $.] Analogamente dev'essere un isomorfismo. 
	\end{itemize}
Mostriamo ora che le condizioni trovate sono sufficienti. Ci basta mostrare che $ \varphi $, costruito con le componenti sopra trovate, è iniettivo. Supponiamo di aver trovato $ (x, y) \in G $ tale che
\[ \varphi(x, y) = (\alpha(x)\beta(y), \gamma(x)\delta(y)) = (e, e) \]
Visto che $ \beta(y) $ e $ \gamma(x) $ stanno nei centri dei rispettivi insiemi, anche $ \alpha(x) $ e $ \delta(y) $, che sono i loro inversi, vi staranno. Ma $ \alpha $ e $ \delta $ sono isomorfismi, pertanto anche $ x, y $ staranno nei centri dei loro rispettivi gruppi! Ma $ \beta $ e $ \gamma $ contengono i centri nei loro nuclei, quindi si annullano, così come i rispettivi isomorfismi. Così $ x, y $ sono necessariamente l'elemento neutro del proprio gruppo e $ \ker\varphi = \{(e, e)\} $. \\

Conosciamo già $ \Aut{D_4} $, cerchiamo, per concludere, di capire chi sia{\tiny } $ \Aut{Q_8} $. \\

Ogni automorfismo $ \alpha $ di $ Q_8 $ deve mandare $ \alpha(-x) = -\alpha(x) $, quindi le coppie 
\[ (i, -i) \quad (j, -j) \quad (k, -k) \]
non vengono scisse, ma solo permutate fra loro. Possiamo quindi far agire $ \Aut{Q_8} $ sull'insieme di queste tre coppie, costruendo così un'omomorfismo
\[ \xi : \Aut{Q_8} \to \mathcal{S}_3 \]
Il nucleo di $ \xi $ è costituito dagli automorfismi che non scambiano nessuna coppia, dunque quello identico e i tre che cambiano segni a due delle coppie, ed è dunque isomorfo a $ \Z_2 \times \Z_2 $.

Se consideriamo ora gli isomorfismi
\[ S: \begin{cases}
i \mapsto j \\ j \mapsto i \\ k \mapsto k
\end{cases}
\qquad T: \begin{cases}
i \mapsto j\\
j \mapsto k\\
k \mapsto i
\end{cases} \]


questi generano un sottogruppo "disgiunto" da $ \Z_2 \times \Z_2 $ isomorfo a $ \mathcal{S}_3 $, quindi
\[ \Aut{Q_8} \cong \Z_2 \times \Z_2 \rtimes_\phi \mathcal{S}_3 \] Per una certa azione $ \phi $ che rende il gruppo $ \mathcal{S}_4 $ (per ragioni magiche non dimostrate).




\end{multicols}


\subsection{Teorema Fondamentale dei Gruppi Abeliani Finiti}
\begin{multicols}{2}
\begin{theorem}[di Struttura dei Gruppi Abeliani Finiti]\label{tfgaf}
	Se $ G $ è un gruppo abeliano finito allora si decompone in modo unico come prodotto diretto di gruppi ciclici di ordine $ n_1, \dots, n_s $ \[ G \cong \mathbb{Z}_{n_1} \times \cdots \times \mathbb{Z}_{n_s} \] con $ n_1 \mid \dots \mid n_s $.
\end{theorem}
\begin{proof}
	Avendo già dimostrato che ogni gruppo abeliano finito si decompone nel prodotto dei suoi $ p $-sylow ci è sufficiente dimostrare la tesi per i $ p $-gruppi. Dato gruppo abeliano $ G $ di ordine $ p^n $, ci basta mostrare che possiamo scriverlo come prodotto diretto del generato da un suo elemento di ordine massimo $ g $ e un altro sottogruppo $ K $
	\[ G \cong \langle g \rangle \times K \]
	così da poter procedere per induzione. \\
	
	Mostriamo questo risultato intermedio per induzione sull'ordine del $ p $-gruppo $ G $. Se $ |G|=p $ allora il gruppo è ciclico ed è generato da $ g $. Supponiamo ora la tesi vera per ogni $ k $ con $ 1 \leq k < n $ e prendiamo $ g $ un elemento di ordine massimo, diciamo $ p^m $. Prendiamo ora un elemento $ h \in G  $ che non stia nel sottogruppo $ \langle g \rangle $ e in modo che abbia ordine minimo possibile, se non esiste abbiamo $ G = \langle g \rangle $ e abbiamo finito. \\
	
	Vogliamo ora mostrare che 
	\[ \langle g \rangle \cap \langle h \rangle = \{e\} \]
	L'ordine di $ h^p $ è ovviamente minore di quello di $ h $, dunque $ h^p \in \langle g \rangle $, ovvero esiste un intero $ r \in \mathbb{Z} $ tale che
	\[ h^p = g^r \]
	L'ordine di $ g^r $ è al più $ p^{m-1} $, per tanto non è un generatore di $ \langle g \rangle $, dunque per un qualche intero $ s $ abbiamo
	\[ h^p = g^r = g^{ps} \]
	e succede che 
	\[ (g^{-s}h)^p = g^{-sp}h^p = e \]
	esiste un elemento di ordine $ p $ che non appartiene a $ \langle g \rangle $! Quindi anche l'ordine di $ h $ è $ p $ e i due sottogruppi devono essere disgiunti. \\
	
	Osserviamo ora che, detto $ H = \langle h \rangle $, l'ordine di $ gH $ in $ \faktor{G}{H} $ è lo stesso di $ g $ in $ G $, in particolare è ancora massimo. Se fosse più piccolo, sarebbe al più $ p^{m-1} $ e
	\[ H = (gH)^{p^{m-1}} = g^{p^{m-1}}H \]
	e pertanto $ g^{p^{m-1}} \in H $, assurdo.
	 Per l'ipotesi induttive e il teorema di corrispondenza
	\[ \faktor{G}{H} \cong \langle gH \rangle \times \faktor{K}{H} \]
	per un certo sottogruppo $ H < K < G $. Mostriamo che $ K $ è il sottogruppo che cercavamo
	\begin{itemize}
		\item $ \langle g \rangle \cap K = \{e\} $. Infatti se $ b $ stesse nell'intersezione, $ bH $ apparterrebbe all'intersezione $ \langle gH \rangle \cap \faktor{K}{H} $ che è $ H $, dunque $ b \in H $.
		\item $ G = \langle g \rangle K $. Per ragioni di cardinalità. 
	\end{itemize}

L'unicità è lasciata al lettore.

\end{proof}
	
\textbf{Classificazione dei Gruppi di Ordine 30.}\\
Tiriamo a caso qualche gruppo di quest'ordine
\[ \boxed{\mathbb{Z}_2\times\mathbb{Z}_3\times\mathbb{Z}_5} \qquad \boxed{D_{15}} \qquad \boxed{D_5\times\mathbb{Z}_3} \qquad \boxed{D_{3}\times\mathbb{Z}_5} \]
questi sono distinti perché il primo è l'unico abeliano e i centri di dei seguenti sono rispettivamente $ \{e\},\, \mathbb{Z}_3,\, \mathbb{Z}_5 $. Sappiamo che
\[ n_5 \equiv 1 \mod{5} \qquad\text{e}\qquad n_5 \mid 6 \]
per il Teorema di Sylow \ref{sylow} e perché $ n_5 \mid |G| $ in quanto cardinalità dell'orbita dell'azione di coniugio, rispettivamente. E, analogamente
\[ n_3 \equiv 1 \mod{3} \qquad\text{e}\qquad n_3 \mid 10 \]
Allora, se $ P_5 $ non è normale, ci sono sei 5-sylow, quindi 24 elementi di ordine 5. Tra i pochi elementi che rimangono non ci stanno sicuramente dieci 3-sylow e pertanto $ P_3 $ è normale. Allora $ P_3 $ e $ P_5 $ commutano (perché uno dei due è contenuto nel normalizzatore dell'altro), dunque
\[ P_3P_5 < G \]
e avendo indice $ 2 $ è normale, nonché ciclico.

Abbiamo allora che
\[ G \cong \mathbb{Z}_{15} \rtimes_\varphi \mathbb{Z}_2 \]
per una qualche azione di coniugio 
\fun{\varphi}{\mathbb{Z}_2}{\Aut{\mathbb{Z}_{15}}\cong \mathbb{Z}_5^\times\times\mathbb{Z}_3^\times}{y}{\varphi_y: x \mapsto yxy^{-1} = x^a}
sapendo che $ \varphi_y^2(x) = x^{a^2} = x $, dobbiamo avere che $$  a^2 \equiv 1 \mod{15}  $$ e risolvendo il sistema di diofantee troviamo
\[ a = \pm 1,\, \pm 4 \mod{15} \]
e ognuna di questa azioni induce un prodotto semidiretto isomorfo a uno dei gruppi trovati all'inizio. In particolare $ a = 1 $ è l'automorfismo identico, che induce il prodotto diretto, che restituisce il gruppo abeliano, mentre $ a = -1 $ sappiamo già essere l'omomorfismo che genera il gruppo diedrale. Per $ a = 4 $ troviamo l'automorfismo che fissa $ \mathbb{Z}_3 $, per $ a = -4 $ quello che fissa $ \mathbb{Z}_5 $, in entrambi i casi uno dei fattori a sinistra del prodotto semidiretto commuta anche col fattore di destra, siamo così autorizzati a raccoglierlo all'esterno per ottenere, rispettivamente, $ D_5\times\mathbb{Z}_3 $ e $ D_{3}\times\Z_5 $.
	
Se invece $ P_5 $ fosse normale?
	
\end{multicols}

\subsection{Lemmini ed esercizietti}
\begin{multicols}{2}
	\begin{enumerate}
		\item Dati $$  H \lhd K \lhd G  $$ quali inclusioni devono essere caratteristiche per far si che $ H $ sia normale o carattertico in $ G $?
		
		\item \textbf{Sia $ G $ un gruppo di ordine $ 2d $, dove $ d $ è dispari. Allora esiste un sottogruppo di indice 2.}
		
		Il Teorema di Cayley ci fornisce l'immersione seguente, per prodotto sinistro
		\begin{align*}
		 \Phi \colon & G \hookrightarrow \S_{2d} \\
		 & g \mapsto \varphi_g \colon x \mapsto gx
		\end{align*}
		Dato un elemento $ g $, conosciamo la decomposizione in cicli $ \varphi_g $, questa dev'essere prodotto di cicli del tipo
		$$  (x \; gx \; \cdots \; g^{m-1}x)  $$
		al variare di $ x $ in un opportuno sistema di rappresentanti e dove $ m $ è l'ordine di $ g $. Per il Teorema di Cauchy esiste un elemento di ordine $ 2 $, che avrà come immagine il prodotto di $ d $ 2-cicli. Pertanto $ G \nless \mathcal{A}_{2d} $ e quindi $$  [G : G \cap \mathcal{A}_{2d}] = 2  $$
		
		\item Sia $ G $ un gruppo finito. Se esiste un sottogruppo $ H < G $ di indice $ n $, allora esiste un sottogruppo normale $ N \lhd G $ di indice divisore di $ n! $.
		
		\item \textbf{Sia $ G $ un gruppo semplice e finito. Se esiste un sottogruppo $ H < G $ di indice $ n $, allora esiste un'immersione di $ G $ in $ \mathcal{A}_n $.}
		
		Facendo agire $ G $ per moltiplicazione sinistra sull'insieme degli $ n $ laterali di $ H $ otteniamo un omomorfismo
		
		\[ \Phi \colon G \to \mathcal{A}_n \]
		
		il cui nucleo dev'essere però banale, per non contraddire la normalità di $ G $. Ovviamente l'omomorfismo non è banale, dunque $ \Phi $ è un'immersione.
		
		\item Un gruppo di ordine 112 non è semplice.
		
		\item Un gruppo di ordine 144 non è semplice.
		
		\item Quanti sono i $ p $-sylow di $ \textbf{GL}_n(\mathbb{F}_p) $?
		
		\item Dato un gruppo di ordine $ |G|= p^3 $
		\begin{enumerate}
			\item dimostrare che $ |Z(G)| = p $.
			\item dimostrare che $ G' = Z(G) $.
			\item contare il numero di classi di coniugio.
		\end{enumerate}
		
		\item In un $ p $-gruppo, il centro di uno stabilizzatore di un elemento non nel centro è più grande del centro di tutto il gruppo.
		
		\item $ \Z_n^\times $ è ciclico se e solo se $ n = 2, 4, p^n, 2p^n $.
	
	\end{enumerate}
	
	
\end{multicols}