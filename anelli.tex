\setcounter{section}{0} % per scegliere la lettera giusta
\section{Anelli}
\subsection{Prime definizioni}
\begin{multicols}{2}
Per \textquotedblleft Anello" si intende un anello commutativo con identità.

\begin{definition}[Divisori di 0]
	Un elemento di un anello $ x \in A $ si dice \emph{divisore di zero} quando
	\[ \exists y \in A \quad\text{    tale che    }\quad yx = 0 \]
	Un anello $ A $ si dice \emph{dominio d'integrità} quando l'unico divisore di zero è lo 0 stesso
	\[ \mathcal{D}:= \{\text{divisori di zero}\} = \{0\} \]
\end{definition}
\begin{definition}[Nilpotenza]
	Un elemento di un anello $ x \in A $ si dice \emph{nilpotente} quando
	 \[ \exists n \in \mathbb{N} \quad\text{    tale che    }\quad x^n = 0 \]
	 Un anello $ A $ si dice \emph{ridotto} quando l'unico elemento nilpotente è 0
	 \[ \mathcal{N}: = \{\text{nilpotenti}\} = \{0\} \]
\end{definition}

\begin{prop}[Prime proprietà]
	Valgono:
	\begin{enumerate}
		\item $ A^\times $ è un gruppo moltiplicativo.
		\item $ A^\times \cap \mathcal{D} = \phi $.
		\item Se $ A $ è finito $ A = A^\times \sqcup \mathcal{D} $.
	\end{enumerate}
\end{prop}
\begin{remark}
	Un dominio d'integrità finito è un campo.
\end{remark}

\subsubsection{Anelli di polinomi}
Dato un anello $ A $, consideriamo il corrispondente anello dei polinomi a coefficienti in $ A $: $$  A[\,x\,] = \{ a_0 + \dots + n x^n \mid a_k \in A \}  $$

\textbf{Quali sono gli elementi nilpotenti di $ A[\,x\,] $ ?} \\
Prendiamo un polinomio nilpotente 
\[ f = a_0 + \dots + a_nx^n \]
e osserviamo che il termine di grado maggiore in $ f^k $ è $ a_n^kx^{nk} $. Quindi, affinché $ f $ sia nilpotente, $ a_n $ dev'essere nilpotente in $ A $ e pertanto $ a_nx^n $ sarà nilpotente in $ A[\, x\,] $. Osserviamo che \emph{la somma di elementi nilpotenti è nilpotente}:
\[ a^n = 0,\, b^m = 0 \quad\Rightarrow\quad (a+b)^{n+m} = 0 \]
per concludere che anche $ f-a_nx^n $ sarà nilpotente, e quindi, iterando
\[ \boxed{f \in \mathcal{N}(A[\,x\,]) \LR a_0, \,\dots,\, a_n \in \mathcal{N}(A)}  \]

La freccia inversa è banale: elevando il polinomio a una potenza sufficientemente alta, otteniamo prodotti di potenze dei coefficienti tali che in ogni prodotto compare almeno un coefficiente con potenza maggiore del suo indice di nilpotenza. \\

\textbf{Quali sono gli elementi invertibili di $ A[\,x\,] $ ?} \\
Procediamo come prima, prendendo un polinomio invertibile
\[ f = a_0 + \dots + a_nx^n \]
e il suo inverso
\[ g = b_0 + \dots + b_mx^m \quad \text{tale che } fg = 1 \]
e osserviamo le relazioni tra i coefficienti di $ fg $ e quelli dei fattori:
\begin{align*}
	1 & = a_0 b_0 \\
	0 & = a_0 b_1 + a_1b_0 \\
	&  \vdots \\
	0 & = a_nb_{m-2} + a_{n-1}b_{m-1}+ a_{n-2}{b_m}\\
	0 & = a_nb_{m-1} + a_{n-1}b_m \\
	0 & = a_nb_m
\end{align*}
Innanzitutto, $ a_0 $ e $ b_0 $ sono invertibili. Moltiplicando la penultima relazione per $ a_n $ otteniamo
\[ 0 = a_n^2b_{m-1} + a_{n}a_{n-1}b_m = a_n^2b_{m-1} \]
moltiplicando la terzultima per $ a_n^2 $
\[ 0 = a_n^3b_{m-2} \]
e iterando
\[ 0 = a_n^{m+1}b_0 \]
ma $ b_0 $ è invertibile, quindi non può essere un divisore di zero, pertanto $ a_n^{m+1} = 0 $. Procedendo come prima
\[ \boxed{f \in A[\,x\,]^\times \LR a_0 \in A^\times \text{ e } f-a_0 \in \mathcal{N}(A[\,x\,])} \]
Per la freccia inversa scriviamo 
\[ f = a_0 + g \quad \text{con } a_0 \in A^\times \text{ e } g \in \mathcal{N}(A[\,x\,])  \]
e consideriamo i polinomi
\[ h_m(x) = a_0^{2m} - a_0^{2m-1}g + a_0^{2m-2}g^2 + \dots + g^{2m}  \]
che moltiplicati per $ f $ restituiscono
\[ fh_m = a_0^{2m+1} + g^{2m+1}  \]
scelto $ m $ maggiore dell'indice di nilpotenza di $ g $, abbiamo che $$  a_0^{-(2m+1)}f h_{m} = 1  $$

\textbf{Quali sono i divisori di zero in $ A[\,x\,] $ ?} \\
Prendiamo un polinomio divisore di zero
\[ f = a_0 + \dots + a_nx^n \]
e un polinomio 
\[ g = b_0 + \dots + b_mx^m \quad \text{tale che } fg = 0 \]
di grado minimo possibile.
Consideriamo, al variare di $ 0 \leq k \leq n $, i prodotti
 $ a_k q $.
Se questi non sono tutti nulli, consideriamo il più grande $ k $ tale che \[  a_kq \neq 0 \]
Allora
\[ 0 = fg = (a_0 + \dots a_kx^k) (b_0 + \dots + b_mx^m) \]
osserviamo che il coefficienti del termine di grado massimo è
\[ a_kb_m = 0 \]
e quindi
\[ p \cdot (a_k q) = 0 \]
ma $ \deg{a_k q} < \deg{q} $, quindi dobbiamo avere necessariamente che $ a_kq = 0 \;\forall k $, quindi che $ a_kb_0 = 0 \;\forall k $, ossia
\[ \boxed{f \in \mathcal{D}(A[\,x\,]) \LR \exists b \neq 0 \in A \mid b a_k = 0 \;\;\forall k} \] 


\subsection{Ideali}
\begin{definition}[ideale]
	Chiamiamo \emph{ideale} un sottogruppo additivo $ I \subseteq A $ che assorbe per moltiplicazione.
\end{definition}

\begin{remark}
	Abbiamo una bella caratterizzazione degli ideali propri
	\[ I \subsetneq A \LR I \cap A^\times = \phi \LR 1 \notin I \]
	(è quasi completamente ovvia guardando la contronominale)
\end{remark}


\begin{remark}
	Un campo $ \mathbb{K} $ ha solo ideali banali.
\end{remark}

\begin{definition}[Omomorfismo]
	Una funzione tra due anelli $ A $ e $ B $ è detta omomorfismo di anelli se
	\begin{enumerate}
		\item è omomorfismo dei rispettivi gruppi additivi.
		\item $ f(ab)=f(a)f(b) $
		\item $ f(1_A)=f(1_B) $
	\end{enumerate}
\end{definition}

\begin{theorem}[di omomorfismo]\label{omoanelli}
	Dato un omomorfismo di anelli $ f $ e un ideale $ I \subseteq \ker f $, esiste ed è unico l'omomorfismo $ \bar{f} $ che fa commutare il seguente diagramma
	\[ \begin{tikzcd}
	A \arrow{r}{f} \arrow[swap]{d}{\pi} & B \\
	\faktor{A}{I} \arrow[swap, dashed]{ru}{\bar{f}}
	\end{tikzcd} \]
\end{theorem}
\begin{proof}
	Come per l'analogo teorema sui gruppi, consideriamo la mappa
	\fun{\bar{f}}{\faktor{A}{I}}{B}{a + I}{f(a)}
	e verifichiamo che è ben definita e un omomorfismo.
\end{proof}
\begin{remark}
	Gli ideali sono tutti e soli i nuclei di omorfismi.
\end{remark}

\begin{prop}\label{immideali}
	Dato un omomorfismo $ f : A \to B $
	\begin{itemize}
		\item La controimmagine $ f^{-1}(J) $ di un ideale $ J \subseteq B $ è un ideale di $ A $.
		\item L'immagine $ f(I) $ di un ideale $ I \subseteq A $ è un ideale di $ B $, se $ f $ è suriettiva.
	\end{itemize}
\end{prop}
\begin{proof}
	Preso $ J \subseteq B $, osserviamo che la sua controimmagine $ f^{-1}(J) $ assorbe per prodotto
	\[ f(af^{-1}(J)) = f(a)J = J \quad\Rightarrow\quad af^{-1}(J) = f^{-1}(J)  \]
	ed è additivamente chiusa
	\[ f(i + j) = f(i) + f(j) \in J \qquad \forall i, \, j \in f^{-1}(J) \]
	
	Assumendo $ f $ suriettiva, per ogni $ b \in B $ peschiamo un elemento $ a $ della controimmagine, allora
	\[ f(aI) = f(a)f(I) = af(I) = f(I) \]
	e la chiusura additiva è analoga.
\end{proof}

\begin{theorem}[di corrispondenza]\label{corr}
	Ogni proiezione $$  \Pi : A \to \faktor{A}{I}  $$ induce una relazione biunivoca tra gli ideali del quoziente e gli ideali di $ A $ che contengono $ I $
	\[ \{ \text{ideali di }\faktor{A}{I} \} \leftrightarrow \{ \text{ideali di $ A $ che contengono $ I $} \} \]
	che preserva:
	\begin{itemize}
		\item l'ordinamento indotto dall'inclusione.
		\item l'indice.
		\item primalità e massimalità.
	\end{itemize}
\end{theorem}
\begin{proof}
	La controimmagine di un ideale del quoziente, per il lemma precedente (\ref{immideali}), è ancora un ideale.
	Inoltre, essendo un ideale, contiene lo $ 0 $, dunque la sua controimmagine contiene $ \ker f = I $. Osserviamo ora che, essendo la proiezione una mappa suriettiva, l'immagine di ogni ideale è ancora un ideale! Pertanto è sufficiente mostrare che ideali diversi, contenenti $ I $, vengono mandati dalla proiezione in ideali diversi. Questo accade perché se un elemento $ a $ di una classe laterale di $ I $ sta in un ideale $ J $, vi appartiene tutta la classe laterale:
	\[ b \in a + I \quad\Rightarrow\quad \exists i \in I \mid b = a + i \quad\Rightarrow\quad b \in J \]
\end{proof}
\begin{remark}
	Vale sempre che
	\[ IJ \subseteq I \cap J \]
	c'è uguaglianza in caso di \emph{comassimalità}
	\[ I + J = A \RR IJ = I \cap J \]
\end{remark}
\begin{proof}
	Dato l'assorbimento per moltiplicazione
	\[ IJ \subseteq J, \quad JI \subseteq I \quad\Rightarrow\quad IJ \subseteq I \cap J  \]
	Se $ I $, $ J $ sono comassimali, allora
	\[ I + J = A \quad\Rightarrow\quad \exists i \in I, \, j \in J \text{ tali che } i + j = 1 \]
	e per ogni elemento $ x \in I \cap J $ si ha che
	\[ x = xi + xj \in IJ \]
	perché entrambi gli addendi appartengono a $ IJ $.
\end{proof}

\begin{theorem}[cinese degli anelli]
	Dati due ideali $ I, J \subseteq A $ comassimali, si ha che
	\[ \faktor{A}{IJ} \cong \faktor{A}{I} \times \faktor{A}{J} \]
\end{theorem}
\begin{proof}
	La mappa
	\fun{f}{A}{\faktor{A}{I} \times \faktor{A}{J}}{a}{(a + I, a + J)}
	è un omomorfismo di nucleo
	\[ \ker f = I \cap J \]
	Inoltre $ f $ è suriettivo se e solo se $ I + J = A $: se $ f $ è suriettivo, prendiamo un elemento $ a $ nella controimmagine di $ (1 + I, J) $, ossia un elemento tale che 
	$ a-1 \in I $ e $ a \in J $.
	Pertanto 
	\[ a-1 = i \quad\Rightarrow\quad 1 = a + i \in I + J \]
	Se invece $ 1 \in I + J $, allora possiamo trovare due elementi tale che \[ 1 = i + j \]
	e scelta una qualunque coppia $ (x + I, y + J) $, abbiamo che
	\[ y + j(x-y) = xj + yi = x + i(y-x) \]
	Dunque, se $ I + J = A $, abbiamo $ \ker f = I \cap J = IJ $ per l'osservazione di sopra.
\end{proof}

\begin{definition}[ideale primo]
	Un ideale $ I \subseteq A $ si dice \emph{primo} quando
	\[ \forall x, y \in A \qquad xy \in I \;\Rightarrow\; x \in I \text{ o } y \in I \]
\end{definition}

\begin{definition}[ideale massimale]
	Un ideale $ I \subseteq A $ si dice \emph{massimale}, se è massimale rispetto all'inclusione tra gli ideali propri di $ A $.
\end{definition}

\begin{theorem}\label{idealequoziente}
	Dato un ideale $ I \subseteq A $, valgono le seguenti relazioni con il rispettivo anello quoziente:
	\[ I \text{ è primo} \quad\Leftrightarrow\quad \faktor{A}{I} \text{ è  un dominio} \]
	\[ I \text{ è massimale} \quad\Leftrightarrow\quad \faktor{A}{I} \text{ è un campo} \]
\end{theorem}

\begin{proof}
	Per la prima proposizione, osserviamo che
	
	\[xy \in I \quad\Leftrightarrow\quad \pi(xy) = 0 
	\quad\Leftrightarrow\quad \pi(x)\pi(y) = 0 \]
	
	allora
	\[ \pi(x)\pi(y) = 0 \quad\Leftrightarrow\quad \pi(x) =0 \text{ o } \pi(y) = 0   \]
	che è equivalente a 
	\[ xy \in I \quad\Leftrightarrow\quad x \in I \text{ o } y \in I  \]
	
	Per la seconda, ci basta invocare il teorema di corrispondenza e ricordare che un anello è anche un campo se e solo se ha solo ideali banali.
\end{proof}

\begin{remark}
	$ A $ è un dominio se e solo se l'ideale $ \{0\} $ è primo, è un campo se e solo se l'ideale $ \{0\} $ è massimale.
\end{remark}
\begin{remark}
	Un ideale massimale è anche primo, perché ogni campo è anche un dominio.
\end{remark}

\[ \begin{tikzcd}[row sep=0.75cm, column sep=0.5 cm]
I \text{ massimale} \arrow{r} \arrow[swap]{d} &  \faktor{A}{I} \text{ campo} \arrow{l} \arrow[swap]{d} \\
I \text{ primo} \arrow{r}  &  \faktor{A}{I} \text{ dominio} \arrow{l}
\end{tikzcd} \]

\begin{prop}[del massimale]\label{idmax}
	Ogni ideale proprio è contenuto in un ideale massimale.
\end{prop}
\begin{proof}
	Prendiamo un ideale proprio $ I $ e consideriamo l'insieme di tutti gli ideali propri che lo contengono
	\[ \mathcal{M} = \{ J \subseteq A \text{ ideale proprio} \mid I \subseteq J \} \]
	Questa famiglia non è vuota, visto che $ I \in \mathcal{M} $, e ogni catena $ C $ di ideali ha 
	\[ X = \bigcup C \]
	come massimale: $ X $ è un ideale perché possiamo testare somme e prodotti tra due elementi nel primo ideale in $ C $ che contiene gli elementi in questione, è un ideale proprio perché se $ 1 \in  X $, allora già sarebbe appartenuto a un ideale di $ C $. La tesi segue per il Lemma di Zorn.
\end{proof}
	

\subsubsection{Prodotto diretto tra anelli} Il prodotto cartesiano tra due anelli $ A \times B $, con le operazioni ovvie, è ancora un anello. Inoltre, se ne nessuno dei due è banale \[ (1, 0)(0, 1) = (0, 0) \] il prodotto non è un dominio. I nilpotenti e gli invertibili nel prodotto sono prodotto dei rispettivi sottoinsiemi:
\[ \mathcal{N}_{A \times B} = \mathcal{N}_A \times \mathcal{N}_B \qquad (A\times B)^\times = A^\times \times B^\times \]

\begin{prop}[Fatto Piacevole]
Gli ideali del prodotto sono tutti e soli i prodotti di ideali dei fattori.
\end{prop}
\begin{proof}
	Prendiamo $ I \subseteq A \times B $ e consideriamo le proiezioni $ \pi_A $ e $ \pi_B $. Queste sono omomorfismi suriettivi, quindi $ \pi_A(I) $ è un ideale di $ A $ e $ \pi_B(I) $ è un ideale di $ B $. Basta ora far vedere che $ \pi_A(I) \times \pi_B(I) \subseteq I $, perché abbiamo già l'inclusione opposta, che è un fatto puramente insiemistico. Se $ (a, y), (x, b) \in I $, deve starci anche $ (a, b) $: ma $ (1, 0)(a, y) = (a, 0) \in I $, analogamente $ (0, b) \in I $, dunque
	\[ (a, b) = (a, 0) + (0, b) \in I \]
\end{proof}
 

\subsubsection{Estensione e Contrazione di Ideali} Dati due anelli, ognuno col proprio ideale $ I \subseteq A $ e $ J  \subseteq B $, e un omomorfismo $ f $ tra i due

\[ \begin{tikzcd}
A \arrow{r}{f}  & B \\
I \arrow[swap, hook]{u}{} & J \arrow[swap, hook]{u}{}
\end{tikzcd} \]

Tenendo a mente la proposizione \ref{immideali}

\begin{definition}
	Chiamiamo ideale contratto $ J^c $ la sua controimmagine $$  J^c = f^{-1}(J)  $$
\end{definition}
\begin{definition}
	Chiamiamo ideale esteso $ I^e $ il generato dalla sua immagine
	\[I^e = \left(f(I)\right) \]
\end{definition}

Poiché possiamo spezzare la mappa $ f $ passando per il quoziente
\[ \begin{tikzcd}
A \arrow{r}{f} \arrow[swap]{d}{\pi} & B \\
\faktor{A}{\ker f} \arrow[swap]{ru}{\varphi}
\end{tikzcd} \]
e il comportamento degli ideali attraverso la proiezione $ \pi $ è completamente determinato dal Teorema di Corrispondenza (\ref{corr}), possiamo limitarci a studiare cosa succede attraverso l'omomorfismo $ \varphi $, che essendo iniettivo possiamo scambiare con l'omomorfismo di inclusione.

Prendiamo quindi $ A \subseteq B $, abbiamo
\begin{enumerate}
	\item $ J^c = J \cap A $
	\item $ I^e = (I) $
\end{enumerate}

\begin{prop}\label{contr}
	La contrazione di un ideale primo è ancora un ideale primo.
\end{prop}
\begin{proof}
	$ J $ è primo se e solo se $ \faktor{B}{J} $ è un dominio. Consideriamo la mappa composizione 
	\[ \varphi : A \hookrightarrow B \to \faktor{B}{J} \]
	questa è un omomorfismo di nucleo
	\[ \ker\varphi = J \cap A = J^c \]
	Dunque per il primo teorema di omomorfismo esiste un omomorfismo iniettivo
	\[ \faktor{A}{J^c} \hookrightarrow \faktor{B}{J} \]
	e i sottoanelli di domini sono a loro volta domini.
\end{proof}

Teoremi analoghi con ideali massimali, o estensioni, non funzionano.

\subsubsection{Fatti sul radicale}
\begin{definition} Dato un ideale $ I \subseteq A $, chiamo radicale l'insieme
	$$  \sqrt{I}:= \{ x \in A \mid x^n \in I \}  $$
	costituito da tutti gli elementi che, elevati a una qualche potenza finita, cadono in $ I $.
\end{definition}

\begin{remark}
Il radicale $ \sqrt{I} $ è un ideale. Infatti è un sottoinsieme additivamente chiuso
\[ a^n \in I,\;  b^m \in I \quad\Rightarrow\quad (a+b)^{n+m} \in I \]
e assorbente per moltiplicazione
\[ a^n \in I \quad\Rightarrow\quad (sa)^{n} = s^na^n \in sI = I \]

Inoltre, ovviamente, contiene l'ideale di partenza: $ I \subseteq \sqrt{I} $.
\end{remark}


\begin{remark}
	$ \sqrt{IJ} = \sqrt{I \cap J} = \sqrt{I} \cap \sqrt{J}  $. Infatti grazie alla seguente implicazione
	\[ I \subseteq J \quad\Rightarrow\quad \sqrt{I} \subseteq \sqrt{J} \]
	possiamo scrivere la catena di inclusioni
	\[ \sqrt{IJ} \subseteq \sqrt{I \cap J} \subseteq \sqrt{I} \cap \sqrt{J} \subseteq \sqrt{IJ} \]
	dove le prime due inclusioni sono l'inclusione dell'ideale di partenza, mentre l'ultima si ottiene da
	\[ a^n \in I, b^m \in J \quad\Rightarrow\quad (ab)^{m + n} \in I \cap J )\]
	
	Da questo segue anche che $ \sqrt{I} = A $ se e solo se $ I = A $.
\end{remark}
\begin{remark}
	Se $ P $ è un ideale primo, allora $ \boxed{\sqrt{P} = P} $. Infatti se $ x^n \in P $, ho $ x = x \cdot x \cdots x $ per $ n $ volte e almeno uno dei fattori sta in $ P $
\end{remark}

\begin{prop}
	L'ideale dei nilpotenti è intersezione di tutti gli ideali primi di $ A $
	$$  \mathcal{N} = \bigcap_{P \subseteq A} P  $$
\end{prop}
\begin{proof}
	Chiaramente se $ x $ è nilpotente $ x ^n \in P $ per ogni ideale primo $ P $.
	
	Mostriamo che, viceversa, se $ x $ appartiene a tutti gli ideali primi allora è nilpotente. Ragioniamo per assurdo: fissato $ x $ nell'intersezione, assumiamo $$  x^n \neq 0 \qquad \forall n \in \mathbb{N} $$ Consideriamo la famiglia di ideali
	$$  \mathcal{F} = \{ I \subseteq A \mid x^n \notin I \;\forall n \in \mathbb{N}  \}  $$
	Osserviamo che per ipotesi $ {0} \in \mathcal{F} $, quindi la famiglia non è vuota. Inoltre è induttiva! (L'unione di una catena di ideali è un ideale e la proprietà di famiglia si conserva). Dunque, per il Lemma di Zorn, abbiamo un ideale $ P $ massimale in $ \mathcal F $.
	
	Mostriamo che $ P $ deve essere primo, giungendo all'assurdo. Preso $ ab \in P $ se uno degli ideali 
	\[ P + (a) \qquad \qquad P + (b) \]
	appartenesse alla famiglia $ \mathcal{F} $, allora sarebbe sottoinsieme di $ P $ (e dunque $ a \in P $). Se nessuno dei due appartenesse alla famiglia, allora
	\[ x^n \in P + (a) \qquad x^m \in P + (b) \]
	ma
	\[ x^{n + m} \in (P + (a))(P + (b)) = P \]
	che dunque dev'essere primo.
\end{proof}
\begin{theorem}
	Ogni radicale è l'intersezione di tutti i primi contenenti l'ideale di partenza
	$$  \sqrt{I} = \bigcap_{I \subseteq P} P  $$
\end{theorem}
\begin{proof}
	Il radicale $ \sqrt{I} $ è costituito dai nilpotenti del quoziente $ \faktor{A}{I} $, la tesi segue dal teorema di corrispondenza.
\end{proof}

\end{multicols}

\subsection{Domini}
\begin{multicols}{2}

\subsubsection{Campo dei quozienti. $ \Q(A) $}

\begin{definition}
	Un insieme $ S $ di elementi di un dominio $ A $ è detto \emph{parte moltiplicativa} se
	\begin{enumerate}
		\item è moltiplicativamente chiuso
		\item $ 1 \in S $
		\item $ 0 \notin S $
	\end{enumerate}
\end{definition}

\begin{remark}
	$ A $ è dominio se e sole se $ A - \{0\} $ è parte moltiplicativa.
\end{remark}

Costruiamo il campo dei quozienti del dominio $ A $ come il quoziente di $ A \times (A \setminus \{0\}) $ per la relazione di equivalenza $$  (a, b) \sim (c, d) \quad\Leftrightarrow\quad ad = bc $$
Denotiamo questo insieme con $ \Q(A) $. Indichiamo i suoi elementi come $ \frac{a}{b} $ e definiamo la somma e il prodotto in modo che
\[ \frac{a}{b} + \frac{c}{d} = \frac{ad + bc}{bd} \qquad\qquad \frac{a}{b}\cdot \frac{c}{d} = \frac{ac}{bd} \]

\begin{remark}
	$ (Q(A),\, +,\, \cdot\,) $ sopra definito è un campo e
	\begin{align*}
	q \colon A &\hookrightarrow \Q(A) \\
	a &\mapsto \frac{a}{1}
	\end{align*}
	è omomorfismo iniettivo.
\end{remark}

\begin{theorem}
	$ (\Q(A),\, +,\, \cdot\,) $ è il più piccolo campo in cui possiamo immergere $ A $. Nel senso che l'immersione di $ A $ in un campo $ \mathbb{K} $ si estende all'immersione di $ \Q(A) $ in $ \mathbb{K} $.
	\[ \begin{tikzcd}
	A \arrow[hook]{r}{\varphi} \arrow[swap, hook]{d}{q} & \K \\
	\Q(A) \arrow[swap, dashed]{ru}{\bar{\varphi}}
	\end{tikzcd} \]
\end{theorem}
\begin{proof}
	L'estensione è naturale
	\[ \bar{\varphi}: \frac{a}{b} \mapsto \varphi(a) \varphi(b)^{-1} \]
	Una volta verificato che è un omomorfismo, questo dev'essere necessariamente iniettivo perché $ \Q(A) $ è un campo, pertanto ha solo ideali banali.
\end{proof}

\columnbreak
\subsubsection{Divisibilità.}
In modo naturale, diciamo che $ a \mid b $ se esiste $ c  $ tale che $ b = ac $. E' immediato tradurre la relazione di divisibilità in termini di ideali generati, infatti 
\[ a \mid b \quad\Leftrightarrow\quad (b) \subseteq (a) \]

\begin{definition}
	Diciamo che due elementi $ a, a' \in A $ sono \emph{associati} quando si dividono vincendevolmente

\[ (a) = (a') \quad\Leftrightarrow\quad a' = au \;\text{ per } u \in A^\times \]

lo indicheremo con $ a \sim a' $.
\end{definition}

Possiamo definire  il \textsc{mcd} tra due elementi $ a,\, b $ come quegli elementi $ d $ tali che
\begin{enumerate}
	\item $ d \mid a \;$   e   $\; d \mid b $.
	\item Se $ c \mid a\; $ e $ \; c \mid b $, allora $ c \mid d $.
\end{enumerate}
è immediato osservare che gli \textsc{mcd} di una coppia sono associati.

\begin{definition}[Primo]
	Diciamo che un elemento non invertibile e non nullo $ x \in A\setminus(A^\times \cup \{0\}) $ è \emph{primo} se \[ x \mid ab \quad\Rightarrow\quad x \mid a \; \text{ o }\; x \mid b \]
\end{definition}

\begin{definition}[Irriducibile]
	Diciamo che un elemento non invertibile e non nullo $ x \in A\setminus(A^\times \cup \{0\}) $ è \emph{irriducibile} se \[ x = ab \quad\Rightarrow\quad a\in A^\times \; \text{ o }\; b \in A^\times \]
\end{definition}

I seguenti risultati sono semplicemente la traduzione delle ultime definizioni nel linguaggio degli ideali:

\begin{theorem}\label{prpr}\label{mama}
	Un elemento $ x $ è primo se e solo se l'ideale da lui generato $ (x) $ è primo e non banale.
	\[ x \text{ è primo} \quad\Leftrightarrow\quad (x) \text{ è primo } \neq \{0\}  \]
	Un elemento $ x $ è massimale se e solo se l'ideale da lui generato $ (x) $ è massimale tra i principali.
	\[ x \text{ è irriducibile} \quad\Leftrightarrow\quad (x) \text{ è massimale tra i principali} \]
\end{theorem}

\begin{theorem}\label{epei}
	Ogni elemento primo è anche irriducibile.
	\[ x \text{ è primo} \quad\Rightarrow\quad x \text{ è irriducibile} \]
\end{theorem}
Riassumiamo in uno schemino tutte le relazioni legate alla divisibilità:
\end{multicols}
\bigskip
\begin{center}
	\begin{tikzpicture}[commutative diagrams/every diagram]
\node (P0) at (90:2.3cm) {$ (x) $ massimale};
\node (P1) at (90+72:2cm) {$ (x) $ primo} ;
\node (P2) at (90+2*72:1.75cm) {\makebox[5ex][r]{$ (x) $ max. tra princ.}};
\node (P3) at (90+3*72:1.75cm) {\makebox[5ex][l]{$x$ irr.}};
\node (P4) at (90+4*72:2cm) {$x$ primo};

\node (A) at (90+4*72-40:4cm) {};
\node (B) at (90+4*72-10:3.8cm) {};
\node (C) at (90+4*72+15:4.5cm) {};

\path[commutative diagrams/.cd, every arrow, every label]
(P0) edge node[swap] {} (P1)
(P1) edge node[swap] {} (P2)
(P1) edge node {} (P4)
(P2) edge node {} (P3)
(P3) edge node {} (P2)
(P4) edge node {} (P3)
(P4) edge node {} (P1)
(P0) edge node {} (P4)

(A) edge node {\textsc{ufd}} (B)
(B) edge node {\textsc{pid}} (C);
\end{tikzpicture}
\end{center}

\subsection{Domini Speciali}
\begin{multicols}{2}
	Diamo un po' di definizioni insieme
	\begin{definition}
		Ci piacciono i seguenti domini speciali:
		\begin{itemize}
			\item [(\textsc{ufd}):] Ogni elemento non nullo e non invertibile si scrive in modo unico come prodotto di irriducibili.
			
			\item [(\textsc{pid}):] Tutti gli ideali sono principali.
			
			\item [(\textsc{ed}):] Esiste una funzione grado \[{d}:{A\setminus\{0\}}\to{\mathbb{N}}\]
			tale che 
			\begin{enumerate}
				\item $ d(a) \leq d(ab) $ su tutti i valori in cui è definita.
				\item Presi comunque due elementi $ a, b $ con $ b \neq 0 $ esistono $ q, r $ tali che
				\[ a = qb + r \]
				e tali che $ d(r) < d(b) $ oppure $ r = 0 $.
			\end{enumerate}
			
		\end{itemize}
	\end{definition}

\begin{theorem}
	Vale la seguente catena di implicazioni
	\[ (\textsc{ed})\Rightarrow(\textsc{pid})\Rightarrow(\textsc{ufd}) \]
\end{theorem}
\begin{proof}
	Segue dai lemmi \ref{abc} e \ref{def}. 
\end{proof}

\textbf{Esempi e Controesempi.}
\begin{enumerate}
	\item $ \Z $ con la funzione $ \mid \cdot\mid $ è un dominio euclideo.
	\item $ \K[\,x\,] $ con la funzione $ \deg(\,\cdot\,) $ è un dominio euclideo.
	\item $ \Z[\,i\,] $ con la norma $ N(x) = x\bar{x} $ è un dominio euclideo.
	\item $ \K[[\,x\,]] $ con la funzione $ \deg(f) = \min \{n \in \mathbb{N} \mid a_n \neq 0 \} $ è un dominio euclideo.
	\item $ \Z[\,x\,] $ è \textsc{ufd} ma non è \textsc{pid}.
	\item $ \Z\left[ \dfrac{1 + \sqrt{-19}}{2} \right] $ è \textsc{pid} ma non \textsc{ed}.
\end{enumerate}
	
\subsubsection{Dominio Euclideo}

\begin{prop}
	Gli elementi invertibili sono tutti e soli quelli di grado minimo.
\end{prop}
\begin{proof}
	Osserviamo innanzitutto che ha senso parlare di grado minimo, poiché l'immagine del grado è un sottoinsieme non vuoto dei naturali.
	
	Se $ x $ è invertibile, per ogni $ a \in A $ non nullo
	\[ d(x) \leq d(x \cdot ax^{-1}) = d(a). \]
	
	Viceversa, sia $ x $ un elemento di grado minimo e $ a $ un elemento qualunque. Abbiamo che
	\[ a = qx + r \]
	e, escludendo $ d(r) < d(x) $, dobbiamo avere che $ r = 0 $. Quindi $ x $ divide ogni elemento ed è pertanto invertibile. 
\end{proof}


\textbf{MCD.} Esiste il \textsc{mcd} e lo possiamo calcolare con l'algoritmo di Euclide.

	
\begin{prop}\label{abc}
	 Abbiamo che $ \textsc{ed}\Rightarrow\textsc{pid}$.
\end{prop}
\begin{proof}
	Prendiamo un ideale $ I $ di un \textsc{ed} e sia $ x $ il suo elemento di grado minimo. Vogliamo dimostrare che 
	\[ I = (x) \]
	per assurdo. Supponiamo che esista $ y \in I $ tale che
	\[ x = qy + r \]
	con $ r \neq 0 $, abbiamo allora necessariamente che $ d(r) < d(x) $. Purtroppo però
	\[ r = x - qy \in I \]
	contraddice la minimalità di $ x $.
\end{proof}

\subsubsection{Dominio a Ideali Principali}

\textbf{MCD.} Il \textsc{mcd}$ (a, b) $ esiste: è il generatore dell'ideale $$  (a, b) = (d)  $$


\begin{theorem}\label{primossemassimale}
	Gli ideali primi sono anche massimali.
\end{theorem}
\begin{proof}
	\begin{align*}
		(x) \text{ primo } & \Rightarrow x \text{ è un elemento primo} & \text{(\ref{prpr})} \\
		& \Rightarrow x \text{ è irriducibile} &\text{(\ref{epei})} \\
		& \Rightarrow (x) \text{ è massiamale tra gli ideali primi} &\text{(\ref{mama})} \\
		& \Rightarrow (x) \text{ è massimale} &\text{(\textsc{pid})}
	\end{align*}
\end{proof}

\begin{remark}
	Ripercorrendo la catena di implicazioni sopra, ci si accorge che abbiamo mostrato anche l'equivalenza tra le definizioni di irriducibile e primo  nei \textsc{pid}.
\end{remark}

\subsubsection{Dominio a Fattorizzazione Unica}

\textbf{MCD.} Negli anelli \textsc{ufd} l'$ \textsc{mcd}(a, b) $ esiste, perché possiamo caratterizzarlo attraverso la fattorizzazione, come sugli interi. Però non è detto che appartenga all'ideale $ (a, b) $. Per esempio: in $ \Z[x] $ abbiamo che $ (2, x) = 1 $ ma non esistono polinomi $ f, g $ tali che
\[ 1 = 2f + xg \]
infatti valutando l'espressione in $ 0 $ otteniamo $ 1 = 2f(0) $.

\begin{theorem}[Caratterizzazione degli \textsc{ufd}] \label{carufd}
	Un dominio A è \textsc{ufd} se e solo se valgono le seguenti condizioni
	\begin{enumerate}
		\item irriducibile $ \Rightarrow $ primo
		\item Ogni catena discendente di divisibilità è stazionaria.
	\end{enumerate}
\end{theorem}

\begin{proof}
	Cominciamo con la freccia più interessante, ovvero che ogni elemento di un anello con le suddette proprietà ammette fattorizzazione unica. \\
	
	\emph{La proprietà 2 ci garantisce l'esistenza della fattorizzazione}. Ragioniamo per assurdo: prendiamo $ x \in A\setminus(A^\times \cup \{0\}) $ e supponiamo che non ammetta fattorizzazione. Non può essere irriducibile, altrimenti avremmo trovato una fattorizzazione, pertanto si può scrivere come
	\[ x = y_0z_0 \quad\text{con } y_0, z_0 \notin A^\times \]
	Però questa non può essere una fattorizzazione, pertanto uno dei due non è né irriducibile né fattorizzabile, diciamo $ y_0 $. Allora lo possiamo scrivere come
	\[ y_0 = y_1 z_1 \]
	e trovandoci nelle stesse ipotesi di prima, possiamo iterare il procedimento indefinitamente, ottenendo
	\[ \cdots \mid y_2 \mid y_1 \mid y_0 \]
	una catena discendente infinita e non stazionaria, assurdo.\\
	
	\emph{La proprietà 1 ci garantisce l'unicità della fattorizzazione}. Procediamo per induzione forte sulla lunghezza minima $ l_x $ della fattorizzazione di un certo elemento $ x $. Se $ l_x = 1 $ e
	\[ p = x = q_1 \cdots q_s \]
	poiché $ p $ è irriducibile, tutti i $ q_i $ tranne uno sono invertibili e quello che si salva non può che essere associato a $ p $. Supponiamo ora che tutti gli elementi per cui $ l_x < n $ abbiamo fattorizzazione unica e prendiamo un elemento $ x $ tale che $ l_x = n $. Allora prese
	\[ p_1 \cdots p_n = x = q_1 \cdots q_s \]
	sappiamo che $ p_n $ è irriducibile e quindi primo, dunque
	\[ p_n \mid q_1 \cdots q_s \quad\Rightarrow\quad \textsc{wlog }\; p_n \mid q_s  \]
	poiché anche $ q_n $ è irriducibile e quindi primo, devono essere associati. Pertanto dobbiamo avere che
	\[ p_1 \cdots p_{n-1} = \tilde{x} = q_1 \cdots q_{s-1} \]
	ma le due fattorizzazione sono uguali per ipotesi induttiva, poiché $ l_{\tilde{x}} \leq n-1 < n $.\\
	
	\emph{Viceversa}, se $ A $ è \textsc{ufd}, presa una catena discendente di divisibilità
	\[ \cdots \mid a_2 \mid a_1 \mid a_0 \]
	possiamo considerare gli insiemi $ A_n $ di tutti i fattori di $ a_n $ presi con molteplicità e osservare che la catena discendente
	\[ \cdots \subseteq A_2 \subseteq A_1 \subseteq A_0 \]
	si stabilizza, perché le cardinalità corrispondenti formano una catena discendente di interi.
	Inoltre, preso un elemento irriducibile $ x $, se
	\[ x \mid ab \]
	un associato di $ x $ deve comparire nella fattorizzazione di almeno uno tra $ a $ e $ b $, quindi $ x $ divide almeno uno dei due e pertanto soddisfa la definizione di elemento primo.
\end{proof}

\begin{prop}\label{def}
	Segue che $ \textsc{pid}\Rightarrow \textsc{ufd} $.
\end{prop}
\begin{proof}
	La prima condizione l'abbiamo già mostrata. Presa una catena discendente di divisibilità, possiamo tradurla in una catena ascendente di ideali
	\[ (a_0) \subseteq (a_1 ) \subseteq (a_2) \subseteq \cdots \]
	Però l'ideale $ I = \bigcup (a_i) $ è generato da un elemento $ a $, perché tutti gli ideali sono principali. Ma per come è definito $ I $, l'elemento $ a $ già apparteneva a un certo $ (a_n) $, generando lui come tutti i suoi successori. 
\end{proof}

\textbf{Esempi.}
\begin{enumerate}
	\item $ \K [x, \sqrt{x}, \sqrt[3]{x}, \cdots ] $ non è \textsc{ufd}. Infatti 
	\[ \cdots \mid \sqrt[8]{x} \mid \sqrt[4]{x} \mid \sqrt[2]{x} \mid x \]
	è una catena discendente infinita di divisibilità.
	
	\item $ \Z[\sqrt{-5}] $ non è \textsc{ufd}. Infatti $ 2 $ è irriducibile ma non primo.
	Se \[ 2 = (a+b\sqrt{-5})(c + d\sqrt{-5}) \] in norma abbiamo una contraddizione, mentre
	\[ (1 + \sqrt{-5})\cdot (1 - \sqrt{-5}) = 2 \cdot 3 \]
	dunque \[ 2 \mid (1 + \sqrt{-5})\cdot (1 - \sqrt{-5})  \] ma
	\[ 4 = N(2) \nmid N(1 \pm \sqrt{-5}) = 6 \]
	
\end{enumerate}

\begin{definition}
	Chiamiamo \emph{contenuto} del polinomio $ f \in A[\,x\,] $ l'\textsc{mcd} dei suoi coefficienti
	\fun{c}{A[\,x\,]}{A}{a_0 + \dots + a_nx^n}{\textsc{mcd}(a_0,\, \dots,\, a_n)}
\end{definition}

\begin{definition}
	Chiamiamo \emph{primitivo} un polinomio $ f $ con contenuto unitario
	\[ c(f) = 1 \]
\end{definition}
\begin{theorem}[Lemma di Gauss]\label{LemmaGauss}
	In ogni anello $ A[\,x\,] $:
	\begin{enumerate}
	\item Il prodotto di polinomi primitivi è primitivo.
	\item La funzione contenuto è completamente moltiplicativa.
	\item Se $ f \in A[\,x\,] $ e $ f = hg $ in $ \Q(A)[\,x\,] $,\\ esistono $ h_1, g_1 \in A[\,x\,] $ tali che $ f = h_1g_1 $.
	\end{enumerate}
	
\end{theorem}
\begin{proof}
	Prendiamo due polinomi $ h, g \in A[\,x\,] $ il cui prodotto non è primitivo e mostriamo che almeno uno dei due non è primitivo. Se $ hg $ non è primitivo, possiamo trovare un primo $ \pi $ di $ A $ tale che
	\[ \pi \mid hg \quad\text{ in } A[\,x\,] \]
	e ridurci a ragionare nell'anello $ \faktor{A}{(\pi)}[\,x\,] $, dove
	\[ \bar{h}\bar{g} = \overline{hg} = 0 \]
	 Dalle considerazioni fatte sui divisori di zero all'inizio del capitolo, segue che l'anello dei polinomi costruito su un dominio è a sua volta un dominio, pertanto possiamo assumere, senza perdita di generalità
	 \[ \bar{h} = 0 \RR \pi \mid h \RR \pi \mid c(h) \]
	 da cui segue il primo punto.
	 
	 Il secondo punto segue subito dal primo, separando ogni polinomio $ f $ in parte primitiva e contenuto:
	 \[ f = c(f) \tilde{f} \]
	 
	 Per il terzo punto, riscriviamo $ g $ e $ h $ facendo minimo comune multiplo, in modo che
	 \[ g = \frac{\tilde{g}}{d}, \quad h = \frac{\tilde{h}}{d'} \quad \text{ con } d, d' \in A \text{ e } \tilde{g}, \tilde{h} \in A[\,x\,]   \]
	 sapendo che
	 \[ \tilde{g}\tilde{h} = dd'f \]
	 visto $ \tilde{g} $ e $ \tilde{h} $ sono primitivi, anche $ dd'f $ lo deve essere e, pertanto, $ dd' \in A^{\times} $. In questo modo i polinomi tildati devono avere coefficienti in $ A $.
\end{proof}
\columnbreak
\begin{theorem}
	$ A $ è \textsc{ufd} $ \Rightarrow $ $ A[\,x\,] $ è \textsc{ufd}
\end{theorem}
\begin{proof}
	Vogliamo utilizzare la caratterizzazione dei domini a fattorizzazione unica (\ref{carufd}). 
	Iniziamo verificando che dato un polinomio $ f \in A[x] $ irriducibile, questo è anche primo. Ovviamente 
	\[ c(f) \mid f \quad\Rightarrow\quad c(f) = f \;\text{ o }\; c(f) = 1 \]
	Se $ c(f) = f $, allora $ f \in A $ e necessariamente $ f $ dev'essere irriducibile in $ A $.
	Se $ c(f) = 1 $, allora $ f $ è primitivo e deve necessariamente essere essere irriducibile anche in $ \Q(A)[\,x\,] $, altrimenti potremmo riportare la decomposizione in $ A[\,x\,] $ mediante il Lemma di Gauss.
	Essendo $ \Q(A)[\,x\,] $ un \textsc{ed}, $ f $ è anche primo, lì dentro. L'ideale primo generato da $ f $ si contrae a un ideale primo in $ A[\,x\,] $, dunque $ f $ è primo anche nell'anello in esame. Riassumendo:
	\begin{align*}
		f \text{ irriducibile in } A[\,x\,] & \Rightarrow f \text{ irriducibile in } \Q(A)[\,x\,] & (\ref{LemmaGauss}) \\
		& \Rightarrow f \text{ primo in } \Q(A)[\,x\,] & (\ref{primossemassimale}) \\
		& \Rightarrow f \text{ primo in } A[\,x\,] & (\ref{contr}) \\
	\end{align*}
	
	
	Mostriamo ora che ogni catena discendente di divisibilità è stazionaria. Prendiamo una sequenza di polinomi $ \{ f_n \} $ tale che
	\[ \cdots \mid f_3 \mid f_2 \mid f_1 \mid f_0 \]
	Possiamo decomporre ogni polinomio nel prodotto della parte primitiva per il contenuto
	\[ f = c(f)f' \]
	e, osservando che la relazione di divisibilità è soddisfatta se e solo se lo è per componenti
	\[ f \mid g \quad\Leftrightarrow\quad c(f) \mid c(g) \;\text{ e }\; f' \mid g', \]
	ci riduciamo a contemplare due diverse catene discendenti di divisibilità
	\[ \cdots \mid c(f_3) \mid c(f_2) \mid c(f_1) \mid c(f_0) \]
	stazionaria perché  $ A $ è \textsc{ufd}
	\[ \cdots \mid f'_3 \mid f'_2 \mid f'_1 \mid f'_0 \]
	stazionaria perché  $ \Q(A)[\,x\,] $ è \textsc{ufd}.
\end{proof}



\begin{theorem}[Criterio di Eisenstein]
	Dato $ f \in A[x] $ primitivo e $ p \in A $ primo, se
	\begin{enumerate}
		\item  $ p \nmid a_n $.
		\item $ p \mid a_i $ per $ 0\leq i\leq n-1  $.
		\item $ p^2 \nmid a_0 $
	\end{enumerate}
	allora $ f $ è irriducibile.
\end{theorem}
\begin{proof}
	Ragioniamo per assurdo. Se il nostro polinomio si decomponesse in 
	\[ f = gh \]
	guardando questa relazione in $ \faktor{A}{(p)}[\,x\,] $ scopriamo che
	\[ a_nx^n = \bar{f} = \overline{hg} = \bar{h}\bar{g} \]
	e dunque $ \bar{0} = \overline{h_0g_0} = \bar{h_0}\bar{g_0} $, che significa
	\[ p \mid h_0 \text{ e } p \mid g_0 \RR p^2 \mid h_0g_0 = a_0 \]
	il che contraddice le ipotesi!
\end{proof}

\textbf{Esempi.}
\begin{enumerate}
	\item  In $ \K[\,x\,][\,t\,] $, l'elemento $ f = t^n - x $ è irriducibile per il criterio di Eisenstein con $ p = x $. 
	\item $ \K[\{x_i\}_{i \in \mathbb{N}}] $ non è Noetheriano
	\[ (x_0) \subsetneq (x_0, x_1) \subsetneq (x_0, x_1, x_2) \subsetneq \dots  \]
	ma è \textsc{ufd}. Infatti preso un polinomio $ f $, questo è composto da un numero finito di addendi di grado finito, quindi è contenuto in $ \K[x_1, \dots, x_m] $ per un qualche intero $ m $, che è un $ \textsc{ed} $. Qui dentro ammette fattorizzazione unica. Inoltre, nella fattorizzazione non compaiono termini del tipo $ x_k $ con $ k > m $: se così fosse, guardandoli come polinomi in $ x_k $, non sarebbero rispettate le condizioni sulla funzione grado in $ \K[x_1, \dots, x_k] $.
\end{enumerate}


\end{multicols}







