\documentclass[a4paper]{article}

\makeatletter
\title{Algebra I}\let\Title\@title
\author{Andrea Gallese}\let\Author\@author
\date{\today}\let\Date\@date

\usepackage[italian]{babel}
\usepackage[utf8]{inputenc}

\usepackage{mathtools}
\usepackage{amssymb}
\usepackage{amsthm}
\usepackage{faktor}
\usepackage{wasysym}
\usepackage{thmtools}

\usepackage[margin=1.5cm]{geometry}
\usepackage{fancyhdr}
\usepackage[position=top]{subfig}
\usepackage{multirow}

\usepackage{lipsum}
\usepackage{titlesec}
\usepackage{multicol}
\usepackage{setspace}
\usepackage{mdframed}
\usepackage{enumitem}

% magia
\usepackage{hyperref}
\hypersetup{
	colorlinks,
	citecolor=black,
	filecolor=black,
	linkcolor=black,
	urlcolor=black
}

% Formato Teoremi, Dimostrazioni, Definizioni
\newtheorem{theorem}{Teorema}[section]
\newtheorem{prop}[theorem]{Lemma}
\theoremstyle{remark}
\newtheorem*{remark}{Osservazione}
\theoremstyle{definition}
\newtheorem*{definition}{Definizione}
\renewcommand\qedsymbol{$\clubsuit$}

% Frontespizio e piè di pagina
\pagestyle{fancy}
\fancyhf{}
\rhead{\textsf{\Author}}
\chead{\textbf{\textsf{\Title}}}
\lhead{\textsf{\today}}
\cfoot{\thepage}

% Crea una nuova pagina per ogni sottosezione
\newcommand{\subsectionbreak}{\clearpage}

% Per avere le sezioni con le lettere
\renewcommand{\thesection}{\Alph{section}}

%Comandi specifici
\newcommand{\N}{\mathbb{N}}
\newcommand{\Z}{\mathbb{Z}}
\newcommand{\Q}{\mathbb{Q}}
\newcommand{\R}{\mathbb{R}}
\newcommand{\K}{\mathbb{K}}
\newcommand{\F}{\mathbb{F}}

\renewcommand{\S}{\mathcal{S}}

\newcommand{\Aut}[1]{\mathrm{Aut}\left( #1 \right)}
\newcommand{\Int}[1]{\mathrm{Int}\left( #1 \right)}
\newcommand{\Orb}[1]{\mathcal{O}rb\left( #1 \right)}
\newcommand{\Stab}[1]{\mathcal{S}tab\left( #1 \right)}
\newcommand{\gen}[1]{\langle #1 \rangle}
\newcommand{\Gal}[1]{\mathcal{G}al\left( #1 \right)}

\newcommand{\LR}{\quad\Leftrightarrow\quad}
\newcommand{\RR}{\quad\Rightarrow\quad}

\newcommand{\fun}[5]{
	\begin{align*}
	#1 \colon #2 &\to #3 \\
	#4 &\mapsto #5
	\end{align*}
}

% indentazione
\setlength{\parindent}{0pt}

% multicols
\usepackage{multicol}
\setlength\columnsep{20pt}
\setlength{\columnseprule}{0,5pt}

% Per disegnare diagrammi commuatativi
\usepackage{tikz-cd}
\usepackage{tikz}


% Bullet delle liste puntate
\renewcommand\labelitemi{$ \blacktriangleright $}

\begin{document}

\newgeometry{left=3cm,bottom=1cm}

\begin{center}
	\vspace*{0,5 cm}
	{\Huge \textsf{\textbf{\Title}}} \\
	\vspace{0,75 cm}
	\textsc{\Author} \hspace{1cm} \textsc{\Date}
	\thispagestyle{empty}
	\vspace{0,7 cm}
\end{center}

\small

\tableofcontents

\bigskip
Per tutto quello che riguarda gli interi di Gauss, rifarsi a \href{http://www.math.uconn.edu/~kconrad/blurbs/ugradnumthy/Zinotes.pdf}{\texttt{kconrad}}.

\restoregeometry

\setcounter{section}{6} % per scegliere la lettera giusta
\section{Teoria dei Gruppi}
\subsection{Automorfismi e Azioni}
\begin{multicols}{2}
\begin{theorem}
	Se G è un gruppo, $ \left(\Aut{G}, \circ\right) $ è un gruppo.
\end{theorem}

\textbf{Esempi}.
\begin{enumerate}
	\item $ \Aut{\mathbb{Z}} \cong \{\pm id\} \cong \mathbb{Z}_2 $
	\item $ \Aut{\mathbb{Z}_n} \cong \mathbb{Z}_n^\times $
	\item $ \Aut{\mathbb{Q}} \cong \mathbb{Q}^\times $
	\item $ \Aut{\mathbb{R}} \cong \; ? $
\end{enumerate} 

\begin{definition}[automorfismi interni]
	Chiamiamo $$  \Int{G} = \{ \varphi_g \mid g \in G \}  $$ il sottogruppo degli automorfismi interni, ovvero degli automorfismi di $ G $ che sono dati dal coniugio per un elemento del gruppo stesso: \[ \varphi_g(x) = gxg^{-1} \quad\forall x \in G.\]
\end{definition}

Si osserva che $ \Int{G} \lhd \Aut{G} $.

\begin{prop}[degli Automorfismi Interni]
	\[ \Int{G} \cong \faktor{G}{Z(G)} \]
\end{prop}
\begin{proof}
	La funzione \fun{\Phi}{G}{\Int{G}}{g}{\varphi_g} è un omomorfismo con nucleo $ Z(G) $.
\end{proof}

\begin{remark}
	Un sottogruppo è normale se e solo se viene rispettato da tutti gli automorfismi interni:
	\[ H \lhd\, G \LR \varphi_g\left(H\right) = H \quad \forall \varphi_g \in \Int{G}. \]
\end{remark}

\begin{definition}[Sottogruppo caratteristico]
	Un sottogruppo $ H < G $ si dice \emph{caratteristico} se è rispettato da tutto $ \Aut{G} $, i.e. $$ \varphi\left(H\right) = H \quad \forall \varphi \in \Aut{G} $$
\end{definition}
\begin{remark}
	Un sottogruppo caratteristico è anche normale, ma non è sempre vero il viceversa: $ \langle (0, 1) \rangle \lhd\, \mathbb{Z}_2 \times \mathbb{Z}_2 $.
\end{remark}

\begin{definition}[Azione]
	Si dice azione di un gruppo $ G $ su un insieme $ X $ un omomorfismo \begin{align*}
	\varphi \colon G &\to \mathcal{S}\left(X\right) \\
	g &\mapsto \varphi_g(x) = g \cdot x.
	\end{align*}
\end{definition}

\textbf{Esempio.} Siano $ G = C = \{ z \in \mathbb{C} \mid |z| = 1 \} $, $ X = \mathbb{R}^2 $ e sia $ \varphi $ l'azione: \begin{align*}
\varphi \colon C &\to \mathcal{S}\left(\mathbb{R}^2\right) \\
z &\mapsto \mathcal{R}(O, \arg z)
\end{align*}

\begin{remark}
	Un'azione induce naturalmente una relazione di equivalenza su $ X $: $ x \sim y \Leftrightarrow \exists g \in G $ tale che $ g \cdot x = y $. Viene quindi spontaneo prendere in considerazione la partizione di $ X $ così ottenuta.
\end{remark}
\begin{definition}[Orbita]
	Si dice \emph{orbita} di un elemento $ x \in X $ l'insieme di tutti gli elementi che posso essere raggiunti da $ x $ tramite l'azione: \[ \Orb{x} = \{ g \cdot x \mid \forall g \in G \}. \]
\end{definition}
\begin{remark}
	Detto $ R $ un insieme di rappresentanti delle varie orbite, visto che queste formano una partizione: \[ X = \bigsqcup_{x \in R} \Orb{x} \quad\Rightarrow\quad |X| = \sum_{x \in R} |\Orb{x}|. \]
\end{remark}

\begin{definition}[Stabilizzatore]
	Si dice \emph{stabilizzatore} di un elemento $ x \in X $ l'insieme di tutti gli elementi di $ G $ che agiscono in modo banale su $ x $: \[ \Stab{x} = \{ g \in G \mid g \cdot x = x \}. \]
\end{definition}
\begin{remark}
	Lo stabilizzatore è un sottogruppo $$ \Stab{x} < G $$ ma non è necessariamente normale. $$  Z_{\mathcal{A}_{5}}(1 \; 2 \; 3) < \mathcal{A}_{5}$$
\end{remark}

\begin{prop}[Relazione Orbite-Stabilizzatori]
	\[ |G| = |\Orb{x}||\Stab{x}| \]
\end{prop}
\begin{proof}
	La funzione $ f $ così definita
	\begin{align*}
	f \colon \{ g \Stab{x} \mid g \in G \} &\to \{\Orb{x} \mid x \in X\} \\
	g\Stab{x} &\mapsto g \cdot x
	\end{align*}
	è biunivoca, infatti:
	\begin{align*}
		g \cdot x = h \cdot x & \LR \varphi_g(x) = \varphi_h(x) \\
		& \LR \varphi_h^{-1} \varphi_g (x) = x \\
		& \LR\varphi_{h^{-1}g} (x) = x \\
		& \LR h^{-1}g \cdot x = x \\
		& \LR h^{-1}g \in \Stab{x} \\
		& \LR g \in h\Stab{x} \\
		& \LR g\Stab{x} = h\Stab{x}
	\end{align*}
\end{proof}
\begin{remark}
	Dall'osservazione precedente \[ |X| = \sum_{x \in R} \frac{|G|}{|\Stab{x}|} \]
\end{remark}

\textbf{Esempi.}
\begin{enumerate}
	\item $ [G = C,\; X = \mathbb{R}^2 ] $ e l'azione dell'ultimo esempio. Questa ruota ogni punto attorno all'origine, pertanto le orbite sono circonferenze centrate nell'origine e gli stabilizzatori sono tutti banali, tranne quello dell'origine che coincide con $ G $.
	\item $ [G = \mathbb{R},\; X = \mathbb{R}^2 ] $ e l'azione che trasforma $ r \in \mathbb{R} $ nella traslazione orizzontale di lunghezza $ r $. Le orbite sono le rette parallele alla traslazione e gli stabilizzatori sono tutti banali.
	\item $ [G,\; X = G ] $ e l'azione sia la mappa che manda un elemento $ g $ nel coniugio per questo $ \varphi_g(x) = gxg^{-1} $. L'orbita di un elemento contiene tutti i coniugati di questo ed è detta \emph{classe di coniugio} di $ x $ $ (\mathcal{C}_x) $. Lo stabilizzatore di $ x $ contiene tutti e soli gli elementi tali che $ xg = gx $, ovvero il sottogruppo di tutti gli elementi che commutano con $ x $, è detto \emph{centralizzatore} di $ x $ ($ Z_G(x) $).
	\item $ [G,\; X = \{ H \mid H < G \} ] $ e l'azione di coniugio. Lo stabilizzatore di un sottogruppo è detto \emph{Normalizzatore} di $ H $, $ N(H) $, ed è il più grande sottogruppo di $ G $ in cui $ H $ è normale.
\end{enumerate}
\begin{remark}
	$ H \lhd\, G \;\Leftrightarrow\; N(H) = G $.
\end{remark}
\begin{remark}
	Le azioni più comuni sono quelle naturali: il coniugio, la moltiplicazione a sinistra e, talvolta, la moltiplicazione a destra per l'inverso.
\end{remark}
\end{multicols}

\subsection{Formula delle Classi e Cauchy}
\begin{multicols}{2}

\begin{theorem}[Formula delle Classi]
	In ogni gruppo finito
	\[ |G| = |Z(G)| + \sum_{\substack{x \in R'}} \frac{|G|}{|Z_G(x)|}. \]
\end{theorem}
\begin{proof}
	Facciamo agire $ G $ su se stesso per coniugio (l'esempio 3 di sopra): abbiamo già osservato che $ X $ viene partizionato in orbite, alcune delle quali saranno banali:
	
	\[ |X| = \sum_{\substack{x \in R \\ \Orb{x} = \{x\} }} 1 + \sum_{\substack{x \in R \\ \Orb{x} \neq \{x\} }} \frac{|G|}{|\Stab{x}|}. \]
	
	L'orbita di $ x $ è banale se e solo se $ gxg^{-1} = x \; \forall g \in G $, ovvero nel caso in cui $ x $ commuta con tutti gli elementi di $ G $: vale a dire, quando vive nel centro $ Z(G) $.  Per concludere è sufficiente ricordare la definizione di centralizzatore.
\end{proof}

\begin{definition}[$ p $-gruppo]
	Dato un primo $ p \in \mathbb{N} $, si dice $ p $-gruppo un gruppo finito $ G $ di ordine potenza di $ p $: $ |G| = p^n $.
\end{definition}

\textbf{Proprietà.}
\begin{enumerate}
	\item \textbf{Un $ p $-gruppo $ G $ ha centro non banale.} Tutti i centralizzatori degli elementi di $ R' $ hanno dimensione $ p^k $ per un intero $ 0 \leq k < n $, dunque
	\[ p \mid \frac{|G|}{|Z_G(x)|} \quad\forall x \in R'. \]
	Scrivendo la formula delle classi, scopriamo che il centro ha cardinalità divisibile per $ p $:
	\[ p \mid |G| - \sum_{\substack{x \in R'}} \frac{|G|}{|Z_G(x)|} = |Z(G)|. \]
	Contenendo già $ e $, il centro deve avere almeno altri $ p-1 $ elementi.
	
	\item \textbf{I gruppi di ordine $ p^2 $ sono abeliani.} Il centro di $ G $ avrà, per quanto appena dimostrato, ordine $ p $ o $ p^2 $. Nel secondo caso abbiamo finito. Nel primo \[  \left|\faktor{G}{Z(G)}\right| = p, \] dunque il quoziente è ciclico e possiamo appellarci al seguente lemma:
	\begin{prop}
		Se il quoziente tra un gruppo e il suo centro è ciclico, allora il gruppo è abeliano.
	\end{prop}
	Presi due elementi qualunque $ x, y \in G $ possiamo esprimerli come $ x = g^h a $ e $ y = g^k b $, dove $ g $ è il generatore del quoziente e $ a, b \in Z(G) $. Allora, sfruttando la commutatività degli elementi del centro, ricaviamo la commutativa per tutti gli elementi del gruppo: \[ xy = (g^h a) (g^k b) = g^{h+k} ab = g^{k+h} ba = (g^k b) (g^h a) = yx. \]
	
	
	\item Una possibile dimostrazione del Teorema di Cauchy:
\end{enumerate}
\columnbreak
\begin{theorem}[di Cauchy] \label{cauchy}
	Per ogni fattore primo $ p $ di $ |G| $ esiste un elemento $ g $ di $ G $ di ordine $ p $.
\end{theorem}
\begin{proof}[Dimostrazione Classica]
	Sia $ |G|= pn $, procediamo per induzione su $ n $.
	Se $ n = 1 $, $ G $ è ciclico, quindi ha un generatore di ordine $ p $.
	Supponiamo ora, come ipotesi induttiva, che tutti i gruppi più piccoli di qualche valore fissato (di ordine $ kp $ con $ k < m $) abbiamo un elemento di ordine $ p $. Prendiamo ora un gruppo con cardinalità proprio $ |G|=pm $. \`{E} chiaro che se $ G $ avesse un sottogruppo proprio di ordine multiplo di $ p $, lì vi troveremmo per ipotesi induttiva l'elemento cercato. Supponiamo quindi che nessun sottogruppo di $ G $ abbia ordine divisibile per $ p $, in particolare nemmeno i centralizzatori (che ricordiamo essere sottogruppi) hanno ordine divisibile per $ p $, da cui la seguente relazione di divisbilità:
		\[ p \mid \frac{|G|}{|Z_G(x)|} \forall x \in R' .\]
	Grazie alla formula delle classi scopriamo che
		\[ p \mid |G| - \sum_{\substack{x \in R'}} \frac{|G|}{|Z_G(x)|} = |Z(G)| \]
	ma avevamo supposto che i sottogruppi propri non avessero ordine multiplo di $ p $, dunque il centro deve coincidere con l'intero gruppo, che risulta pertanto commutativo.

\end{proof}
\begin{proof}[Dimostrazione Magica]
	Sia $$ X = \{(x_1, \dots, x_p) \in G^p \mid x_1\cdots x_n = 1 \} $$ questo insieme ha esattamente $ |G|^{p-1} $ elementi, infatti scelti i primi $ p-1 $ l'ultimo dev'essere il suo unico inverso.
	Facciamo agire $ \Z_p $ su $ X $, in modo che cicli le $ p $-uple. Ogni elemento ha un'orbita banale o lunga $ p $. Gli elementi con orbita banale sono quelli con $ x_1 = \dots = x_p = g $ e $ g^p = 1 $, dunque proprio quelli che stiamo cercando: diciamo di averne $ n $. Consideriamo ora il numero di elementi con orbita \emph{non} banale:
	visto che le orbite hanno cardinalità $ p $ e partizionano l'insieme delle $ p $-uple rimanenti
	\[ p \mid |G|^{p-1} - n \quad\Rightarrow\quad p \mid n \]
	e, assodato che $ e^p = e $, ci devono essere almeno $ p - 1 $ elementi di ordine $ p $.
\end{proof}


\subsubsection{Fatti che non stavano da nessuna parte}

\begin{definition}[Sottogruppo generato]
	Sia $ S \subset G $ un \emph{sottoinsieme} di $ G $. Chiamiamo $ \langle S \rangle $ il più piccolo sottogruppo contenente $ S $, sottogruppo generato da $ S $. \[ \langle S \rangle = \bigcap_{\substack{H \leq G \\ S \subseteq H }} H  \]
\end{definition}
\begin{prop}[Caratterizzazione dei sottogruppi generati]
	\[ \langle S \rangle = \{ s_1 \cdots s_k \mid k \in \mathbb{N}, \; s_i \in S \cup S^{-1} \} \]
\end{prop}
\begin{proof}
	\`{E} sufficiente osservare che tutti gli elementi di $ S $ devono comparire in tutti i gruppi che contengono $ S $ e che l'insieme proposto è un sottoinsieme chiuso per le operazioni di gruppo, dunque un sottogruppo.
\end{proof}
\columnbreak
\textbf{Proprietà.}
\begin{enumerate}
	\item $ \langle S \rangle $ è abeliano se e solo se tutti gli elementi di $ S $ commutano fra loro.
	\item $ \langle S \rangle $ è normale se e solo se ogni ogni elemento di $ S $ rimane in $ \langle S \rangle $ per coniugio.
	\item $ \langle S \rangle $ è caratteristico se e solo se ogni elemento di $ S $ viene mandato in $ \langle S \rangle $ da ogni automorfismo di $ G $.
	\item $ G' = \langle ghg^{-1}h^{-1} \mid g, h \in G \rangle $ è detto \emph{Gruppo dei Commutatori o Gruppo Derivato di $ G $}. Questo gruppo gode di alcune proprietà fondamentali
	
	\begin{enumerate}
		\item $ G' = \{ e \} \;\Leftrightarrow\; G $ abeliano.
		\item $ G' $ è caratteristico e pertanto normale in $ G $.
		\item \textbf{Dato $ H \lhd G $, il quoziente $ \faktor{G}{H} $ è abeliano se e solo se $ G' < H $.}
	
		
	\end{enumerate}

\begin{proof}
	La verifica delle proprietà $ (a) $ e $ (b) $ è banale. Rimane l'ultima $ (c) $:
	\begin{align*}
	\faktor{G}{H} \;\text{abeliano} & \Leftrightarrow\; xHyH = yHxH &\forall x, y \in G \\
	& \Leftrightarrow\; xyH = yxH &\forall x, y \in G\\
	& \Leftrightarrow\; x^{-1}y^{-1}xy \in H &\forall x, y \in G\\
	& \Leftrightarrow\; g' \in H &\forall g' \in G'
	\end{align*}
\end{proof}
\begin{definition}
	$ \faktor{G}{G'} $ è detto l'abelianizzato di $ G $, perché è sempre abeliano!
\end{definition}
\end{enumerate}
\end{multicols}

\subsection{Gruppi Diedrali  $ D_n $}
\begin{multicols}{2}
\begin{definition}[Gruppo Diedrale]
	Sia $ D_n $ il gruppo delle isometrie dell'$ n $-agono regolare.
\end{definition}
\begin{theorem}[Caratterizzazione di $ D_n $]
	Si ha \[ D_n = \langle \rho, \sigma \mid \rho ^n = e, \sigma^2 = e, \sigma \rho \sigma = \rho^{-1} \rangle \]
\end{theorem}
\textsf{Questo enunciato non si capisce, non si è mai parlato prima di presentazione di gruppi. Rileggendo la dimostrazione aumenta questo assurdo senso di incompletezza: dove stiamo andando? cosa vogliamo dimostrare? qual è il significato profondo della vita, l'universo e tutto quanto? Non lo so.}
\begin{proof}
	Tutti gli elementi sopra definiti possiamo ridurli a un elemento della forma $ \rho^k $ o $ \sigma \rho^k $ per un qualche $ 0 \leq k < n $. Questo perché i generatori si presentano in questa forma e ogni operazione permessa (composizione e inversione) tra due generatori produce un risultato che può immediatamente essere ridotto alla forma cercata attraverso le cancellazioni consentite. Inoltre possiamo immergere questo gruppo in un sottogruppo di $ \mathbf{O}_2(\mathbb{R}) $ di ordine $ 2n $ attraverso l'omomorfismo suriettivo \begin{align*}
	\Phi \colon \gen{\rho, \sigma} &\to \mathbf{O}_2(\mathbb{R}) \\
	\sigma &\mapsto \left(\begin{matrix}
	-1 & 0 \\
	0 & 1
	\end{matrix}\right) \\
	\rho &\mapsto \left(\begin{matrix}
	\cos\frac{2\pi}{n} & \sin\frac{2\pi}{n} \\
	-\sin\frac{2\pi}{n} & \cos\frac{2\pi}{n}
	\end{matrix}\right).
	\end{align*}
	Ne concludiamo che ognuno dei rappresentati sopra trovati individua un'effettiva trasformazione distinta, da cui l'uguaglianza con $ D_2 $.
\end{proof}

\begin{remark}
	Conosciamo già un gruppo diedrale: $ D_3 \cong \mathcal{S}_3 $.
\end{remark}
\begin{remark}
	Il sottogruppo $ C_n $ delle rotazioni, generato da $ \rho $, è ovviamente ciclico e, avendo indice 2, è anche normale in $ D_n $: $$ \langle \rho \rangle = C_n \lhd\, D_n. $$
\end{remark}

\begin{prop}[Ordine degli elementi di $ D_n $]
	Sappiamo che
	\begin{itemize}
		\item tutte le simmetrie hanno ordine 2.
		\item ci sono $ \varphi(m) $ rotazioni di ordine $ m $, per ogni $ m \mid n $.
	\end{itemize}
\end{prop}
\begin{proof}
	La seconda parte è immediata conseguenza della ciclicità del sottogruppo delle rotazioni.
	L'ordine delle riflessioni possiamo calcolarlo esplicitamente notando che $$  \left(\sigma\rho^k\right)\left(\sigma\rho^k\right) = \left(\sigma\rho^k\sigma\right)\rho^k = \rho^{-k}\rho^{k} = e,  $$ dove abbiamo usato la terza proprietà imposta nella caratterizzazione.
\end{proof}
\columnbreak
\begin{prop}[Sottogruppi di $ D_n $]
	I sottogruppi $ H < D_n $ rientrano in una di queste due categorie:
	\begin{itemize}
		\item $ H < C_n $: di cui ne abbiamo esattamente uno per ogni ordine divisore di $ n $.
		\item $ H = (H \cap C_n) \sqcup \tau(H\cap C_n) $: di cui ce ne sono $ d $ di ordine $ \frac{2n}{d} $ per ogni $ d \mid n $.
	\end{itemize}
\end{prop}
\begin{proof}
	Se $ H < C_n $ il risultato viene da Aritmetica. Se $ H \nless C_n $, $ H $ contiene almeno una rotazione $ \tau = \sigma\rho^i $. Consideriamo l'omomorfismo $ f $ che fa commutare il diagramma
	\[ \begin{tikzcd}
	D_n \arrow{r}{\Phi} \arrow[swap, dashed]{dr}{f} & \mathbf{O}_2(\mathbb{R}) \arrow{d}{det} \\
	& \{\pm 1\} \cong \mathbb{Z}_2	\end{tikzcd}
	\]
	Restringiamo l'omomorfismo trovato ad $ H $
	\[ \begin{tikzcd}
	D_n \arrow{r}{\Phi} \arrow[swap]{dr}{f} & \mathbf{O}_2(\mathbb{R}) \arrow{d}{det} \\
	H \arrow[dashed]{r}{} \arrow[swap, hook]{u}{} & \mathbb{Z}_2	\end{tikzcd}
	\]
	Visto che $ \ker f = C_n \lhd D_n $, il sottogruppo $ H $ viene scomposto in nucleo e laterale
	\[ H = f^{-1}(0) \sqcup f^{-1}(1) = (H \cap C_n) \sqcup \tau (H\cap C_n). \]
	Infine, dato che il nucleo è contenuto nel sottogruppo ciclico $ (H \cap C_n) < C_n $, possiamo pensarlo come il sottogruppo generato da una potenza della rotazione elementare \[ H \cap C_n = \langle \rho^d \colon d \mid n \rangle. \] Il suo unico laterale sarà allora composto dagli $ d $ elementi della forma  \begin{align*}\tau(H\cap C_n) &=  \{\tau\rho^d, \tau\rho^{2d}, \dots, \tau\rho^{n-d} \} \\ &= \{\sigma\rho^{d+i}, \sigma\rho^{2d+i}, \dots, \sigma\rho^{n-d +i}, \}\end{align*}  che dipende solamente dalla classe di $ i \mod d $.
\end{proof}

\textbf{Esercizi importanti.}
\begin{enumerate}
	\item Quali sottogruppi di $ D_n $ sono normali?
	\item Quali sottogruppi di $ D_n $ sono caratteristici?
	\item Quali sono i quozienti di $ D_n $?
	\item $ (\star) $ Chi è $ \Aut{D_n} $?
\end{enumerate}

\end{multicols}

\subsection{Gruppi di Permutazioni $ \mathcal{S}_n $}
\begin{multicols}{2}
\begin{definition}[Gruppi di Permutazioni]
	Dato un insieme $ X $, chiamiamo
	\[  \mathcal{S}(X) = \{ f: X \to X \mid f  \text{ è bigettiva}  \},  \]
	con l'operazione di composizione, \emph{gruppo delle permutazioni} di $ X $. Se l'insieme è finito $ |X| = n $, allora
	\[  \mathcal{S}(X) \cong S(\{ 1 ,2, \dots, n \})  \]
	e lo chiamiamo $ \mathcal{S}_n $.
\end{definition}

\begin{theorem}[Cayley]\label{Cayley}
	Possiamo immergere ogni gruppo finito $ G $ in un gruppo di permutazioni.
\end{theorem}
\begin{proof}
	L'azione di moltiplicazione a sinistra è fedele
	\fun{\Phi}{G}{\mathcal{S}(G)}{g}{\varphi_g(x) = gx}
	ovvero, iniettiva.
\end{proof}

\begin{prop}[di decomposizione]
	Ogni permutazione $ \sigma \in \mathcal{S}_n $ si scrive in modo unico come prodotto di cicli disgiunti.
\end{prop}
\begin{remark}
	Cicli disgiunti commutano.
\end{remark}

%\begin{remark} Questa cosa non vuol dire niente!
%	I cicli sono orbite dell'applicazione di immersione. 
%\end{remark}

\begin{remark}
	$ \mathcal{S}_n $ è generato dai suoi cicli.
\end{remark}
\begin{prop}[delle permutazioni coniugate]
	Due permutazioni $ \sigma, \tau \in \mathcal{S}_n $ sono coniugate se e solo se hanno lo stesso tipo di decomposizione in cicli disgiunti.
\end{prop}
\begin{proof}
	E' più che sufficiente osservare che dato un ciclo $$  \sigma = (a_1 \; \cdots \; a_k)  $$ e una permutazione tale che $ \tau : a_i \mapsto b_i $, si ha
	\[\tau\sigma\tau^{-1} = (\tau(a_1) \; \cdots \; \tau(a_k)) = (b_1 \; \cdots \; b_k). \]
\end{proof}
\begin{remark}
	$ \mathcal{S}_n $ è generato dalle sue trasposizioni.
\end{remark}
\begin{remark}
	La decomposizione in trasposizioni non è unica. Ma la parità del numeri di trasposizioni lo è:
\end{remark}
\begin{theorem}[delle trasposizioni]
	La parità del numero di trasposizioni della scomposizione di una qualunque permutazione $ \sigma \in \mathcal{S}_n $ non dipende dalla scomposizione.
\end{theorem}
\begin{proof}
	Consideriamo
	\fun{\text{sgn}}{\mathcal{S}_n}{\mathbb{Z}^\times = \{\pm 1\}}{\sigma}{\prod_{1 \leq i < j \leq n}\frac{\sigma(i)-\sigma(j)}{i-j}}
	questo è un omomorfismo di gruppi. Infatti:
	\begin{enumerate}
		\item è ben definito, ovvero $ |\text{sgn}(\sigma)| = 1 $: tutte le differenze che compaiono a denominatore compaiono anche a numeratore, poiché $ \sigma $ è una permutazione, magari con ordine o segno differente.
		\item Si comporta bene con la composizione
		\begin{align*}
			\text{sgn}(\sigma \circ \tau) &= \prod_{i < j}\frac{\sigma(\tau(i))-\sigma(\tau(j))}{i-j} \\
			& = \prod_{i < j}\frac{\sigma(\tau(i))-\sigma(\tau(j))}{\tau(i)-\tau(j)}\cdot \frac{\tau(i)-\tau(j)}{i-j} \\
			& = \prod_{i < j}\frac{\sigma(\tau(i))-\sigma(\tau(j))}{\tau(i)-\tau(j)}\cdot\prod_{i < j} \frac{\tau(i)-\tau(j)}{i-j}\\
			& = \text{sgn}(\sigma)\cdot \text{sgn}(\tau)
		\end{align*}
	\end{enumerate}
	Per concludere, osserviamo che tutte le trasposizioni hanno segno negativo. 
\end{proof}

\begin{definition}[Gruppo Alterno]
	Chiamiamo $ \mathcal{A}_n $ o \emph{gruppo alterno} il sottogruppo delle permutazioni pari
	\[ \ker(\text{sgn}) = \mathcal{A}_n \lhd\, \mathcal{S}_n. \]
\end{definition}
\begin{remark}
	Ogni sottogruppo $ H < \mathcal{S}_n $ è contenuto interamente in $ \mathcal{A}_n $ o viene spezzato a metà dal gruppo alterno. Questo perché possiamo restringere il segno al solo $ H $, che dunque può solo spezzarsi in nucleo e laterale, o immergersi interamente nel nucleo.
\end{remark}

\begin{prop}[di spezzamento]
	La classe di coniugio di $ \sigma $ in $ \S_n $ si spezza in $ \mathcal{A}_n $ se e solo se $ \sigma $ è composto da cicli di lunghezze dispari e distinte.
\end{prop}


\begin{remark}
	$ A_n $ è generato dai suoi 3-cicli.
\end{remark}
\begin{remark}
	Tutti i 3-cicli sono coniugati in $ A_n $.
\end{remark}
\begin{theorem}
	Il gruppo alterno $ \mathcal{A}_n $ è semplice per $ n \geq 5 $.
\end{theorem}
\begin{proof}
	Iniziamo a dimostrare che $ \mathcal{A}_5 $ è semplice. Le classi di coniugio di questo gruppo hanno dimensione \[1,\, 12,\, 12,\, 15,\, 20.\] Un sottogruppo normale dovrebbe essere unione di un certo numero di queste orbite, ma sommando le loro cardinalità non ottengo mai divisori di 60. Allo stesso modo, $ A_6 $ è semplice: le sue classi coniugio hanno dimensione
	\[ 1,\, 40,\, 40,\, 45,\, 72,\, 72,\, 90. \]
	Siamo ora pronti per affrontare $ n \geq 7 $. Supponiamo che esista un sottogruppo normale
	\[ \{e\} \neq N \lhd A_n \]
	e mostriamo che contiene un 3-ciclo. Essendo non banale, ci troviamo una permutazione non identica $ \sigma $ che possiamo supporre, senza perdita di generalità, non fissare l'1 (tale che $  \sigma(1) \neq 1  $).
	Prendiamo anche una permutazione $ \tau = (i\; j\; k) $ di tre indici diversi da 1 tali che $ \sigma(1) \in \{i,\, j,\, k\} $. Nell'operare questa scelta, abbiamo bisogno che $ n \geq 4 $. Inoltre
	\[ \tau\sigma\tau^{-1}(1) = \tau(\sigma(1)) \neq \sigma(1), \]
	dunque $ \tau\sigma\tau^{-1} \neq \sigma $, così siamo sicuri che la permutazione
	\[ \phi = \tau\sigma\tau^{-1}\sigma^{-1} \]
	non sia l'identità. Quest'ultima $ \phi $ è sia il prodotto tra un coniugato di un elemento nel nostro sottogruppo normale per il suo inverso 
	\[ \phi = (\tau\sigma\tau^{-1})\sigma^{-1} \in N,  \]
	che il prodotto tra un 3-ciclo e un suo coniugato
	\[ \phi = \tau(\sigma\tau^{-1}\sigma^{-1}), \]
	quindi permuta al più $ 6 $ numeri. Aggiungiamo, se necessario, numeri a caso, fino ad averne esattamente 6 e chiamiamo $ H $ il sottogruppo di $ A_n $, isomorfo ad $ A_6 $, che li permuta. Abbiamo mostrato che
	\[ \phi \in N \cap H \]
	dunque l'intersezione è non banale, nonché un sottogruppo normale di $ H $! Deve pertanto coincidere con tutto il gruppo. Allora
	 \[ H < N \] 
	e $ A_6 $ contiene tutti i 3-cicli di cui abbiamo bisogno.
\end{proof}

\begin{theorem}[del ribelle]
	Per ogni $ n \geq 3 $ i gruppi di permutazioni sono i propri automorfismi $$  \Aut{\S_n} \cong \S_n  $$ ma non per $ \S_6 $, che è un ribelle.
\end{theorem}
\begin{proof}
	Per cominciare, osserviamo che
	\[ \Int{\S_n} \cong \faktor{\S_n}{Z(S_n)} \cong \S_n \]
	dunque $ \S_n \lhd \Aut{\S_n} $, pertanto ci basta dimostrare che tutti gli automorfismi sono interni. Sfruttiamo il fatto che gli automorfismi mandano classi di coniugio in classi di coniugio e che preservano l'ordine degli elementi per mostrare che
	\begin{itemize}
		\item  Ogni automorfismo rispetta la classe delle trasposizioni. Guardiamo come sono fatte le classi di coniugio degli elementi di ordine 2. Sia $ C_k $ la classe delle permutazioni composte da $ k $ trasposizioni disgiunte, abbiamo che
		\[ |C_k|  = \frac{1}{2^k \cdot k!}\cdot \frac{n!}{(n-2k)!} \]
		Imporre $ |C_1| = |C_k| $, equivale a risolvere
		\[ 2^{k-1} = \frac{(n-2)!}{(n-k)!}{n-k \choose k } \]
		\begin{itemize}
			\item Per $ \boxed{k = 2} $ abbiamo
			\[ 4 = (n-2)(n-3) \]
			che non ha soluzione, perché i due fattori a destra sono interi consecutivi, quindi uno di loro conterrà un primo dispari.
			\item Per $ \boxed{k = 3} $ abbiamo
			\[ 4 = (n-2) {n-3 \choose 3} \]
			che ammette $ n = 6 $ come unica soluzione (il caso famigerato!), visto che $ n-2 \mid 4 $ limita la ricerca delle soluzioni ai soli $ n = 3, 4, 6 $.
			\item Infine, per $ \boxed{k > 3} $ notiamo che il fattore 
			\[ \frac{(n-2)!}{(n-k)!} \]
			contiene almeno un primo dispari.
		\end{itemize}
		Pertanto ogni automorfismo rispetta $ C_1 $, perché le altre classi sono troppo grandi. 
		\item Osserviamo che, preso $ \varphi \in \Aut{\S_n} $, si ha
		$$  \varphi(1, i) = (a_1, a_i)  $$
		con tutti gli $ a_i $ distinti.
		\item Dunque $ \varphi $ coincide, in realtà, con il coniugio per la permutazione
		$$  \sigma : i \mapsto a_i  $$
	\end{itemize}
\end{proof}
\textbf{Esercizi.}
\begin{enumerate}
	\item Quanti $ k $-cicli ci sono in $ \mathcal{S}_n $?
	\item Come conto gli elementi con una composizione fissata in un $ \mathcal{S}_n $ dato? Per esempio, come calcolo le permutazioni del tipo $ 3+3+2+2+2 $ in $ \mathcal{S}_{10} $?
	\item L'ordine di $ \sigma $ è il minimo comune multiplo delle lunghezze dei suoi $ k $-cicli.
	
	\item Notiamo che il centralizzatore di $ \sigma $ coincide con lo stabilizzatore dell'azione di coniugio di $ \mathcal{S}_n $ in se. Dunque
		\[ |Z(\sigma)| = \frac{n!}{|\mathcal{C}(x)|} \]
	
	
	\item Data una permutazione $ \sigma $ trovare $ |N(\langle \sigma \rangle)| $.
	
	Osserviamo che
	\[ N(\gen{\sigma}) = \{ \tau \mid \tau\sigma\tau^{-1} = \sigma^k \} \]
	dunque il normalizzatore contiene il centralizzatore di $ \sigma $ e, visto che il coniugio preserva la scomposizione in cicli, che possiamo prendere solo i $ k $ coprimi coll'ordine di $ \sigma $. Inoltre prese due permutazioni $ \tau_1, \tau_2 \in N(\gen{\sigma})$ che generano lo stesso $ \sigma^k $, abbiamo che
	\[ \tau_1 \sigma \tau_1 ^{-1} = \tau_2 \sigma \tau_2 ^{-1} \quad\Leftrightarrow\quad (\tau_2^{-1}\tau_1) \sigma (\tau_2^{-1}\tau_1) ^{-1} = \sigma \]
	Dunque $ \tau_2^{-1}\tau_1 \in Z(\sigma) $, ovvero $ \tau_1 \in \tau_2 Z(\sigma) $. Pertanto il normalizzatore dev'essere composto da tutti i laterali del centralizzatore indotti da permutazioni che mi danno $ \sigma^k $ dello stesso tipo di $ \sigma $.
	Ovvero
	\[ N(\gen{\sigma}) = \bigcup_{(i, ord(\sigma))= 1} \tau_i Z(\sigma) \]
	e pertanto
	\[ |N(\gen{\sigma})| = |Z(\sigma)| \cdot \phi(\text{ord}(\sigma)) \]
\end{enumerate}



\end{multicols}
\subsection{Prodotti diretti}

\begin{multicols}{2}



\begin{theorem}[di Struttura]{\label{struttura}}
	Sia $ G $ un gruppo e $ H, K < G $ due sottogruppi. Se
	\begin{enumerate}
		\item $ H \lhd\, G $   e   $ K \lhd\, G $
		\item $ HK = G $
		\item $ H \cap K = \{e\} $
	\end{enumerate}
allora $$  G \cong H \times K  $$
\end{theorem}
\begin{proof}
	Mostriamo innanzitutto che $ hkh^{-1}k^{-1} $ appartiene ad entrambi i sottogruppi:
	\[ H  \ni h(kh^{-1}k^{-1}) = h(kh^{-1}k^{-1}) = (hkh^{-1})k^{-1} \in K. \]
	Avendo i due sottogruppi intersezione banale, un elemento che appartiene ad entrambi è necessariamente l'elemento neutro $$ hkh^{-1}k^{-1} = e, $$ quindi tutti gli elementi di un sottogruppo commutano con quelli dell'altro. \\
	
	Consideriamo ora l'isomorfismo che prende elementi che vivono nei due diversi sottogruppi, li immerge nel gruppo $ G $ in cui sono nati e, lì, ne prende il prodotto
	 \fun{\Phi}{H\times K}{G}{(h,g)}{hg}
	e verifichiamo che
	\begin{enumerate}
		\item è un omomorfismo: \[ \Phi(hh', kk') = hh'kk' = hkh'k' = \Phi(h, k)\Phi(h',k'); \]
		\item è suriettivo, grazie all'ipotesi 2;
		\item è iniettivo per l'ipotesi 3; infatti \[ \ker\Phi = \{ (h, k) \mid hk = e \} = \{ (e, e) \} \]
		perché se $ h $ e $ k $ sono inversi l'uno dell'altro, vivono nello stesso sottogruppo, dunque nell'intersezione, che è banale.
	\end{enumerate}
\end{proof}

\begin{remark}
	Nel prodotto diretto i fattori commutano.
\end{remark}

\textbf{Proprietà di $ G = H \times K $.}
\begin{enumerate}
	\item $ Z(G) = Z(H) \times Z(K) $.
	\item $ \Int{G} \cong \Int{H} \times \Int{K} $.
	\item $ \Aut{H} \times \Aut{K} < \Aut{G} $.
\end{enumerate}
\begin{theorem}[degli automorfismi prodotto]
	Si ha $$  \Aut{H} \times \Aut{K} < \Aut{H \times K}  $$ e sono isomorfi se e solo se $ H $ e $ K $ sono caratteristici.
\end{theorem}
\begin{proof}
	Consideriamo la mappa \fun{\Phi}{\Aut{H} \times \Aut{K}}{\Aut{H \times K}}{(f, g)}{\varphi_{fg}: (h, k) \mapsto \left(f(h), g(k)\right)}
	e verifichiamo che
	\begin{itemize}
		\item è bene definita, ovvero $ \varphi $ è un automorfismo. Immediata conseguenza del fatto che $ f $ e $ g $ sono a loro volta automorfismi.
		\item è un omomorfismo.  \begin{align*}
			\Phi(ff',gg') &= \left( f(f'(h)), g(g'(k)) \right)\\& = (\varphi_{fg} \circ \varphi_{f'g'})(h, k) \\&= \Phi(f, g)\Phi(f',g') 
		\end{align*}
		\item è iniettiva.
		\[ \ker\Phi = \{ (id, id) \} \]
		perché restringendo l'identità non possiamo certo sperare di permutare qualcosa.
		
		\item è suriettiva se e solo se $ H\times\{e_K\} $ e $ \{e_H\}\times K $ sono caratteristici in $ H \times K $.
		
		$ \Rightarrow $. Se $ \Phi $ è suriettivo, allora tutti gli automorfismi di $ H \times K $ sono della forma di cui sopra e pertanto $ \varphi_{fg} $ agisce sugli elementi di $ H $ come $ \varphi_{fg|H} = f \in \Aut{H} $.
		
		$ \Leftarrow $. Viceversa, supponiamo $ H $ e $ K $ caratteristici, preso un automorfismo $ \varphi \in \Aut{H \times K}$ consideriamo le sue restrizioni ai due sottogruppi caratteristici.
		\[ f = \Pi_H \left( \varphi_{|H\times\{e_K\}} \right) \qquad g = \Pi_K \left( \varphi_{|\{e_H\}\times K} \right) \]
		Notiamo che $ f \in \Aut{H} $.
		\begin{itemize}
			\item $ f $ è iniettiva. Se $ f(h) = f(h') $ allora \[\Pi_H \left( \varphi(h, e_K) \right) = \Pi_H \left( \varphi(h', e_K) \right) \]
			poiché $ H\times\{e_K\} $ è caratteristico \[ \varphi(h, e_K) = (a, e_K) \qquad   \varphi(h', e_K) = (b, e_K) \]
			ma necessariamente $ a = f(h) $ e $ b = f(h') $, pertanto
			\[ \varphi(a, e_K) = (f(h), e_K) = (f(h'), e_K) = \varphi(b, e_K) \]
			e, visto che $ \varphi $ è iniettivo, $ h = h' $.
			\item $ f $ è suriettiva. Fissiamo un qualunque $ h \in H $. Essendo $ H\times\{e_K\} $ caratteristico, necessariamente la controimmagine di $ (h, e_K) $ è un suo elemento \[ \varphi^{-1}(h, e_K) = (h', e_K) \]
			dunque \[ f(h') =\Pi_H \left( \varphi(h', e_K) \right) = \Pi_H \left( h, e_K \right) = h \]
		\end{itemize}
		Infine osserviamo che $ \Phi(f, g) = \varphi $. Infatti
		\begin{align*} \varphi_{fg}(h, k) &= \left(f(h), g(k)\right)\\& = \left(\Pi_H \left( \varphi(h, e_K) \right), \Pi_K \left( \varphi(e_H, k) \right)\right) \\& =
		\left(\Pi_H \left( \varphi(h, k) \right), \Pi_K \left( \varphi(h, k) \right)\right) \\& =
		 \varphi(h, k) \end{align*}
		 dove la terza uguaglianza segue da
		 \begin{align*}
		 \Pi_H \left( \varphi(h, e_K) \right) &= \Pi_H \left( \varphi(h, e_K) \right)\Pi_H(\varphi(e_H, k)) \\
		 &= \Pi_H \left( \varphi(h, e_K)\varphi(e_H, k)\right) \\&= \Pi_H \left( \varphi(h, k)\right)
		 \end{align*}
	\end{itemize}
\end{proof}

\textbf{Esercizio.} Trovare $ \Aut{\mathbb{Z}_{20} \times \mathbb{Z}_{2} } $.
\end{multicols}

\subsection{Classificazione dei Gruppi di ordine 8}
\[ \begin{tikzcd}
\boxed{\exists g \mid \text{ord}(g) = 8} \arrow{r}{\text{sì}} \arrow[swap]{d}{no} & \mathbb{Z}_8 \\
\boxed{g^2 = e \;\;\forall g \arrow{r}{\text{sì}}} \arrow[swap]{d}{no} & \left(\mathbb{Z}_2\right)^3 \\
\boxed{\varphi_h \mid h \notin C_4 \arrow{d}{\text{-id}}} \arrow{r}{\text{id}} & \mathbb{Z}_2 \times \mathbb{Z}_4  \\
\boxed{\text{ord}(h) = 2} \arrow{r}{\text{sì}} \arrow{d}[swap]{no}  & D_4 \\
Q_8
\end{tikzcd}
\]

\begin{multicols}{2}
	Prendiamo un gruppo $ G $ di ordine 8.
\begin{itemize}
	\item Se esiste un elemento di ordine 8 il gruppo è ciclico e pertanto isomorfo a $ \mathbb{Z}_8 $.
	\item Se $ G $ ha solo elementi di ordine 2, allora è isomorfo a $ \left(\mathbb{Z}_2\right)^3 $. Mostriamo un risultato appena più generale. 
\end{itemize}
\begin{theorem}[dei gruppi solipsisti]
	Se $ |G| $ ha solo elementi di ordine due ed è finito, allora $ G \cong \left(\mathbb{Z}_2\right)^n $.
\end{theorem}
\begin{proof}
	Osserviamo che $$  a^2b^2 = e = (ab)^2 = abab  $$ e, moltiplicando per $ a $ a sinistra e per $ b $ a destra, otteniamo $$  ab = ba \text{ per ogni } a, b \in G  $$ Pertanto $ G $ è abeliano. Possiamo ora procedere per induzione sulla dimensione di $ G $. Se $ |G| = 2 $ il risultato è chiaro. Supponiamo ora che sia vero per tutti i gruppi di ordine $ < 2^n $ e supponiamo $ 2^ n \leq |G| < 2^{n+1} $. Quando prendiamo un insieme minimale di $ h < n $ generatori $ \langle g_1, \dots g_h \rangle $ di un sottogruppo $ H < G $, questo sarà isomorfo a $ \left(\mathbb{Z}_2\right)^h $ per ipotesi induttiva. Prendiamo un elemento $ g \notin H$, abbiamo che $ H $ e $ \langle g \rangle \cong \mathbb{Z}_2 $ sono entrambi sottoinsiemi normali la cui intersezione è banale, pertanto il sottoinsieme \[ \langle g, g_1, \dots g_h \rangle \cong H \times\langle g \rangle \cong \left(\mathbb{Z}_2\right)^h \times \mathbb{Z}_2 \cong \left(\mathbb{Z}_2\right)^{h+1} \]
	per il teorema di struttura \ref{struttura}.
	Così facendo possiamo continuare ad aggiungere elementi fino alla saturazione.
\end{proof}
Se $ G $ non ha elementi di ordine $ 8 $ e non hanno tutti ordine $ 2 $, allora esiste un $ g \in G $ di ordine 4 e sia $ C_4 = \langle g \rangle $. Sia $ h \notin C_4 $ e consideriamo l'azione di coniugio di $ h $ su $ C_4 $
\fun{\varphi_h}{C_4}{C_4}{x}{hxh^{-1}}
(ben definita perché $ C_4 $, avendo indice $ 2 $, è normale in $ G $). Poiché $ \Aut{\mathbb{Z}_4} \cong \mathbb{Z}_2 $, abbiamo solo due possibilità: $$  \varphi_g =  \pm 1.  $$

Considerando i possibili ordini moltiplicativi di $ h $ ci riduciamo a controllare quattro casi:

\begin{itemize}
	\item $ \left[\varphi_h =  1, \, \text{ord}(h) =2\right] $. Gli elementi di $ C_4 $ commutano con $ h $, l'intersezione tra $ C_4 $ e $ \langle h \rangle $ è banale e il loro prodotto genera $ G $ per ragioni di cardinalità, pertanto \[ G \cong  \langle h \rangle\times C_4  \cong  \mathbb{Z}_2\times\mathbb{Z}_4. \]
	\item $ \left[\varphi_h =  1, \, \text{ord}(h) =4\right] $. In questo caso è facile trovare tutti gli elementi e mostrare che il gruppo dev'essere abeliano.
	\item $ \left[\varphi_h =  -1, \, \text{ord}(h) =2\right] $. Abbiamo che $ hgh = g^{-1} $, per la caratterizzazione dei gruppi diedrali \[ G \cong D_4. \]
	\item $ \left[\varphi_h =  -1, \, \text{ord}(h) =4\right] $. Anche ord$ (gh) = 4 $. Infatti
	\[ e = ghgh = ghgh^{-1}hh = hh \neq e. \]
	Dunque abbiamo trovato l'ordine di tutti gli elementi, possiamo costruire un isomorfismo esplicito con $ Q_8 $.
\end{itemize}
\begin{definition}[Quaternioni]
	Sia $ Q_8 $ l'insieme $ \{ \pm 1, \pm i, \pm j, \pm, k \} $ con l'operazione che soddisfa
	\[ i^2 = j^2 = k^2 = ijk = -1 \]
\end{definition}
\end{multicols}



\subsection{Prodotto Semidiretto}
\begin{multicols}{2}
	\begin{definition}[Prodotto semidiretto]
		Siano $ H, K $ due gruppi e $ \varphi: K \rightarrow \Aut{H} $ un omomorfismo Si dice prodotto semidiretto
		\[ H \rtimes_\varphi K \]
		l'insieme dato dal prodotto cartesiano, dotato dell'operazione
		\[ (h, k) \cdot (h', k') = (h \varphi_k(h'), kk') \]
	\end{definition}

\begin{remark}
	Il prodotto semidiretto è un gruppo.
\end{remark}
\begin{remark}
	Il prodotto diretto è un prodotto semidiretto in cui $ \varphi $ manda tutti gli elementi di $ K $ nell'identità su $ H $.
\end{remark}
\begin{remark}
	Sia $ \bar{H} = H \times {e_k} $. Si ha $$  \ker\Pi_K =  \bar{H} \lhd H \rtimes_\varphi K  $$ qualunque sia l'omomorfismo $ \varphi$. Infatti $ \bar{H} $ è il nucleo dell'omomorfismo di proiezione su $ K $.
\end{remark}
\begin{remark}
	Inoltre $ \bar{K}$ se e solo se il prodotto è diretto. $$  \bar{K} \lhd H \rtimes_\varphi K  \;\Leftrightarrow\; \rtimes = \times $$
\end{remark}

\begin{theorem}[di decomposizione]
	Siano $ G $ un gruppo e $ H, K < G $ sottogruppi. Se
	\begin{enumerate}
		\item $ H \lhd\, G $
		\item $ HK = G $
		\item $ H \cap G = \{e\} $
	\end{enumerate}
allora $$  G \cong H \rtimes_\varphi K  $$ dove $ \varphi $ manda $ k $ nella corrispondente azione di coniugio \fun{\varphi}{K}{\Aut{H}}{k}{\varphi_k : h \mapsto hkh^{-1}}
\end{theorem}
\begin{proof}
	Consideriamo la stessa mappa che già avevamo preso in considerazione parlando di prodotti diretti \fun{\Phi}{H \rtimes_\varphi K}{G}{(h, k)}{hk}
	questo
	\begin{itemize}
		\item è un omomorfismo, perché \begin{align*}
			\Phi((h, k)(h', k')) &= \Phi(h\varphi_k(h'), kk') \\
			&= \Phi(hkh'k^{-1}, kk') \\
			&= hkh'k^{-1}kk' \\
			&= hkh'k'\\
			&= \Phi(h, k)\,\Phi(h', k')\\
		\end{align*}
		\item è iniettivo e suriettivo per le ipotesi, rispettivamente, 2 e 3, come nella decomposizione in prodotto diretto.
	\end{itemize}
dunque $ \Phi $ è un isomorfismo come desiderato.
\end{proof}

\textbf{Esempi.}
\begin{enumerate}
	\item $ \mathcal{S}_n \cong A_n \rtimes_\varphi \langle (1 \; 2) \rangle $, con $ \varphi $ di coniugio.
	\item $ D_n \cong \langle \rho \rangle \rtimes_\varphi \langle \sigma \rangle $, con $ \varphi $ di coniugio.
\end{enumerate}

\columnbreak
\subsubsection{Classificazione dei gruppi di ordine pq}

Se $ p = q $, allora $ |G| = p^2 $, quindi $ G $ è abeliano. In questo caso già sappiamo che
\[ G \cong \mathbb{Z}_{p^2}  \qquad \text{oppure} \qquad G \cong \mathbb{Z}_{p}\times\mathbb{Z}_{p}. \]
Se $ p < q $, allora ho due elementi $ x, y $ di ordine, rispettivamente, $ p $ e $ q $, che generano relativi gruppi ciclici. Il più grande dei quali sarà normale perché ha indice $ p $ (vedi \ref{ppp}):
$$  \mathbb{Z}_p < G \quad\text{e}\quad \Z_q \lhd G $$
Osserviamo inoltre che i due sottogruppi hanno intersezione banale e pertanto $ \Z_p\Z_q = G $ per ragioni di cardinalità. Quindi
\[ G \cong \Z_q \rtimes_{\varphi} \Z_p \]
dove $ \varphi $ è un qualche omorfismo della forma
\fun{\varphi}{\Z_p}{\Aut{\Z_q}  \cong \mathbb{Z}_q^\times \cong \mathbf{GL}_1(\mathbb{F}_q)}{1}{\varphi_1: x \mapsto kx}

Siamo passati in notazione additiva, perché i gruppi in questione sono abeliani e cilici. A questo punto è molto comodo pensare a $ \Z_q $ come spazio vettoriale di dimensione 1 e al suo gruppo di automorfismi come al gruppo delle "matrici" invertibili che vi agisce sopra (che corrisponde al gruppo degli invertibili di $ \mathbb{F}_p $). Abbiamo definito l'omomorfismo solo per il generatore di $ \Z_p $, perché possiamo ottenere gli altri da
\[ y \mapsto \varphi_y : x \mapsto k^yx \]
ed è molto semplice convincersene: stiamo componendo applicazioni lineari, dunque moltiplicando fra loro le corrispondenti matrici.
Non tutte le scelte di $ k \in \Z_q $ sono accettabili però, per esempio se 
$$  p \nmid q-1 = |\Z_q^\times| $$
l'unico omomorfismo $ \varphi $ possibile è quello banale che ci induce
\[ \boxed{G \cong \Z_p \times \Z_q} \]

Se invece 
$ p \mid q -1 $
possiamo scegliere $ k $ come un qualunque generatore del solo sottogruppo di ordine $ p $ di $ \Z_q^\times \cong \Z_{q-1} $. Fissiamo un generatore $ g $, allora tutte le azioni contemplate saranno della forma 
\fun{\psi^m}{\Z_p}{\Aut{\Z_q}}{1}{\psi^m_{1}: x \mapsto g^mx}
al variare di $ m $ in $ \{1, \dots, p-1\} $.
Osserviamo dunque che

\[ \psi^m_1(x) = g^{m}(x) = \psi^1_m(x) \]


e possiamo dunque costruire la funzione
\fun{\Phi}{\Z_q \rtimes_{\psi^m} \Z_p}{\Z_q \rtimes_{\psi^1} \Z_p}{(x, y)}{(x, my)}
e verificare che è un isomorfismo:
\begin{itemize}
	\item è un omomorfismo
	\begin{align*} \Phi(x,\, y)\,\Phi(x',\, y') &= (x,\, my)(x',\, my') \\&= (x + \psi_{my}^1(x'), \,my + my')
	\\&= (x + \psi_y^{m}(x'),\, my + my') \\&= \Phi(x + \psi^m_y(x'),\, y + y')  \end{align*} 
	\item è iniettivo: se $ \Phi(h, k) = (e, e) $, allora $ h = e $ e, poiché $ k^m $ è un automorfismo di $ \Z_p $, $ k = e $.
\end{itemize}
pertanto, in questo caso, esiste \emph{un unico} gruppo di ordine $ pq $ non abeliano. 
\[ \boxed{G \cong \Z_q \rtimes \Z_p} \]



\end{multicols}

\subsection{Teorema di Sylow}
\begin{multicols}{2}
	\begin{definition}
		Chiamiamo $ p $-sylow ogni $ p $-sottogruppo di ordine massimo: un $ H < G $, dove se $ |G| = p^mn $ con $ (p, n) = 1 $, si ha $ |H| = p^m $.
	\end{definition}
\begin{theorem}[di Sylow]\label{sylow}
	Sia G un gruppo finito di ordine $ |G| = p^nm  $, dove $ p $ è primo e $ m $ è un intero a lui coprimo. Allora:
	\begin{itemize}
		\item[$ \exists $.] Per ogni $ 0 \leq \alpha \leq n $, esiste un sottogruppo $ H<G $ di ordine $|H| = p^\alpha $.
		\item[$ \subseteq $.] Ogni $ p $-sottogruppo è incluso in un $ p $-sylow.
		\item[$ \varphi_g $.] Due qualsiasi $ p $-sylow sono coniugati.
		\item[$ n_p $.] Il numero $ n_p $ di $ p $-sylow è congruo a $ 1 \mod{p}$.
	\end{itemize}
\end{theorem}
\begin{proof}
		Dimostriamo i punti nello stesso ordine
	\begin{itemize}

		\item[$ \exists $.] Fissiamo $ 0 \leq \alpha \leq n $. Sia $ \mathcal{M}_\alpha $ l'insieme di tutti i sottoinsiemi di $ G $ di cardinalità $ p^\alpha $ \[ \mathcal{M}_\alpha = \{ M < G \mid |M| = p^\alpha \}, \] possiamo allora calcolarne la cardinalità
		\[ |\mathcal{M}_\alpha| = {p^nm \choose p^\alpha} = p^{n-\alpha}m \prod_{i = 1}^{p^\alpha - 1}\frac{p^nm-i}{p^\alpha -i} \]
		e, in particolare, osservare che
		\[  v_p\left(\left|\mathcal{M}_\alpha\right|\right) = n - \alpha. \]
		Consideriamo ora l'azione di $ G $ su $ \mathcal{M}_\alpha $ per moltiplicazione a sinistra
		\fun{\Psi}{G}{\mathcal{S}(\mathcal{M}_\alpha)}{g}{\psi_g: M \mapsto gM}
		Cerchiamo il sottoinsieme richiesto tra gli stabilizzatori di $ \Psi $. Dato il solito partizionamento in orbite
		\[ |\mathcal{M}_\alpha| = \sum |\Orb{M_i}|, \]
		scopriamo che non tutte le orbite posso essere divisibili per potenze di $ p $ troppo grandi, in particolare
		\[ \exists i \text{ tale che }  p^{n-\alpha +1}\nmid |\Orb{M_i}|. \]
		 Il corrispondente stabilizzatore avrà pertanto cardinalità divisibile per $ p^\alpha $;
		 se fissiamo un elemento $ x \in M_i $ e consideriamo la funzione iniettiva
		 \fun{f}{\Stab{M_i}}{M_i}{g}{gx}
		 scopriamo che lo stabilizzatore non può avere una cardinalità maggiore dell'insieme che stabilizza
		 \[ p^\alpha \mid |\Stab{M_i}| \leq |M_i| = p^\alpha \]
		 ed è dunque il sottogruppo che cercavamo.
		
		\item[$ \subseteq $.] Sia $ H < G $ un $ p $-sottogruppo $ |H| = p^\alpha $ e $ S $ un $ p $-sylow. Consideriamo l'azione di $ H $ sull'insieme $ X $ delle classi laterali di $ S $ per moltiplicazione a sinistra
		\fun{F}{H}{\mathcal{S}(X)}{h}{\psi_h : gS \mapsto hgS}
		Per la decomposizione in orbite
		\[ m=[G:S]=|X| = \sum\frac{|H|}{|\Stab{gS}|} = \sum\frac{p^\alpha}{p^{e_i}} \]
		ma, non potendo $ p $ dividere $ m $, esiste un laterale $ \bar{g}S $ stabilizzato da tutto $ H $. Ovvero
		\[ h\bar{g}S = \bar{g}S \;\Leftrightarrow\; h \in \bar{g}S\bar{g}^{-1} \; \forall h \in H \]
		Dunque $ H \in \bar{g}S\bar{g}^{-1} $, che è il $ p $-sylow cercato.
		
		\item[$ \varphi_g $.] Siano $ A, B $ $ p $-sylow. Per il punto precedente  \[ \exists g \in G \text{ tale che } A < gBg^{-1} \]
		dunque $ A $ e $ B $, avendo la stessa cardinalità, coincidono.
		\item[$ n_p $.] Consideriamo l'azione di coniugio di un $ p $-sylow $ S $ sull'insieme $ Y $ dei suoi coniugati
		\fun{\Phi}{S}{\mathcal{S}(Y)}{g}{\varphi_g: H \mapsto gHg^{-1}}
		Mostriamo che l'orbita di $ S $ è l'unica banale. Infatti se $ H \in Y $ ha orbita banale significa che è stabilizzato da $ S $, dunque che i due commutano e pertanto il loro prodotto è un sottogruppo di $ G $. \[ |HS| = \frac{|H||S|}{|H \cap S|} = \frac{p^{2n}}{|H \cap S|} \mid p^nm \] Necessariamente $ |H \cap S| = p^n $ e dunque $ H = S $. Per una formula ancora mai usata
		\[ n_p = |Y| = \Orb{S} + \sum_{H \neq S}\frac{|S|}{\Stab{H}} \equiv 1 \mod{p} \]
		Per concludere è sufficiente osservare che se $$  \Stab{H} \lneq S  $$ allora l'orbita corrispondente ha cardinalità divisibile per $ p $.
	\end{itemize}
\end{proof}

\begin{remark}
	$ n_p $ è l'indice del normalizzatore di un $ p$-sylow in $ G $. Infatti, estendendo l'azione di coniugio a tutto il gruppo
	\[ n_p = |\Orb{P}| = \frac{|G|}{|\Stab{P}|} = \frac{|G|}{|N(P)|} = [G : N(P)] \]
\end{remark}

\begin{theorem}
	Ogni gruppo $ G $ abeliano finito è prodotto diretto dei suoi $ p $-sylow.
\end{theorem}
\begin{proof}
	Usiamo la nozione additiva e sia $ G = p^nm $ come al solito. Per ogni divisore $ d$ dell'ordine del gruppo sia
	\[ G_d = \ker\psi_d = \{ g \in G \mid dg = 0 \} \]
	Ci è sufficiente mostrare che
	\[ G \cong G_{p^n} \times G_m \]
	Osserviamo innanzitutto che $ G_{p^n} $ è un p-sylow. Dev'essere un $ p $-gruppo perché se $ |G_{p^n}| $ fosse divisibile per un primo $ q $, allora per il Teorema di Cauchy \ref{cauchy} conterrebbe almeno un elemento di ordine $ q $, contro la sua definizione. A questo punto, dovendo contenere l'unico $ p $-sylow di $ G $  (il coniugio è banale negli abeliani) non può che esserlo.
	Verifichiamo che
	\begin{enumerate}
		\item I due sottogruppi sono normali in $ G $, perché è abeliano.
		\item La loro intersezione è banale, perché tutti gli elementi del $ p $-sylow hanno ordine divisibile per un primo che non divide l'ordine $ m $ dell'altro sottogruppo.
		\item La loro somma è $ G $. Infatti per Bezout esistono interi $ a, b $ tali che
		\[ ap^n + bm = 1 \]
		che moltiplicato per un qualunque elemento di $ g \in G $ diventa
		\[ a(gp^n) + b(gm) = g \]
		Osserviamo che $ gp^n \in G_{m} $, poiché $$  m (gp^n) = (mp^n) g = |G| g = 0  $$
		Analogamente $ gm \in G_{p^n} $ e pertanto la somma dei due sottogruppi contiene $ G $.
	\end{enumerate}
\end{proof}

\textbf{Esercizi.}
\begin{enumerate}
	\item Chi è il 2-sylow di $ \mathcal{S}_4 $?
	\item Chi sono i gruppi di ordine $ 12 $?
\end{enumerate}
\bigskip
\subsubsection{Classificazione dei gruppi di ordine 12}

Quanti possono essere i $ p $-sylow?
I 3-sylow sono necessariamente di ordine $ 3 $, pertanto ciclici, e possono essere $ n_3 = 1 $ o $ 4 $. I 2-sylow sono di ordine $ 4 $, quindi isomorfi a $ \Z_2 \times \Z_2 $ o $ \Z_4 $, e sono $ n_2 = 1 $ o $ 3 $. Se $ P_3 $ non è normale, allora ne ho 4 copie con intersezione banale e rimane spazio solo per un $ P_2 $, che sarà normale. \\

Quindi almeno uno tra un 2-sylow e un 3-sylow sarà normale. Inoltre devono avere intersezione banale e il loro prodotto ha necessariamente cardinalità $ 12 $. Quindi $ G $ è un prodotto semidiretto tra un sylow e l'altro. Analizziamo le varie possibilità

\begin{itemize}
	\item $ \Z_4 \rtimes_\varphi \Z_3 $. Abbiamo
	\[ \varphi : \Z_3 \to \Aut{\Z_4} \cong \Z_2 \]
	che è necessariamente banale. Otteniamo
	\[ \boxed{G \cong \Z_4 \times \Z_3} \]
	\item $ \Z_2 \times \Z_2 \rtimes_\varphi \Z_3 $. Abbiamo
	\[ \varphi : \Z_3 \to \Aut{\Z_2\times\Z_2} \cong \mathcal{S}_3 \]
	la cui immagine dev'essere un sottogruppo di $ \mathcal{A}_3 $. Otteniamo l'automorfismo banale, da cui
	\[  \boxed{G \cong \Z_2 \times \Z_2 \times \Z_3} \]
	e quelli associati a $ \sigma = (1\,2\,3) $ e $ \sigma^2 = (1\,3\,2\,) $, che vogliamo mostrare indurre lo stesso prodotto. Infatti, scelto un $ \varphi $ non banale, possiamo far agire $ G $ sull'insieme dei suoi 3-sylow per coniugio: sia \fun{\Phi}{G}{\mathcal{S}({\text{3-sylow di G}})\cong \mathcal{S}_4}{g}{\varphi_g : H \mapsto gHg^{-1}}
	Osserviamo che $ N(P_3) = P_3 $, per la formula delle classi. Allora $$  \ker\Phi = \bigcap \Stab{H} = \bigcap N(H) = \bigcap H = \{ e \}  $$ dunque $ \Phi $ è iniettivo e mappa $ G $ in un sottogruppo di ordine 12 di $ \mathcal{S}_4 $. Ma l'unico sottogruppo di questa dimensione è $ \mathcal{A}_4 $, quindi entrambi i gruppi generati dal prodotto non diretto sono isomorfi a questo sottogruppo.
	\[ \boxed{G \cong \mathcal{A}_4} \]
	\item $ \Z_3 \rtimes_\varphi \Z_4 $. Abbiamo
	\[ \varphi : \Z_4 \to \Aut{\Z_3} \cong \Z_2 \]
	che può essere solo $ \pm id $. Il caso banale ci restituisce un prodotto diretto, già considerato, l'altro è un gruppo buffo
	\[ \boxed{G \cong \Z_3 \rtimes_{-id} \Z_4} \]
	\item $ \Z_3 \rtimes_\varphi \Z_2 \times \Z_2  $. Abbiamo
	\[ \varphi : \Z_2 \times \Z_2  \to \Aut{\Z_3} \cong \Z_2 \]
	e abbiamo, oltre all'omomorfismo banale, 3 modi di proiettare $ \Z_2 \times \Z_2  $ su un suo fattore. A meno di isomorfismi di $ \Z_2 \times \Z_2  $, $ \Z_3 $ commuta con uno dei fattori e agisce con $ -id $ sull'altro quindi
	\[ \boxed{G \cong \Z_3\times \Z_2 \rtimes_{-id} \Z_2 \cong D_6} \]
\end{itemize}




\end{multicols}

\subsection{Automorfismi di un gruppo buffo}
\begin{multicols}{2}
	Vogliamo scoprire chi è $ \Aut{Q_8 \times D_4} $. Per far questo, possiamo scomporre un qualunque automorfismo $ \varphi $ nelle sue restrizioni ai due termini del prodotto e proiettarli sulle due componenti. Il seguente diagramma magico è molto esplicativo
	\[ \begin{tikzcd}[row sep = 0.05 cm]
	Q_8 \arrow[swap, hook]{dr}{} & & & Q_8 \\
	 & G \arrow{r}{\varphi} & G \arrow[swap]{ur}{\Pi_Q}\arrow[swap]{dr}{\Pi_D} & \\
	D_4 \arrow[swap, hook]{ur}{} &  &  & D_4
	\end{tikzcd}\]
	Dunque possiamo scomporre l'automorfismo nei quattro omomorfismi
	\[ \varphi= \left(\begin{matrix}
	\alpha & \beta \\
	\gamma & \delta \\
	\end{matrix}\right) \]
	dove 
	\begin{align*}
		\alpha: Q_8 \rightarrow Q_8 \qquad & \qquad \beta: D_4 \rightarrow Q_8 \\
		\gamma: Q_8 \rightarrow D_4 \qquad & \qquad \delta: D_4 \rightarrow D_4
	\end{align*}
	 Iniziamo ad analizzare i possibili omomorfismi.
	 \begin{itemize}
	 	\item [$ \beta $.] Consideriamo le possibili immagini per dimensione, tra i sottogruppi dei quaternioni:
	 	\begin{itemize}
	 		\item [$\{e\}  $.][$ \checkmark $] Ovviamente abbiamo un omomorfismo banale.
	 		\item [$ \mathbb{Z}_2 $.][$ \checkmark $] Il nucleo dev'essere un sottogruppo di indice 2 e il diedrale ne ha tre: $ \langle \rho \rangle, \langle \rho^2, \sigma \rangle, \langle \rho^2, \sigma\rho \rangle $.
	 		\item [$ \mathbb{Z}_4 $.] L'unico sottogruppo di indice $ 4 $ del diedrale è $ \langle \rho \rangle \cong \mathbb{Z}_4 $ ed è il nucleo di un omomorfismo che uccide i termini di ordine $ 4 $.
	 		\item [$ Q_8 $.] Non è possibile, sarebbe un isomorfismo!
	 	\end{itemize}
	 	Tutti questi omomorfismi preservano necessariamente i centralizzatori, perché l'unico sottogruppo dei quaternioni di ordine $ 2 $ è il centro. Dunque sembrano accettabili tutti gli omomorfismi
	 	\[ \boxed{\beta: D_4 \to Z(Q_8)} \]
	 	
	 	\item [$ \gamma $.] Consideriamo le possibili immagini, per dimensione:
	 	\begin{itemize}
	 		\item [$\{e\}  $.][$ \checkmark $] Ovviamente abbiamo un omomorfismo banale.
	 		\item [$ \mathbb{Z}_2 $.][$ \checkmark $] Il nucleo dev'essere un sottogruppo di indice 2 e i quaternioni ne hanno tre: $ \langle i \rangle, \langle j \rangle, \langle k \rangle $.
	 		\item [$ \mathbb{Z}_4 $.] L'unico sottogruppo di indice $ 4 $ dei quaternioni è $ \{\pm 1\} $ ed è il nucleo di un omomorfismo che uccide i termini di ordine $ 4 $.
	 		\item [$ \mathbb{Z}_2\times\mathbb{Z}_2 $.] Possiamo mandare i quaternioni in $ \left(\mathbb{Z}_2\right)^3 $ usando i tre omomorfismi con immagine $ \mathbb{Z}_2 $, questo omomorfismo non sarà suriettivo, altrimenti sarebbe un isomorfismo, e ha almeno $ 4 $ elementi nell'immagine, visto che gli omomorfismi di sopra sono distinti. Quindi, permutando le componenti opportunamente, otteniamo $ 6 $ omomorfismi.
	 		\item [$ D_4 $.] Non è possibile, sarebbe un isomorfismo!
	 	\end{itemize}
	 	possiamo però escludere alcuni omomorfismi osservando che l'automorfismo $ \varphi $ deve preservare i centralizzatori. Infatti osservando il magico diagramma
	 	\begin{align*}
	 		Q_8 &\hookrightarrow Q_8 \times D_4 &\rightarrow Q_8 \times D_4 \\
	 		i &\mapsto (i, e) &\mapsto (\alpha(i), \gamma(i))
	 	\end{align*}
		 scopriamo che $ Z(i, e) \cong \mathbb{Z}_4 \times D_4 $. Possiamo ora cercare di capire cosa dovrebbe essere $ Z(\alpha(i))\times Z(\gamma(i)) $, per esempio elencando i possibili prodotti di sottogruppi di ordine $ 32 $
		 \begin{itemize}
		 	\item [$ Q_8 \times \left(\mathbb{Z}_2\right)^2 $.]
		 	Che però ha solo $ 25 $ elementi di ordine $ 2 $.
		 	\item [$ Q_8 \times \mathbb{Z}_4 $.]
		 	Che ha sol $ 11 $ elementi di ordine $ 2 $.
		 	\item [$ \mathbb{Z}_4 \times \left(\mathbb{Z}_2\right)^2 $.] Che però è abeliano.
		 	\item [$ \mathbb{Z}_4 \times \mathbb{Z}_4 $.] Che è abeliano.
		 	\item [$ \mathbb{Z}_4 \times D_4$.][$ \checkmark $] Che sicuramente è il gruppo che cerchiamo.
		 \end{itemize}
	 quindi necessariamente il centralizzatore di $ Z(\gamma(i)) \cong D_4 $ e pertanto $ \gamma(i) $ è un elemento del centro di $ D_4 $, che ha solo due elementi. Quindi gli omomorfismi $ \gamma $ accettabili sono solo quello banale e i tre che hanno immagine in $ \mathbb{Z}_2 $. Dunque sembrano accettabili tutti gli omomorfismi
	 \[ \boxed{\gamma: Q_8 \to Z(D_4)} \]
	 
	 \item [$ \alpha $.] Dev'essere un \underline{isomorfismo}. Se non fosse un isomorfismo l'immagine non potrebbe avere dimensione $ 4 $, perché come già visto i sottogruppi di indice adatto eliminano gli elementi di ordine $ 4 $, e non potrebbe avere dimensione più piccola, perché altrimenti il primo termine dell'immagine di $ \varphi $ apparterrebbe sempre al centro di $ G $.
	 
	 \item [$ \delta $.] Analogamente dev'essere un \underline{isomorfismo}. 
	\end{itemize}
Mostriamo ora che le condizioni trovate sono sufficienti. Ci basta mostrare che $ \varphi $, costruito con le componenti sopra trovate, è iniettivo. Supponiamo di aver trovato $ (x, y) \in G $ tale che
\[ \varphi(x, y) = (\alpha(x)\beta(y), \gamma(x)\delta(y)) = (e, e) \]
Visto che $ \beta(y) $ e $ \gamma(x) $ stanno nei centri dei rispettivi insiemi, anche $ \alpha(x) $ e $ \delta(y) $, che sono i loro inversi, vi staranno. Ma $ \alpha $ e $ \delta $ sono isomorfismi, pertanto anche $ x, y $ staranno nei centri dei loro rispettivi gruppi! Ma $ \beta $ e $ \gamma $ contengono i centri nei loro nuclei, quindi si annullano, così come i rispettivi isomorfismi. Così $ x, y $ sono necessariamente l'elemento neutro del proprio gruppo e $ \ker\varphi = \{(e, e)\} $. \\

Conosciamo già $ \Aut{D_4} $, cerchiamo, per concludere, di capire chi sia{\tiny } $ \Aut{Q_8} $. \\

Ogni automorfismo $ \alpha $ di $ Q_8 $ deve mandare $ \alpha(-x) = -\alpha(x) $, quindi le coppie 
\[ (i, -i) \quad (j, -j) \quad (k, -k) \]
non vengono scisse, ma solo permutate fra loro. Possiamo quindi far agire $ \Aut{Q_8} $ sull'insieme di queste tre coppie, costruendo così un omomorfismo
\[ \xi : \Aut{Q_8} \to \mathcal{S}_3 \]
Il nucleo di $ \xi $ è costituito dagli automorfismi che non scambiano nessuna coppia, dunque quello identico e i tre che cambiano segni a due delle coppie, ed è dunque isomorfo a $ \Z_2 \times \Z_2 $.

Se consideriamo ora gli isomorfismi
\[ S: \begin{cases}
i \mapsto j \\ j \mapsto i \\ k \mapsto k
\end{cases}
\qquad T: \begin{cases}
i \mapsto j\\
j \mapsto k\\
k \mapsto i
\end{cases} \]


questi generano un sottogruppo "disgiunto" da $ \Z_2 \times \Z_2 $ isomorfo a $ \mathcal{S}_3 $, quindi
\[ \boxed{\Aut{Q_8} \cong \Z_2 \times \Z_2 \rtimes_\phi \mathcal{S}_3} \] Per una certa azione $ \phi $ che rende il gruppo $ \mathcal{S}_4 $ (per ragioni magiche non dimostrate).




\end{multicols}


\subsection{Teorema Fondamentale dei Gruppi Abeliani Finiti}
\begin{multicols}{2}
\begin{theorem}[di Struttura dei Gruppi Abeliani Finiti]\label{tfgaf}
	Se $ G $ è un gruppo abeliano finito allora si decompone in modo unico come prodotto diretto di gruppi ciclici \[ G \cong \mathbb{Z}_{n_1} \times \cdots \times \mathbb{Z}_{n_s} \] con $ n_1 \mid \dots \mid n_s $.
\end{theorem}
\begin{proof}
	Avendo già dimostrato che ogni gruppo abeliano finito si decompone nel prodotto dei suoi $ p $-sylow ci è sufficiente dimostrare la tesi per i $ p $-gruppi. Dato gruppo abeliano $ G $ di ordine $ p^n $, ci basta mostrare che possiamo scriverlo come prodotto diretto del generato da un suo elemento di ordine massimo $ g $ e un altro sottogruppo $ K $
	\[ G \cong \langle g \rangle \times K \]
	così da poter procedere per induzione. \\
	
	Mostriamo questo risultato intermedio per induzione sull'ordine del $ p $-gruppo $ G $. Se $ |G|=p $ allora il gruppo è ciclico ed è generato da $ g $. Supponiamo ora la tesi vera per ogni $ k $ con $ 1 \leq k < n $ e prendiamo $ g $ un elemento di ordine massimo, diciamo $ p^m $. Prendiamo ora un elemento $ h \in G  $ che non stia nel sottogruppo $ \langle g \rangle $ e in modo che abbia ordine minimo possibile, se non esiste abbiamo $ G = \langle g \rangle $ e abbiamo finito. \\
	
	Vogliamo ora mostrare che 
	\[ \langle g \rangle \cap \langle h \rangle = \{e\} \]
	L'ordine di $ h^p $ è ovviamente minore di quello di $ h $, dunque $ h^p \in \langle g \rangle $, ovvero esiste un intero $ r \in \mathbb{Z} $ tale che
	\[ h^p = g^r \]
	L'ordine di $ g^r $ è al più $ p^{m-1} $: $$  (g^r)^{p^{m-1}} = (h^p)^{p^{m-1}} = e $$ pertanto non è un generatore di $ \langle g \rangle $, dunque per un qualche intero $ s $ abbiamo
	\[ h^p = g^r = g^{ps} \]
	e succede che 
	\[ (g^{-s}h)^p = g^{-sp}h^p = e \]
	esiste un elemento (ovvero $ g^{-s}h $)  di ordine $ p $ che non appartiene a $ \langle g \rangle $! Quindi anche l'ordine di $ h $ è $ p $ e i due sottogruppi devono essere disgiunti. \\
	
	Osserviamo ora che, detto $ H = \langle h \rangle $, l'ordine di $ gH $ in $ \faktor{G}{H} $ è lo stesso di $ g $ in $ G $, in particolare è ancora massimo. Se fosse più piccolo, sarebbe al più $ p^{m-1} $ e
	\[ H = (gH)^{p^{m-1}} = g^{p^{m-1}}H \]
	e pertanto $ g^{p^{m-1}} \in H $, assurdo.
	 Per l'ipotesi induttive e il teorema di corrispondenza
	\[ \faktor{G}{H} \cong \langle gH \rangle \times \faktor{K}{H} \]
	per un certo sottogruppo $ H < K < G $. Mostriamo che $ K $ è il sottogruppo che cercavamo
	\begin{itemize}
		\item $ \langle g \rangle \cap K = \{e\} $. Infatti se $ b $ stesse nell'intersezione, $ bH $ apparterrebbe all'intersezione $ \langle gH \rangle \cap \faktor{K}{H} $ che è $ H $, dunque $ b \in H $.
		\item $ G = \langle g \rangle K $. Per ragioni di cardinalità. 
	\end{itemize}


L'unicità è noiosa e poco interessante e, pertanto, lasciata al lettore. 

\end{proof}
	\columnbreak
\subsubsection{Classificazione dei Gruppi di Ordine 30}
Tiriamo a caso qualche gruppo di quest'ordine
\[ \boxed{\mathbb{Z}_2\times\mathbb{Z}_3\times\mathbb{Z}_5} \qquad \boxed{D_{15}} \qquad \boxed{D_5\times\mathbb{Z}_3} \qquad \boxed{D_{3}\times\mathbb{Z}_5} \]
questi sono distinti perché il primo è l'unico abeliano e i centri di dei seguenti sono rispettivamente $ \{e\},\, \mathbb{Z}_3,\, \mathbb{Z}_5 $. Sappiamo che
\[ n_5 \equiv 1 \mod{5} \qquad\text{e}\qquad n_5 \mid 6 \]
per il Teorema di Sylow \ref{sylow} e perché $ n_5 \mid |G| $ in quanto cardinalità dell'orbita dell'azione di coniugio, rispettivamente. E, analogamente
\[ n_3 \equiv 1 \mod{3} \qquad\text{e}\qquad n_3 \mid 10 \]
Allora, se $ P_5 $ non è normale, ci sono sei 5-sylow, quindi 24 elementi di ordine 5. Tra i pochi elementi che rimangono non ci stanno sicuramente dieci 3-sylow e pertanto $ P_3 $ è normale. Dunque almeno uno tra $ P_5 $ e $ P_3 $ è normale. Allora $ P_3 $ e $ P_5 $ commutano (perché uno dei due è contenuto nel normalizzatore dell'altro), dunque
\[ P_3P_5 < G \]
e avendo indice $ 2 $ è normale, nonché ciclico.

Abbiamo allora che
\[ G \cong \mathbb{Z}_{15} \rtimes_\varphi \mathbb{Z}_2 \]
per una qualche azione di coniugio 
\fun{\varphi}{\mathbb{Z}_2}{\Aut{\mathbb{Z}_{15}}\cong \mathbb{Z}_5^\times\times\mathbb{Z}_3^\times}{y}{\varphi_y: x \mapsto yxy^{-1} = x^a}
sapendo che $ \varphi_y^2(x) = x^{a^2} = x $, dobbiamo avere che $$  a^2 \equiv 1 \mod{15}  $$ e risolvendo il sistema di diofantee troviamo
\[ a = \pm 1,\, \pm 4 \mod{15} \]
e ognuna di questa azioni induce un prodotto semidiretto isomorfo a uno dei gruppi trovati all'inizio. In particolare $ a = 1 $ è l'automorfismo identico, che induce il prodotto diretto, che restituisce il gruppo abeliano, mentre $ a = -1 $ sappiamo già essere l'omomorfismo che genera il gruppo diedrale. Per $ a = 4 $ troviamo l'automorfismo che fissa $ \mathbb{Z}_3 $, per $ a = -4 $ quello che fissa $ \mathbb{Z}_5 $, in entrambi i casi uno dei fattori a sinistra del prodotto semidiretto commuta anche col fattore di destra, siamo così autorizzati a raccoglierlo all'esterno per ottenere, rispettivamente, $ D_5\times\mathbb{Z}_3 $ e $ D_{3}\times\Z_5 $.
	
	
\end{multicols}

\subsection{Lemmi vari ed esercizi sparsi}

\begin{multicols}{2}
	
	\begin{prop}[del piccolo indice primo]\label{ppp}
		Siano $ G $ un gruppo finito e $ H $ un sottogruppo che ha come indice il più piccolo primo $ p $ che divide $ G $, allora $ H \lhd G $.
	\end{prop}
	\begin{proof}
		Consideriamo l'azione di $ G $ sull'insieme $ X $ dei laterali di $ H $ per moltiplicazione a sinistra \fun{\Phi}{G}{\mathcal{S}_p}{g}{\Pi_g : xH \mapsto gxH}
		Osserviamo che 
		\begin{align*}
		g \in \Stab{xH} &\Leftrightarrow gxH = xH \\
		&\Leftrightarrow x^{-1}gx \in H \\
		&\Leftrightarrow g \in xHx^{-1} \\
		\end{align*}
		dunque $ \Stab{xH} = xHx^{-1} $ è il sottogruppo coniugato di $ H $ rispetto ad $ x $. Possiamo ora riscrivere il nucleo come
		\[ \ker\Phi = \bigcap_{x \in G} xHx^{-1} < H \]
		e osservare che, per il Primo Teorema di Omomorfismo
		\[ \Phi ' : \faktor{G}{\ker\Phi} \rightarrow \mathcal{S}_p \]
		è iniettivo e pertanto
		\[ \left| \faktor{G}{\ker\Phi} \right| \mid |\mathcal{S}_p| = p! \]
		ma $ p $ era il più piccolo primo a dividere $ |G| $, quindi non potendo $ \ker\Phi $ coincidere con tutto il gruppo, dovrà essere proprio $ H $. Il che conclude la dimostrazione.
	\end{proof}
	
	\begin{prop}
		Sia $ G $ un gruppo di ordine $ 2^kd $, dove $ d $ è dispari, con il 2-sylow ciclico. Allora esiste un sottogruppo di indice 2.
	\end{prop}
	\begin{proof}
		Il Teorema di Cayley ci fornisce l'immersione seguente, per moltiplicazione sinistra
		\begin{align*}
		\Phi \colon & G \hookrightarrow \S_{2d} \\
		& g \mapsto \varphi_g \colon x \mapsto gx
		\end{align*}
		Dato un elemento $ g $, conosciamo la decomposizione in cicli di $ \varphi_g $, questa dev'essere prodotto di cicli del tipo
		$$  (x \; gx \; \cdots \; g^{m-1}x)  $$
		al variare di $ x $ in un opportuno sistema di rappresentanti e dove $ m $ è l'ordine di $ g $. Esiste un elemento di ordine $ 2^k $, che avrà dunque come immagine il prodotto di $ d $ $2^k $-cicli, visto che la moltiplicazione sinistra è un'azione transitiva. Pertanto $ G \nless \mathcal{A}_{2d} $ e quindi $$  [G : G \cap \mathcal{A}_{2d}] = 2  $$
	\end{proof}
	\columnbreak
	\begin{prop}
		Sia $ G $ un gruppo semplice e finito. Se esiste un sottogruppo $ H < G $ di indice $ n $, allora esiste un'immersione di $ G $ in $ \mathcal{A}_n $.
	\end{prop}
	\begin{proof}
		Facendo agire $ G $ per moltiplicazione sinistra sull'insieme degli $ n $ laterali di $ H $ otteniamo un omomorfismo
		
		\[ \Phi \colon G \to \mathcal{A}_n \]
		
		il cui nucleo dev'essere però banale, per non contraddire la normalità di $ G $. Ovviamente l'omomorfismo non è banale, dunque $ \Phi $ è un'immersione.
	\end{proof}
	
	
	\begin{remark}
		Sia $ G $ un gruppo abeliano. Sia \fun{\psi_n}{G}{G}{x}{x^n} preso un qualunque automorfismo $ \varphi \in \Aut{G} $ il seguente diagramma è commutativo
		\[ \begin{tikzcd}
		G \arrow{r}{\psi _n} \arrow[swap]{d}{\varphi} & G \arrow{d}{\varphi} \\
		G \arrow{r}{\psi _n} & G
		\end{tikzcd}
		\]
		quindi $ \ker\psi _n $ e $ \psi _n(G) $ sono caratteristici in $ G $. \\
		
	\end{remark}
	
	\textbf{Esercizi.}
	\begin{enumerate}
		\item Trova $ \Aut{\mathbb{Z}\times\mathbb{Z}_n} $.
		\item Trova $ \Aut{\mathbb{Z}_2\times\mathbb{Z}_4\times\mathbb{Z}_4} $.
		\item Trova $ \Aut{Q_8 \times D_4} $.
		\item Sia $ G $ un gruppo abeliano finito. Se $ H \lhd G $ è ciclico e lo è anche il loro quoziente, allora anche $ G $ è ciclico.
		\item Dati $$  H \lhd K \lhd G  $$ quali inclusioni devono essere caratteristiche per far si che $ H $ sia normale o caratteristico in $ G $?
		
		\item Sia $ G $ un gruppo finito. Se esiste un sottogruppo $ H < G $ di indice $ n $, allora esiste un sottogruppo normale $ N \lhd G $ di indice divisore di $ n! $.
		
		\item Un gruppo di ordine 112 non è semplice.
		
		\item Un gruppo di ordine 144 non è semplice.
		
		\item Quanti sono i $ p $-sylow di $ \textbf{GL}_n(\mathbb{F}_p) $?
		
		\item Dato un gruppo di ordine $ |G|= p^3 $
		\begin{enumerate}
			\item dimostrare che $ |Z(G)| = p $.
			\item dimostrare che $ G' = Z(G) $.
			\item contare il numero di classi di coniugio.
		\end{enumerate}
		
		\item In un $ p $-gruppo, il centro di uno stabilizzatore di un elemento non nel centro è più grande del centro di tutto il gruppo.
		
		\item $ \Z_n^\times $ è ciclico se e solo se $ n = 2, 4, p^n, 2p^n $.
		
	\end{enumerate}
\end{multicols}
	

\setcounter{section}{0} % per scegliere la lettera giusta
\section{Anelli}
\subsection{Prime definizioni}
\begin{multicols}{2}
Per "Anello" si intende un anello commutativo con identità.

\begin{definition}[Divisori di 0]
	Un elemento di un anello $ x \in A $ si dice \emph{divisore di zero} quando
	\[ \exists y \in A \quad\text{    tale che    }\quad yx = 0 \]
	Un anello $ A $ si dice \emph{dominio d'integrità} quando l'unico divisore di zero è lo 0 stesso
	\[ \mathcal{D}:= \{\text{divisori di zero}\} = \{0\} \]
\end{definition}
\begin{definition}[Nilpotenza]
	Un elemento di un anello $ x \in A $ si dice \emph{nilpotente} quando
	 \[ \exists n \in \mathbb{N} \quad\text{    tale che    }\quad x^n = 0 \]
	 Un anello $ A $ si dice \emph{ridotto} quando l'unico elemento nilpotente è 0
	 \[ \mathcal{N}: = \{\text{nilpotenti}\} = \{0\} \]
\end{definition}

\begin{theorem}[Prime proprietà]
	Valgono queste proprietà
	\begin{enumerate}
		\item $ A^\times $ è un gruppo moltiplicativo.
		\item $ A^\times \cap \mathcal{D} = \phi $.
		\item Se $ A $ è finito $ A = A^\times \sqcup \mathcal{D} $.
	\end{enumerate}
\end{theorem}
\begin{remark}
	Un dominio d'integrità finito è un campo.
\end{remark}

\subsubsection{Anelli di polinomi}
Dato un anello $ A $, consideriamo il corrispondente anello dei polinomi a coefficienti in $ A $: $$  A[\,x\,] = \{ a_0 + \dots + n x^n \mid a_k \in A \}  $$

\textbf{Quali sono gli elementi nilpotenti di $ A[\,x\,] $ ?} \\

Prendiamo un polinomio nilpotente 
\[ f = a_0 + \dots + a_nx^n \]
e osserviamo che il termine di grado maggiore in $ f^k $ è $ a_n^kx^{nk} $. Quindi, affinché $ f $ sia nilpotente, $ a_n $ dev'essere nilpotente in $ A $ e pertanto $ a_nx^n $ sarà nilpotente in $ A[\, x\,] $. Osserviamo che \emph{la somma di elementi nilpotenti è nilpotente}:
\[ a^n = 0,\, b^m = 0 \quad\Rightarrow\quad (a+b)^{n+m} = 0 \]
per concludere che anche $ f-a_nx^n $ sarà nilpotente, e quindi, iterando
\[ \boxed{f \in \mathcal{N}(A[\,x\,]) \LR a_0, \,\dots,\, a_n \in \mathcal{N}(A)}  \]

La freccia inversa è banale: elevando il polinomio a una potenza sufficientemente alta, otteniamo prodotti di potenze dei coefficienti tali che in ogni prodotto compare almeno un coefficiente con potenza maggiore del suo indice di nilpotenza. \\

\textbf{Quali sono gli elementi invertibili di $ A[\,x\,] $ ?} \\

Procediamo come prima, prendendo un polinomio invertibile
\[ f = a_0 + \dots + a_nx^n \]
e il suo inverso
\[ g = b_0 + \dots + b_mx^m \quad \text{tale che } fg = 1 \]
e osserviamo le relazioni tra i coefficienti di $ fg $ e quelli dei fattori:
\begin{align*}
	1 & = a_0 b_0 \\
	0 & = a_0 b_1 + a_1b_0 \\
	&  \vdots \\
	0 & = a_nb_{m-2} + a_{n-1}b_{m-1}+ a_{n-2}{b_m}\\
	0 & = a_nb_{m-1} + a_{n-1}b_m \\
	0 & = a_nb_m
\end{align*}

Innanzitutto, $ a_0 $ e $ b_0 $ sono invertibili. Moltiplicando la penultima relazione per $ a_n $ otteniamo
\[ 0 = a_n^2b_{m-1} + a_{n}a_{n-1}b_m = a_n^2b_{m-1} \]
moltiplicando la terzultima per $ a_n^2 $
\[ 0 = a_n^3b_{m-2} \]
e iterando
\[ 0 = a_n^{m+1}b_0 \]
ma $ b_0 $ è invertibile, quindi non può essere un divisore di zero, pertanto $ a_n^{m+1} = 0 $. Procedendo come prima
\[ \boxed{f \in A[\,x\,]^\times \LR a_0 \in A^\times \text{ e } f-a_0 \in \mathcal{N}(A[\,x\,])} \]

Per la freccia inversa scriviamo 
\[ f = a_0 + g \quad \text{con } a_0 \in A^\times \text{ e } g \in \mathcal{N}(A[\,x\,])  \]
e consideriamo i polinomi
\[ h_m(x) = a_0^{2m} - a_0^{2m-1}g + a_0^{2m-2}g^2 + \dots + g^{2m}  \]
che moltiplicati per $ f $ restituiscono
\[ fh_m = a_0^{2m+1} + g^{2m+1}  \]
scelto $ m $ maggiore dell'indice di nilpotenza di $ g $, abbiamo che $$  a_0^{-(2m+1)}f h_{m} = 1  $$

\textbf{Quali sono i divisori di zero in $ A[\,x\,] $ ?} \\

Prendiamo un polinomio divisore di zero
\[ f = a_0 + \dots + a_nx^n \]
e un polinomio 
\[ g = b_0 + \dots + b_mx^m \quad \text{tale che } fg = 0 \]
di grado minimo possibile.
Consideriamo, al variare di $ 0 \leq k \leq n $, i prodotti
 $ a_k q $.
Se questi non sono tutti nulli, consideriamo il più grande $ k $ tale che \[  a_kq \neq 0 \]
Allora
\[ 0 = fg = (a_0 + \dots a_kx^k) (b_0 + \dots + b_mx^m) \]
osserviamo che il coefficienti del termine di grado massimo è
\[ a_kb_m = 0 \]
e quindi
\[ p \cdot (a_k q) = 0 \]
ma $ \deg{a_k q} < \deg{q} $, quindi dobbiamo avere necessariamente che $ a_kq = 0 \;\forall k $, quindi che $ a_kb_0 = 0 \;\forall k $, ossia
\[ \boxed{f \in \mathcal{D}(A[\,x\,]) \LR \exists b \neq 0 \in A \mid b a_k = 0 \;\;\forall k} \] 


\subsection{Fatti su Ideali}
\begin{definition}[ideale]
	Chiamiamo \emph{ideale} un sottogruppo additivamente chiuso $ I \subseteq A $ che assorbe per moltiplicazione.
\end{definition}

\begin{remark}
	Abbiamo una bella caratterizzazione degli ideali propri
	\[ I \subsetneq A \;\Leftrightarrow\; I \cap A^\times = \phi \;\Leftrightarrow\; 1 \notin I \]
	(è quasi completamente ovvia guardando la contronominale)
\end{remark}


\begin{remark}
	Un campo $ \mathbb{K} $ ha solo ideali banali.
\end{remark}

\begin{definition}[Omomorfismo]
	Una funzione tra due anelli $ A $ e $ B $ è detta omomorfismo di anelli se
	\begin{enumerate}
		\item è omomorfismo dei rispettivi gruppi additivi.
		\item $ f(ab)=f(a)f(b) $
		\item $ f(1_A)=f(1_B) $
	\end{enumerate}
\end{definition}

\begin{theorem}[di omomorfismo]\label{omoanelli}
	Dato un omomorfismo di anelli $ f $ e un ideale $ I \subseteq \ker f $, esiste ed è unico l'omomorfismo $ \bar{f} $ che fa commutare il seguente diagramma
	\[ \begin{tikzcd}
	A \arrow{r}{f} \arrow[swap]{d}{\pi} & B \\
	\faktor{A}{I} \arrow[swap, dashed]{ru}{\bar{f}}
	\end{tikzcd} \]
\end{theorem}
\begin{proof}
	Come per l'analogo teorema sui gruppi, consideriamo la mappa
	\fun{\bar{f}}{\faktor{A}{I}}{B}{a + I}{f(a)}
	e verifichiamo che è ben definita e un omomorfismo.
\end{proof}
\begin{remark}
	Gli ideali sono tutti e soli i nuclei di omorfismi.
\end{remark}

\begin{theorem}[di corrispondenza]\label{corr}
	Ogni proiezione $$  \Pi : A \to \faktor{A}{I}  $$ induce una relazione biunivoca tra gli ideali del quoziente e gli ideali di $ A $ che contengono $ I $
	\[ \{ \text{ideali di }\faktor{A}{I} \} \leftrightarrow \{ \text{ideali di $ A $ che contengono $ I $} \} \]
	che preserva:
	\begin{itemize}
		\item l'ordinamento indotto dall'inclusione.
		\item l'indice.
		\item primalità e massimalità.
	\end{itemize}
\end{theorem}
\begin{proof}
	Osserviamo che la controimmagine di un ideale è sempre un ideale. Infatti, preso $ J \subseteq \faktor{A}{I} $, osserviamo che e la sua controimmagine $ f^{-1}(J) $ assorbe per prodotto
	\[ f(af^{-1}(J)) = f(a)J = J \quad\Rightarrow\quad af^{-1}(J) = f^{-1}(J)  \]
	ed è additivamente chiusa
	\[ f(i + j) = f(i) + f(j) \in J \qquad \forall i, \, j \in f^{-1}(J) \]
	
	Inoltre ogni ideale contiene lo $ 0 $, quindi la controimmagine contiene $ \ker f = I $. Ci basta mostrare che l'immagine di ogni ideale, attraverso un omomorfismo suriettivo, è ancora un ideale.
\end{proof}
\begin{remark}
	Vale sempre che
	\[ IJ \subseteq I \cap J \]
	c'è uguaglianza in caso di \emph{comassimalità}
	\[ I + J = A \;\Rightarrow\; IJ = I \cap J \]
\end{remark}
\begin{proof}
	Dato l'assorbimento per moltiplicazione
	\[ IJ \subseteq J, \quad JI \subseteq I \quad\Rightarrow\quad IJ \subseteq I \cap J  \]
	Se $ I $, $ J $ sono comassimali, allora
	\[ I + J = A \quad\Rightarrow\quad \exists i \in I, \, j \in J \text{ tali che } i + j = 1 \]
	e per ogni elemento $ x \in I \cap J $ si ha che
	\[ x = xi + xj \in IJ \]
	perché entrambi gli addendi appartengono a $ IJ $.
\end{proof}

\begin{theorem}[cinese degli anelli]
	Dati due ideali $ I, J \subseteq A $ comassimali, si ha che
	\[ \faktor{A}{IJ} \cong \faktor{A}{I} \times \faktor{A}{J} \]
\end{theorem}
\begin{proof}
	La mappa
	\fun{f}{A}{\faktor{A}{I} \times \faktor{A}{J}}{a}{(a + I, a + J)}
	è un omomorfismo di nucleo
	\[ \ker f = I \cap J \]
	Inoltre $ f $ è suriettivo se e solo se $ I + J = A $: se $ f $ è suriettivo, prendiamo un elemento $ a $ nella controimmagine di $ (1 + I, J) $, ossia un elemento tale che 
	$ a-1 \in I $ e $ a \in J $.
	Pertanto 
	\[ a-1 = i \quad\Rightarrow\quad 1 = a + i \in I + J \]
	Se invece $ 1 \in I + J $, allora possiamo trovare due elementi tale che \[ 1 = i + j \]
	e scelta una qualunque coppia $ (x + I, y + J) $, abbiamo che
	\[ y + j(x-y) = xj + yi = x + i(y-x) \]
	Dunque, se $ I + J = A $, abbiamo $ \ker f = I \cap J = IJ $ per l'osservazione di sopra.
\end{proof}

\begin{definition}[ideale primo]
	Un ideale $ I \subseteq A $ si dice \emph{primo} quando
	\[ \forall x, y \in A \qquad xy \in I \;\Rightarrow\; x \in I \text{ o } y \in I \]
\end{definition}

\begin{definition}[ideale massimale]
	Un ideale $ I \subseteq A $ si dice \emph{massimale}, se è massimale rispetto all'inclusione tra gli ideali propri di $ A $.
\end{definition}

\begin{theorem}\label{idealequoziente}
	Dato un ideale $ I \subseteq A $, valgono le seguenti relazioni con il rispettivo anello quoziente:
	\[ I \text{ è primo} \quad\Leftrightarrow\quad \faktor{A}{I} \text{ è  un dominio} \]
	\[ I \text{ è massimale} \quad\Leftrightarrow\quad \faktor{A}{I} \text{ è un campo} \]
\end{theorem}

\begin{proof}
	Per la prima proposizione, osserviamo che
	
	\[xy \in I \quad\Leftrightarrow\quad \pi(xy) = 0 
	\quad\Leftrightarrow\quad \pi(x)\pi(y) = 0 \]
	
	e osserviamo che allora
	\[ \pi(x)\pi(y) = 0 \quad\Leftrightarrow\quad \pi(x) =0 \text{ o } \pi(y) = 0   \]
	è equivalente a 
	\[ xy \in I \quad\Leftrightarrow\quad x \in I \text{ o } y \in I  \]
	
	Per la seconda, ci basta invocare il teorema di corrispondenza e ricordare che un anello è anche un campo se e solo se ha solo ideali banali.
\end{proof}

\begin{remark}
	$ A $ è un dominio se e solo se l'ideale $ \{0\} $ è primo.
\end{remark}
\begin{remark}
	$ A $ è un campo se e solo se l'ideale $ \{0\} $ è massimale.
\end{remark}
\begin{remark}
	Un ideale massimale è anche primo, perché ogni campo è anche un dominio.
\end{remark}

\subsubsection{Fatti sul radicale}
\begin{definition}
	$$  \sqrt{I}:= \{ x \in A \mid x^n \in I \}  $$
\end{definition}

\begin{enumerate}
	\item $ \sqrt{I} $ è un ideale. \\ (Infatti è un sottoinsieme additivamente chiuso
	\[ a^n \in I,\;  b^m \in I \quad\Rightarrow\quad (a+b)^{n+m} \in I \]
	e assorbente per moltiplicazione
	\[ a^n \in I \quad\Rightarrow\quad (sa)^{n} = s^na^n \in sI = I ) \]
	
	\item $ I \subseteq \sqrt{I} $.
	\item $ \sqrt{IJ} = \sqrt{I \cap J} = \sqrt{I} \cap \sqrt{J} $. \\ (Osserviamo che
	\[ I \subseteq J \quad\Rightarrow\quad \sqrt{I} \subseteq \sqrt{J} \]
	dunque 
	\[ \sqrt{IJ} \subseteq \sqrt{I \cap J} \subseteq \sqrt{I} \cap \sqrt{J} \subseteq \sqrt{IJ} \]
	dove le prime due inclusioni sono conseguenza dell'osservazione di sopra, mentre l'ultima si ottiene da
	\[ a^n \in I, b^m \in J \quad\Rightarrow\quad (ab)^{m + n} \in I \cap J )\]
	
	\item $ \sqrt{I} = A $ se e solo se $ I = A $.
	\item Se $ P $ è un ideale primo, allora $ \sqrt{P} = P $. \\ 
	(Se $ x^n \in P $, ho $ x = x \cdot x \cdots x $ per $ n $ volte e almeno uno dei fattori sta in $ P $)
	
	\item $ \mathcal{N} = \bigcap_{P \subseteq A} P $. \\ (Chiaramente se $ x $ è nilpotente $ x ^n \in P $ per ogni ideale primo $ P $.
	
	Mostriamo che, viceversa, se $ x $ appartiene a tutti gli ideali primi allora è nilpotente. Ragioniamo per assurdo: fissato $ x $ nell'intersezione, assumiamo $$  x^n \neq 0 \qquad \forall n \in \mathbb{N} $$ Consideriamo la famiglia di ideali
	$$  \mathcal{F} = \{ I \subseteq A \mid x^n \notin I \;\forall n \in \mathbb{N}  \}  $$
	Osserviamo che per ipotesi $ {0} \in \mathcal{F} $, quindi la famiglia non è vuota. Inoltre è induttiva! (L'unione di una catena di ideali è un ideale e la proprietà di famiglia si conserva). Dunque, per il Lemma di Zorn, abbiamo un ideale $ P $ massimale in $ \mathcal F $.
	
	Mostriamo che $ P $ deve essere primo, giungendo all'assurdo. Preso $ ab \in P $ se uno degli ideali 
	\[ P + (a) \qquad \qquad P + (b) \]
	appartenesse alla famiglia $ \mathcal{F} $, allora sarebbe sottoinsieme di $ P $ (e dunque $ a \in P $). Se nessuno dei due appartenesse alla famiglia, allora
	\[ x^n \in P + (a) \qquad x^m \in P + (b) \]
	ma
	\[ x^{n + m} \in (P + (a))(P + (b)) = P \]
	che dunque dev'essere primo.)
	
	\item $ \sqrt{I} = \bigcap_{I \subseteq P} P $. \\ (Il radicale sono i nilpotenti del quoziente, il fatto segue dal teorema di corrispondenza.)
	
\end{enumerate}

\subsubsection{Prodotto diretto tra anelli} Il prodotto cartesiano tra due anelli $ A \times B $, con le operazioni ovvie, è ancora un anello. Inoltre, se ne nessuno dei due è banale \[ (1, 0)(0, 1) = (0, 0) \] non è un dominio.

I nilpotenti nel prodotto
\[ \mathcal{N}_{A \times B} = \mathcal{N}_A \times \mathcal{N}_B \]

Gli invertibili
\[ (A\times B)^\times = A^\times \times B^\times  \]

Fatto piacevole: gli ideali sono tutti e soli i prodotti di ideali. Prendiamo $ I \subseteq A \times B $ e consideriamo le proiezioni $ \pi_A $ e $ \pi_B $. Queste sono omomorfismi suriettivi, quindi $ \pi_A(I) $ è un ideali di $ A $. Basta ora far vedere che $ \pi_A(I) \times \pi_B(I) \subseteq I $, perché l'inclusione opposta è un fatto insiemistico. Se $ (a, y), (x, b) \in I $, deve starci anche $ (a, b) $: ma $ (1, 0)(a, y) = (a, 0) \in I $, analogamente $ (0, b) \in I $, dunque
\[ (a, 0) + (0, b) \in I \]

\subsubsection{Estensione e Contrazione di Ideali} Dati due anelli, ognuno col proprio ideale $ I \subseteq A $ e $ J  \subseteq B $, e un omomorfismo $ f $ tra i due

\[ \begin{tikzcd}
A \arrow{r}{f}  & B \\
I \arrow[swap, hook]{u}{} & J \arrow[swap, hook]{u}{}
\end{tikzcd} \]

Dalla dimostrazione del teorema di Corrispondenza (\ref{corr}), aappiamo che 
\begin{enumerate}
	\item $ f^{-1}(J) $ è sempre un ideale.
	\item $ f(I) $ è un ideale se e solo se l'omomorfismo è suriettivo.
\end{enumerate}

\begin{definition}
	Chiamiamo ideale contratto $ J^c $ la sua controimmagine $$  J^c = f^{-1}(J)  $$
\end{definition}
\begin{definition}
	Chiamiamo ideale esteso $ I^e $ il generato dalla sua immagine
	\[I^e = \left(f(I)\right) \]
\end{definition}

Poiché possiamo spezzare la mappa $ f $ passando per il quoziente
\[ \begin{tikzcd}
A \arrow{r}{f} \arrow[swap]{d}{\pi} & B \\
\faktor{A}{\ker f} \arrow[swap]{ru}{\varphi}
\end{tikzcd} \]
e il comportamento degli ideali attraverso la proiezione $ \pi $ è completamente determinato dal Teorema di Corrispondenza (\ref{corr}), possiamo limitarci a studiare cosa succede attraverso l'omomorfismo $ \varphi $, che essendo iniettivo possiamo scambiare con l'omomorfismo di inclusione.

Prendiamo quindi $ A \subseteq B $, abbiamo
\begin{enumerate}
	\item $ J^c = J \cap A $
	\item $ I^e = (I) $
\end{enumerate}

\begin{theorem}
	La contrazione di un ideale primo è ancora un ideale primo.
\end{theorem}
\begin{proof}
	$ J $ è primo se e solo se $ \faktor{B}{J} $ è un dominio. Consideriamo la mappa composizione 
	\[ \varphi : A \hookrightarrow B \to \faktor{B}{J} \]
	questa è un omomorfismo di nucleo
	\[ \ker\varphi = J \cap A = J^c \]
	Dunque per il primo teorema di omomorfismo esiste un omomorfismo iniettivo
	\[ \faktor{A}{J^c} \hookrightarrow \faktor{B}{J} \]
	e i sottoanelli di domini sono a loro volta domini.
\end{proof}

Teoremi analoghi con ideali massimali, o estensioni, non funzionano. Controesempio?

\end{multicols}

\subsection{Domini}
\begin{multicols}{2}

\subsubsection{Campo dei quozienti. $ \Q(A) $}

\begin{definition}
	Un insieme $ S $ di elementi di un dominio $ A $ è detto \emph{parte moltiplicativa} se
	\begin{enumerate}
		\item è moltiplicativamente chiuso
		\item $ 1 \in S $
		\item $ 0 \notin S $
	\end{enumerate}
\end{definition}

\begin{remark}
	$ A $ è dominio se e sole se $ A - \{0\} $ è parte moltiplicativa.
\end{remark}

Costruiamo il campo dei quozienti del dominio $ A $ come il quoziente di $ A \times (A \setminus \{0\}) $ per la relazione di equivalenza $$  (a, b) \sim (c, d) \quad\Leftrightarrow\quad ad = bc $$
Denotiamo questo insieme con $ \Q(A) $. Indichiamo i suoi elementi come $ \frac{a}{b} $ e definiamo la somma e il prodotto in modo che
\[ \frac{a}{b} + \frac{c}{d} = \frac{ad + bc}{bd} \qquad\qquad \frac{a}{b}\cdot \frac{c}{d} = \frac{ac}{bd} \]

\begin{remark}
	$ (Q(A),\, +,\, \cdot\,) $ sopra definito è un campo e
	\begin{align*}
	q \colon A &\hookrightarrow \Q(A) \\
	a &\mapsto \frac{a}{1}
	\end{align*}
	è omomorfismo iniettivo.
\end{remark}

\begin{theorem}
	$ (Q(A),\, +,\, \cdot\,) $ è il più piccolo campo in cui possiamo immergere $ A $. Nel senso che l'immersione di $ A $ in un campo $ \mathbb{K} $ si estende all'immersione di $ \Q(A) $ in $ \mathbb{K} $.
	\[ \begin{tikzcd}
	A \arrow[hook]{r}{\varphi} \arrow[swap, hook]{d}{q} & \K \\
	\Q(A) \arrow[swap, dashed]{ru}{\bar{\varphi}}
	\end{tikzcd} \]
\end{theorem}
\begin{proof}
	L'estensione è naturale
	\[ \bar{\varphi}: \frac{a}{b} \mapsto \varphi(a) \varphi(b)^{-1} \]
	Una volta verificato che è un omomorfismo, questo dev'essere necessariamente iniettivo perché $ \Q(A) $ è un campo, pertanto ha solo ideali banali.
\end{proof}

\columnbreak
\subsubsection{Divisibilità.}
In modo naturale, diciamo che $ a \mid b $ se esiste $ c  $ tale che $ b = ac $. E' immediato tradurre la relazione di divisibilità in termini di ideali generati, infatti 
\[ a \mid b \quad\Leftrightarrow\quad (b) \subseteq (a) \]

\begin{definition}
	Diciamo che due elementi $ a, a' \in A $ sono \emph{associati} quando si dividono vincendevolmente

\[ (a) = (a') \quad\Leftrightarrow\quad a' = au \;\text{ per } u \in A^\times \]

lo indicheremo con $ a \sim a' $.
\end{definition}

Possiamo definire  i \textsc{mcd} tra due elementi $ a,\, b $ come quegli elementi $ d $ tali che
\begin{enumerate}
	\item $ d \mid a \;$   e   $\; d \mid b $.
	\item Se $ c \mid a\; $ e $ \; c \mid b $, allora $ c \mid d $.
\end{enumerate}
è immediato osservare che gli \textsc{mcd} di una coppia sono associati.

\begin{definition}[Primo]
	Diciamo che un elemento non invertibile e non nullo $ x \in A\setminus(A^\times \cup \{0\}) $ è \emph{primo} se \[ x \mid ab \quad\Rightarrow\quad x \mid a \; \text{ o }\; x \mid b \]
\end{definition}

\begin{definition}[Irriducibile]
	Diciamo che un elemento non invertibile e non nullo $ x \in A\setminus(A^\times \cup \{0\}) $ è \emph{irriducibile} se \[ x = ab \quad\Rightarrow\quad a\in A^\times \; \text{ o }\; b \in A^\times \]
\end{definition}

I seguenti risultati sono la semplice traduzione delle ultime definizioni nel linguaggio degli ideali:

\begin{theorem}\label{prpr}
	Un elemento $ x $ è primo se e solo se l'ideale da lui generato $ (x) $ è primo e non banale.
	\[ x \text{ è primo} \quad\Leftrightarrow\quad (x) \text{ è primo } \neq \{0\}  \]
\end{theorem}

\begin{theorem}\label{mama}
	Un elemento $ x $ è massimale se e solo se l'ideale da lui generato $ (x) $ è massimale tra i principali.
	\[ x \text{ è irriducibile} \quad\Leftrightarrow\quad (x) \text{ è massimale tra i principali} \]
\end{theorem}

\begin{theorem}\label{epei}
	Ogni elemento primo è anche irriducibile.
	\[ x \text{ è primo} \quad\Rightarrow\quad x \text{ è irriducibile} \]
\end{theorem}

\end{multicols}


\subsection{Esercizi Notevoli}
\begin{multicols}{2}

\textbf{Un solo esercizio in algebra non commutativa.}
\begin{itemize}
	\item Chi sono gli ideali di $ \mathcal{M}_{n \times n}(\K) $?

\end{itemize}

\textbf{Serie Formali.}
\begin{itemize}
	\item $ f \in K[[\, x\,]] $ è invertibile se e solo se $ a_0 \neq 0 $.
	
	\item $ (x) $ è l'unico ideale massimale
	
	\item $ \deg(f) = \min \{n \in \mathbb{N} \mid a_n \neq 0 \} $ è una funzione grado.
\end{itemize}


\textbf{Parti Moltiplicative.}


\begin{itemize}
	\item 
\end{itemize}
	
\end{multicols}


\newpage
\subsection{Domini Speciali}
\begin{multicols}{2}
	Diamo un po' di definizioni insieme
	\begin{definition}
		Ci piacciono i seguenti domini speciali:
		\begin{itemize}
			\item [(\textsc{ufd})] Ogni elemento non nullo e non invertibile si scrive in modo unico come prodotto di irriducibili.
			
			\item [(\textsc{pid})] Tutti gli ideali sono principali.
			
			\item [(\textsc{ed})] Esiste una funzione grado \[{d}:{A\setminus\{0\}}\to{\mathbb{N}}\]
			tale che 
			\begin{enumerate}
				\item $ d(a) \leq d(ab) $ su tutti i valori in cui è definita.
				\item Presi comunque due elementi $ a, b $ con $ b \neq 0 $ esistono $ q, r $ tali che
				\[ a = qb + r \]
				e tali che $ d(r) < d(b) $ oppure $ r = 0 $.
			\end{enumerate}
			
		\end{itemize}
	\end{definition}

\textbf{Esempi.}
\begin{enumerate}
	\item $ \Z $ con la funzione $ \mid \cdot\mid $ è un dominio euclideo.
	\item $ \K[x] $ con la funzione $ \deg(\,\cdot\,) $ è un dominio euclideo.
	\item $ \Z[\,i\,] $ con la norma $ N(x) = x\bar{x} $ è un dominio euclideo.
\end{enumerate}

\begin{theorem}
	Vale la seguente catena di implicazioni
	\[ (\textsc{ed})\Rightarrow(\textsc{pid})\Rightarrow(\textsc{ufd}) \]
\end{theorem}
	
\textbf{Controesempi.}
\begin{itemize}
	\item $ \Z[x] $ è \textsc{pid} ma non è \textsc{ed}.
\end{itemize}
	
\subsubsection{Dominio Euclideo}

\begin{theorem}
	Gli elementi invertibili sono tutti e soli quelli di grado minimo.
\end{theorem}
\begin{proof}
	Osserviamo innanzitutto che ha senso parlare di grado minimo, poiché l'immagine del grado è un sottoinsieme non vuoto dei naturali.
	
	Se $ x $ è invertibile, per ogni $ a \in A $ non nullo
	\[ d(x) \leq d(x) d(ax^{-1}) \leq d(x \cdot ax^{-1}) = d(a) \]
	e se per sbaglio $ d(ax^{-1}) = 0 $, allora lo stesso vale per il grado di $ x^{-1} $ e quindi di $ x $.
	
	Viceversa, sia $ x $ un elemento di grado minimo e $ a $ un elemento qualunque. Abbiamo che
	\[ a = qx + r \]
	e, escludendo $ d(r) < d(x) $, dobbiamo avere che $ r = 0 $. Quindi $ x $ divide ogni elemento ed è pertanto invertibile. 
\end{proof}

\begin{theorem}
	Esiste il \textsc{mcd} e lo possiamo calcolare con l'algoritmo di Euclide.
\end{theorem}
	
\begin{theorem}
	$ (\textsc{ed})\Rightarrow(\textsc{pid}) $
\end{theorem}
\begin{proof}
	Prendiamo un ideale $ I $ di un \textsc{ed} e sia $ x $ il suo elemento di grado minimo. Vogliamo dimostrare che 
	\[ I = (x) \]
	per assurdo. Supponiamo che esista $ y \in I $ tale che
	\[ x = qy + r \]
	con $ r \neq 0 $, abbiamo allora necessariamente che $ d(r) < d(x) $. Purtroppo però
	\[ r = x - qy \in I \]
	contraddice la minimalità di $ x $.
\end{proof}

\textbf{Controesempio?} 

\subsubsection{Dominio a Ideali Principali}
\begin{remark}
	Il \textsc{mcd}$ (a, b) $ esiste: è il generatore dell'ideale $$  (a, b) = (d)  $$
\end{remark}

\begin{theorem}\label{primossemassimale}
	Gli ideali primi sono anche massimali.
\end{theorem}
\begin{proof}
	\begin{align*}
		(x) \text{ primo} \;& \Rightarrow\; x \text{ è un elemento primo} & \text{(\ref{prpr})} \\
		& \Rightarrow x \text{ è irriducibile} &\text{(\ref{epei})} \\
		& \Rightarrow (x) \text{ è massiamale tra gli ideali primi} &\text{(\ref{mama})} \\
		& \Rightarrow (x) \text{ è massimale} &\text{(\textsc{pid})}
	\end{align*}
\end{proof}

\begin{remark}
	Ripercorrendo la catena di implicazioni sopra, ci si accorge che abbiamo mostrato anche l'equivalenza tra le definizioni di irriducibile e primo  nei \textsc{pid}.
\end{remark}

\subsubsection{Dominio a Fattorizzazione Unica}

\textbf{MCD.} Negli anelli \textsc{ufd} l'$ \textsc{mcd}(a, b) $ esiste, perché possiamo caratterizzarlo attraverso la fattorizzazione, come sugli interi. Però non è detto che appartenga all'ideale $ (a, b) $. Per esempio: in $ \Z[x] $ abbiamo che $ (2, x) = 1 $ ma non esistono polinomi $ f, g $ tali che
\[ 1 = 2f + xg \]
infatti valutando l'espressione in $ 0 $ otteniamo $ 1 = 2f(0) $.

\begin{theorem}[Caratterizzazione degli \textsc{ufd}] \label{carufd}
	Un dominio A è \textsc{ufd} se e solo se valgono le seguenti condizioni
	\begin{enumerate}
		\item irriducibile $ \Rightarrow $ primo
		\item Ogni catena discendente di divisibilità è stazionaria.
	\end{enumerate}
\end{theorem}

\begin{proof}
	Cominciamo con la freccia più interessante.
	
	 La proprietà 2 ci garantisce l'esistenza della fattorizzazione. Ragioniamo per assurdo: prendiamo $ x \in A\setminus(A^\times \cup \{0\}) $ e supponiamo che non ammetta fattorizzazione. Non può essere irriducibile, altrimenti avremmo trovato una fattorizzazione, pertanto si può scrivere come
	\[ x = y_0z_0 \quad\text{con } y_0, z_0 \notin A^\times \]
	Però questa non può essere una fattorizzazione, pertanto uno dei due non è né irriducibile né fattorizzabile, diciamo $ y_0 $. Allora lo possiamo scrivere come
	\[ y_0 = y_1 z_1 \]
	e trovandoci nelle stesse ipotesi di prima, possiamo iterare il procedimento indefinitamente, ottenendo
	\[ \cdots \mid y_2 \mid y_1 \mid y_0 \]
	una catena discendente infinita e non stazionaria, assurdo.
	
	La proprietà 1 ci garantisce l'unicità della fattorizzazione. Procediamo per induzione forte sulla lunghezza minima $ l_x $ della fattorizzazione di un certo elemento $ x $. Se $ l_x = 1 $ e
	\[ p = x = q_1 \cdots q_s \]
	poiché $ p $ è irriducibile, tutti i $ q_i $ tranne uno sono invertibili e quello che si salva non può che essere associato a $ p $. Supponiamo ora che tutti gli elementi per cui $ l_x < n $ abbiamo fattorizzazione unica e prendiamo un elemento $ x $ tale che $ l_x = n $. Allora prese
	\[ p_1 \cdots p_n = x = q_1 \cdots q_s \]
	sappiamo che $ p_n $ è irriducibile e quindi primo, dunque
	\[ p_n \mid q_1 \cdots q_s \quad\Rightarrow\quad \textsc{wlog }\; p_n \mid q_s  \]
	poiché anche $ q_n $ è irriducibile e quindi primo, devono essere associati. Pertanto dobbiamo avere che
	\[ p_1 \cdots p_{n-1} = \tilde{x} = q_1 \cdots q_{s-1} \]
	ma le due fattorizzazione sono uguali per ipotesi induttiva, poiché $ l_{\tilde{x}} \leq n-1 < n $.
	
	Viceversa, se $ A $ è \textsc{ufd}, presa una catena discendente di divisibilità
	\[ \cdots \mid a_2 \mid a_1 \mid a_0 \]
	possiamo considerare gli insiemi $ A_n $ di tutti i fattori di $ a_n $ presi con molteplicità e osservare che la catena discendente
	\[ \cdots \subseteq A_2 \subseteq A_1 \subseteq A_0 \]
	si stabilizza, perché le cardinalità corrispondenti formano una catena discendente di interi.
	
	Inoltre, preso un elemento irriducibile $ x $, se
	\[ x \mid ab \]
	un associato di $ x $ deve comparire nella fattorizzazione di almeno uno tra $ a $ e $ b $, quindi $ x $ divide almeno uno dei due e pertanto soddisfa la definizione di elemento primo.
\end{proof}

\begin{remark}
	Segue che $ \textsc{pid}\Rightarrow \textsc{ufd} $
\end{remark}
\begin{proof}
	La prima condizione l'abbiamo già mostrata. Presa una catena discendente di divisibilità, possiamo tradurla in una catena ascendente di ideali
	\[ (a_0) \subseteq (a_1 ) \subseteq (a_2) \subseteq \cdots \]
	Però l'ideale $ I = \bigcup (a_i) $ è generato da un elemento $ a $, perché tutti gli ideali sono principali. Ma per come è definito $ I $, l'elemento $ a $ già apparteneva a un certo $ (a_n) $, generando lui come tutti i suoi successori. 
\end{proof}

\textbf{Esempi.}
\begin{enumerate}
	\item $ \K [x, \sqrt{x}, \sqrt[3]{x}, \cdots ] $ non è \textsc{ufd}. Infatti 
	\[ \cdots \mid \sqrt[8]{x} \mid \sqrt[4]{x} \mid \sqrt[2]{x} \mid x \]
	è una catena discendente infinita di divisibilità.
	
	\item $ \Z[\sqrt{-5}] $ non è \textsc{ufd}. Infatti $ 2 $ è irriducibile ma non primo.
	Se \[ 2 = (a+b\sqrt{-5})(c + d\sqrt{-5}) \] in norma abbiamo una contraddizione, mentre
	\[ (1 + \sqrt{-5})\cdot (1 - \sqrt{-5}) = 2 \cdot 3 \]
	dunque \[ 2 \mid (1 + \sqrt{-5})\cdot (1 - \sqrt{-5})  \] ma
	\[ 4 = N(2) \nmid N(1 \pm \sqrt{-5}) = 6 \]
	
\end{enumerate}

\begin{definition}
	Chiamiamo \emph{contenuto} del polinomio $ f \in A[x] $ l'\textsc{mcd} dei suoi coefficienti
	\[ c(f) = \textsc{mcd}(a_0, \dots , a_n) \]
\end{definition}

\begin{definition}
	Chiamiamo \emph{primitivo} un polinomio $ f $ con contenuto unitario
	\[ c(f) = 1 \]
\end{definition}

\begin{theorem}[Lemma di Gauss]\label{LemmaGauss}
	Valgono i seguenti enunciati
	\begin{enumerate}
	\item Il prodotto di polinomi primitivi è primitivo.
	\item La funzione contenuto è completamente moltiplicativa.
	\item Se $ c(f) = 1 $ e $ f \mid g $ in $ \Q(A)[x] $, allora $ f \mid g $ in $ A[x] $.
	\item Se $ f = hg $ in $ \Q(A)[x] $, esistono $ h_1, g_1 \in A[x] $ tali che $ f = h_1g_1 $.
	\end{enumerate}
	
\end{theorem}
\begin{proof}
	
\begin{enumerate}
	\item  
\end{enumerate}
\end{proof}

\begin{theorem}
	$ A $ è \textsc{ufd} $ \Rightarrow $ $ A[x] $ è \textsc{ufd}
\end{theorem}
\begin{proof}
	Vogliamo utilizzare la caratterizzazione dei domini a fattorizzazione unica (\ref{carufd}).
	
	
	Iniziamo verificando che dato un polinomio $ f \in A[x] $ irriducibile, questo è anche primo. Ovviamente 
	\[ c(f) \mid f \quad\Rightarrow\quad c(f) = f \;\text{ o }\; c(f) = 1 \]
	Se $ c(f) = f $, allora $ f \in A $ e necessariamente $ f $ dev'essere irriducibile in $ A $.
	Se $ c(f) = 1 $, allora $ f $ è primitivo e deve necessariamente essere essere irriducibile anche in $ \Q(A)[x] $, altrimenti potremmo riportare la decomposizione in $ A[x] $ mediante il Lemma di Gauss. Ma $ \Q(A)[x] $ è un \textsc{ed}, pertanto $ f $ è anche primo. Quindi se $ f \mid gh $ in $ A[x] $, naturalmente lo divide anche in $ \Q(A)[x] $, dove essendo primo divide uno dei due, che quindi divide anche in $ A[x] $ (ancora una volta per Lemma di Gauss).
	
	
	Mostriamo ora che ogni catena discendente di divisibilità è stazionaria. Prendiamo una sequenza di polinomi $ \{ f_n \} $ tale che
	\[ \cdots \mid f_3 \mid f_2 \mid f_1 \mid f_0 \]
	Possiamo decomporre ogni polinomio nel prodotto della parte primitiva per il contenuto
	\[ f = c(f)f' \]
	e osservare che la relazione di divisibilità è soddisfatta se e solo se lo è per componenti
	\[ f \mid g \quad\Leftrightarrow\quad c(f) \mid c(g) \;\text{ e }\; f' \mid g \]
	ci riduciamo dunque a contemplare due diverse catene discendenti di divisibilità
	\[ \cdots \mid c(f_3) \mid c(f_2) \mid c(f_1) \mid c(f_0) \]
	stazionaria perché  $ A[x] $ è \textsc{ufd}
	\[ \cdots \mid f'_3 \mid f'_2 \mid f'_1 \mid f'_0 \]
	stazionaria perché  $ \Q(A)[x] $ è \textsc{ufd}.
\end{proof}



\begin{theorem}[Criterio di Eisenstein]
	Dato $ f \in A[x] $ primitivo e $ p \in A $ primo, se
	\begin{enumerate}
		\item  $ p \nmid a_n $.
		\item $ p \mid a_i $ per $ 0\leq i\leq n-1  $.
		\item $ p^2 \nmid a_0 $
	\end{enumerate}
	allora $ f $ è irriducibile.
\end{theorem}
\begin{proof}
	contenuto...
\end{proof}

\textbf{Esempi.}
\begin{enumerate}
	\item  In $ \K[x][\,t\,] $, l'elemento $ f = t^n - x $ è irriducibile per il criterio di Eisenstein con $ p = x $. 
	\item $ \K[\{x_i\}_{i \in \mathbb{N}}] $ non è Noetheriano
	\[ (x_0) \subsetneq (x_0, x_1) \subsetneq (x_0, x_1, x_2) \subsetneq \dots  \]
	ma è \textsc{ufd}. Infatti preso un polinomio $ f $, questo è composto da un numero finito di addendi di grado finito, quindi è contenuto in $ \K[x_1, \dots, x_m] $ per un qualche intero $ m $, che è un $ \textsc{ed} $. Qui dentro ammette fattorizzazione unica. Inoltre, nella fattorizzazione non compaiono termini del tipo $ x_k $ con $ k > m $: se così fosse, guardandoli come polinomi in $ x_k $, non sarebbero rispettate le condizioni sulla funzione grado.
\end{enumerate}


\end{multicols}








\setcounter{section}{10} % per scegliere la lettera giusta
\section{Teoria dei Campi e di Galois}

\subsection{Richiami}
\begin{multicols}{2}

\begin{definition}
	Data un'estensione di campi $ K \subseteq L $ (o anche $ \faktor{L}{K} $), un elemento $ \alpha \in L $ si dice \emph{algebrico} su $ K $ quando 
	\[ \exists f \in K[x] \quad \text{tale che} \quad f(\alpha) = 0 \]
	Un elemento si dice \emph{trascendente} se non è algebrico.
\end{definition}

\begin{definition}
	Un'estensione di campi $ K \subseteq L $ si dice \emph{algebrica} quando ogni elemento di $ L $ è algebrico su $ K $
\end{definition}

\begin{definition}
	L'indice di un'estensione $ [L:K] $ è la dimensione di $ L $ visto come $ K $ spazio vettoriale.
\end{definition}

\begin{definition}
	Diciamo che un'estensione $ K \subseteq L $ è \emph{semplice} se esiste un elemento $ \alpha \in L $ tale che
	\[ L = K(\alpha) = \left\{ \frac{f(\alpha)}{g(\alpha)} \mid f, g \in K[\,x\,] \quad g(x) \neq 0  \right\} \]
\end{definition}

Possiamo costruire l'estensione semplice a partire dall'anello dei polinomi su K, applicandovi l'omomorfismo di valutazione in $ \alpha $
\fun{\varphi_\alpha}{K[\,x\,]}{K[\,\alpha\,] \subseteq L}{f(x)}{f(\alpha)}
l'immagine $ K[\alpha] $ è un sottoanello di un campo ed è quindi un dominio, di cui possiamo costruire il campo dei quozienti
\[ \Q(K[\,\alpha\,]) = K(\alpha) \]
Possiamo però mostrare che, se $ \alpha $ è algebrico, questa operazione non è necessaria. Infatti, il nucleo dell'omomorfismo di valutazione è costituito da tutti quei polinomi che si annullano in $ \alpha $, i quali costituiscono un ideale principale (perché $ K[\,x\,] $ è \textsc{ed}, dunque \textsc{pid}):
\[ \ker\varphi_\alpha = (\mu(x)) \]
Abbiamo già osservato che  l'immagine è un dominio, dunque $ (\mu(x)) $ è un'ideale primo (\ref{idealequoziente}).
\[ K[\,\alpha\,] \cong \frac{K[\,x\,]}{(\mu(x))} \]
Ma, poiché ci troviamo in un \textsc{pid}, $ (\mu(x)) $ è anche un ideale massimale (\ref{primossemassimale}) e dunque $ K[\,\alpha\,] $ è un campo, che coincide con il proprio campo dei quozienti:
\[ K[\,\alpha\,] = \Q(K[\,\alpha\,]) = K(\alpha) \]
In questo caso, dobbiamo avere anche che $ \mu(x) $ è un polinomio irriducibile, possiamo quindi scegliere un rappresentate privilegiato tra i generatori del nucleo:

\begin{definition}
	Chiamiamo \emph{polinomio minimo} di $ \alpha $ l'unico generatore monico di $ \ker\varphi_\alpha $, che indicheremo con $ \mu_\alpha $:
	\[ \ker\varphi_\alpha = (\mu_\alpha(x)) \]
\end{definition}

\begin{remark}
	Nel caso di estensioni algebriche semplici \[ [K(\alpha) : K ] = \deg \mu_\alpha \] questo perché $ K[\,\alpha\,] $ è un $ K $-spazio vettoriale di base $$  1,\, \alpha,\, \dots,\, \alpha^{\deg \mu_\alpha-1}  $$
\end{remark}

\begin{definition}
	Dato un insieme $ S \subseteq L $  definiamo
	\[ K(S) = \bigcap_{K, S \subseteq F \subseteq L} F  \]
\end{definition}

\begin{definition}
	Chiamiamo \emph{Torre di Estensioni} una catena di inclusioni tra campi
	\[ K \subseteq F \subseteq L \qquad\text{o anche}\qquad \begin{tikzcd}[row sep=0.25cm]
	L \arrow [d,-] \\
	F \arrow [d,-] \\
	K
	\end{tikzcd} \]
\end{definition}

Elenchiamo ora una serie di proprietà delle torri di estensioni:

\begin{itemize}
	\item $ L/K $ è finita $ \Leftrightarrow $ $ L/F $ e $ F/K $ sono finite
	\[ \text{e in tal caso } [L : K] = [L : F][F : K] \]
	
	\item $ L/K $ è finita $ \Rightarrow $ $ L/K $ è algebrica.
	
	\item Il viceversa è falso, si pensi a $ [\bar{\Q} : \Q] $.
	
	\item $ L/K $ algebrica e finitamente generata $ \Rightarrow $ finita.
	
	\item $ L/K $ algebrica $ \Leftrightarrow $ $ L/F $ e $ F/K $ algebriche.
	
	
\end{itemize}

\begin{definition}
	Un campo $ \Omega $ è detto \emph{algebricamente chiuso} quando ogni polinomio $ f \in \Omega[\, x\,] $ ammette almeno una radice in $ \Omega $.
\end{definition}

\begin{definition}
	$ \bar{\Omega} $ è detta \emph{chiusura algebrica} di $ \Omega $ se:
	\begin{enumerate}
		\item $ \bar{\Omega} $ è algebricamente chiuso.
		\item $ \faktor{\bar{\Omega}}{\Omega} $ è algebrica.
	\end{enumerate}
\end{definition}

\begin{theorem}
	Ogni campo $ K $ ammette una chiusura algebrica $ \bar{K} $ e ogni due sue chiusure $ \bar{K}, \bar{\bar{K}} $ sono $ K $ isomorfe, ovvero esiste un isomorfismo
	\[ \varphi : \bar{K} \to \bar{\bar{K}} \]
	tale che $ \varphi_{|K} = id $.
\end{theorem}

\textbf{Esercizi.}
\begin{itemize}
	\item Se $ \alpha_1,\, \cdots,\, \alpha_m $ sono le radici di $ f \in K[\,x\,] $, allora
	 \[ K(\alpha_1,\, \cdots,\, \alpha_m) \leq (\deg f)! \]
	 
	 \item Quante sono le $ K $-immersioni di $ K(\alpha) $ in $ \bar{K} $?
	 
	 \item Quante sono le radici distinte di un polinomio irriducibile?
\end{itemize}


\end{multicols}


\subsection{Separabilità e Normalità}


\begin{multicols}{2}
	
	\begin{theorem}\label{est}
		Sia $ \faktor{K(\alpha)}{K} $ un'estensione algebrica.\\ Ogni immersione
		\[ \varphi: K \hookrightarrow \bar{K} \]
		ammette esattamente tante estensioni quante le radici di $ \mu_\alpha $.
	\end{theorem}
	\begin{proof}
		Consideriamo l'estensione algebrica come quoziente
		\[ K(\alpha) \cong \frac{K[\, x \,]}{(\mu_\alpha)} \]
		e definiamo l'estensione a partire da $ K[\, x \,] $ in modo che
		\begin{align*}
		\Phi\colon &K[\, x \,] \to \bar{K} \\
		& 1 \mapsto 1 \\
		& x \mapsto \beta 
		\end{align*}
		imponendo che il nucleo corrisponda con l'ideale generato dal polinomio minimo di $ \alpha $
		\[ (\mu_\alpha) = \ker{\Phi} = \{ p \in K[\, x \,] \mid \varphi(p)(\beta) = 0 \} \]
		otteniamo una mappa iniettiva, come cercato.
		Dobbiamo pertanto scegliere $ \beta $ tra le radici di $ \varphi(\mu_\alpha) $; facendolo siamo sicuri che
		\[ (\mu_\alpha) \subseteq \ker{\Phi} \]
		ed essendo $ (\mu_\alpha) $ massimale e l'immagine non banale (perché stiamo quantomeno immergendo il campo), otteniamo
		\[ (\mu_\alpha) = \ker{\Phi} \]
	\end{proof}
	\begin{remark}
		Il teorema si generalizza facilmente per induzione ad estensioni finite. Vale inoltre un risultato analogo per estensioni infinite, che garantisce quantomeno l'esistenza dell'estensione.
	\end{remark}


	\begin{definition}[Separabile]
		Un polinomio $ f \in K[x] $ si dice \emph{separabile} se ha tutte le radici distinte in $ \bar{K} $.
		
		Un'estensione algebrica $ \faktor{L}{K} $ si dice \emph{separabile} se, comunque scegliamo $\alpha \in L  $, il corrispondente polinomio minimo $ \mu_\alpha $ è separabile.
	\end{definition}

\begin{theorem}[Criterio della derivata]
	Se siamo in caratteristica $ 0 $ o su un campo finito, allora tutti i polinomi irriducibili sono separabili.
\end{theorem}
\begin{proof}
	Consideriamo $ f $ assieme alla sua derivata formale $ f' $. E' chiaro, per come è definita la derivata, che $ f $ ha radici multiple se e solo se ha radici in comune con $ f' $. Consideriamo
	$$  \textsc{mcd}(f, f')  $$
	che, poiché $ f $ è irriducibile, può essere solo $ 1 $ o $ f $.
	
	Osserviamo che l'\textsc{mcd} di due polinomi appartiene all'anello di definizione di questi, dunque se sono coprimi
	\[ \exists g, h  \qquad fg +f'h = 1 \]
	pertanto lo sono a maggior ragione nell'anello costruito sulla chiusura algebrica. Così, se $  \textsc{mcd}(f, f') = 1  $, abbiamo finito. Altrimenti dobbiamo avere che $ f'=0 $, ma è chiaro che questo non può succedere in caratteristica zero.
	
	Se ci mettiamo su $ \mathbb{F}_p[X] $, possiamo mostrare che
	\[ f' = 0 \quad\Rightarrow\quad f = g^p \]
	semplicemente scrivendo esplicitamente il polinomio e osservando che tutti i coefficienti che possono non essere nulli sono quelli dei termini $ x^{pk} $.
\end{proof}

\begin{theorem}
	Sia $ \faktor{K(\alpha)}{K} $ un'estensione algebrica, di grado $ n $. Se è separabile, ogni immersione
	\[ \varphi: K \hookrightarrow \bar{K} \]
	ammette esattamente n estensioni.
\end{theorem}
\begin{proof}
	Il grado di $ \mu_\alpha $ è $ n $, dunque ammette $ n $ radici distinte.
\end{proof}
\begin{remark}
	Il teorema precedente si generalizza per induzione al caso di estensioni finite.
\end{remark}

\begin{definition}
	In un'estensione algebrica $ \faktor{L}{K} $ si dicono \emph{coniugati} di un elemento $ \alpha \in L $, tutte le soluzioni del polinomio minimo $ \mu_\alpha $.
\end{definition}

\begin{remark}
	Le estensioni $ \varphi_i $ del teorema di sopra sono proprio gli omomorfismi di coniugio, che mandano $ \alpha_1 \mapsto \alpha_i $.
\end{remark}



Introduciamo ora un'altra utile proprietà di alcune estensioni

\begin{definition}[Normalità]
	Un'estensione algebrica $ \faktor{F}{K} $ si dice \emph{normale} se, preso comunque un $ K $-omomorfismo
	\[ \varphi \colon F \to \bar{K} \qquad\text{tale che } \varphi_{|K} = id \]
	si ha che $ \varphi(F) = F $.
\end{definition}

\textbf{Esempi.}
\begin{enumerate}
	\item Tutte le estensioni di grado $ 2 $ sono normali.
	\item $ \faktor{\Q(\sqrt[3]{2})}{\Q} $ non è normale.
\end{enumerate}

\begin{theorem}
	In un'estensione $ \faktor{F}{K} $ algebrica e normale, ogni polinomio $ f \in K[\,x\,] $ irriducibile con almeno una radice in $ F $, ha tutte le radici in $ F $.
\end{theorem}
\begin{proof}
	Consideriamo tutte le radici che $ f $ ammette in $ \bar{K} $: $ \alpha_1, \dots, \alpha_n $. Possiamo assumere \textsc{wlog} che $ \alpha_1 \in F $. Definiamo i $ K $-omomorfismi $ \varphi_1,\, \dots,\, \varphi_n $ in modo che
	\fun{\varphi_i}{K(\alpha_1)}{\bar{K}}{\alpha_1}{\alpha_i}
	Possiamo estenderli grazie al teorema \ref{est} a $ K $-omomorfismi
	\[ \tilde{\varphi}_i \colon F \to K \] 
	ma $ \faktor{F}{K} $ è normale, pertanto 
	$$  \tilde{\varphi}_i(F) = F \qquad\forall \; i \in \{1, \dots, n\} $$
	dunque $ \alpha_i \in F $ per ogni $ i $, come volevamo.
	
\end{proof}

\begin{theorem}\label{carnorm}
	Sia $ \faktor{F}{K} $ un'estensione (algebrica) finita. Questa è normale se e solo corrisponde al campo di spezzamento di una famiglia di polinomi in $ K[\,x\,] $.
\end{theorem}
\begin{proof}
	$ \Rightarrow$. La finitezza ci dice che l'estensione è anche finitamente generata
	 \[ F = K(\alpha_1, \dots, \alpha_n ) \]
	ci basta dunque prendere la famiglia di polinomi $ \{\mu_{\alpha_i} \}_{i} $ e, per il teorema precedente, questi polinomi si spezzano in $ F $.
	
	$ \Leftarrow $. Supponiamo che $ F $ sia il campo di spezzamento della famiglia di polinomi $ \{f_i\}_{i \in I} $, con radici $ \{\alpha_i,j\}_{i,j} $. Abbiamo che 
	\[F = K(\{\alpha_{i, j}\})\]
	 Preso un $ K $-automorfismo
	 \[ \varphi \colon F \to \bar{K} \]
	 abbiamo già osservato nella dimostrazione del teorema \ref{est} che questo deve necessariamente permutare i coniugati, dunque sicuramente
	 \[ \varphi(F) \subseteq F \]
	 ma, visto che $ \varphi $ è iniettivo, $ \varphi(F) $ e $ F $ hanno lo stesso grado e pertanto sono la stessa estensione di $ K $
\end{proof}

\end{multicols}

\subsection{Teoria di Galois}
\begin{multicols}{2}
	
	\begin{definition}
		Un'estensione $ \faktor{E}{K} $ algebrica si dice \emph{di Galois} se è sia normale che separabile.
	\end{definition}
	
	(Da qui in avanti svilupperemo la Teoria di Galois per estensioni \emph{finite}, pertanto quando diremo che un'estensione ha questa proprietà, sottointenderemo la finitezza. )
	
	In questo caso il gruppo degli automorfismi 
	\[ \text{Aut}_K(E) = \{ \varphi : E \to E \mid \varphi_K = id \} \]
	per normalità sono proprio le estensioni delle immersioni di $ K $ nella sua chiusura algebrica
	\[ \text{Aut}_K(E) = \{ \varphi : E \to \bar{K} \mid \varphi_K = id  \} \]
	che, per separabilità, sono tante quante il grado dell'estensione
	\[ |\text{Aut}_K(E)| = [E : K] \]
	e viene chiamato gruppo di Galois dell'estensione
	\[ \Gal{\faktor{E}{K}} := \text{Aut}_K(E) \]
	
	Possiamo anche parlare del gruppo di Galois di un polinomio, riferendoci indirettamente al gruppo di Galois dell'estensione del suo campo di spezzamento, le due definizioni sono equivalenti per la proposizione \ref{carnorm}.
	
	Pensarla in quest'ottica ci permette di procede con una piacevole osservazione:
	\begin{remark}
		Il gruppo di Galois è composto da tutti e soli quegli omomorfismi che coniugano le radici di $ f $, pertanto
		\[ \Gal{f} \hookrightarrow \S_n \]
		Inoltre l'azione del gruppo su queste radici è \emph{transitiva}, nel senso che appartengono tutte alla stessa orbita.
	\end{remark}
	
	\textbf{Esempi.}
	\begin{enumerate}
		\item Se $ \faktor{E}{K} $ è di grado $ 2 $ e separabile, allora è di Galois e
		\[ \Gal{\faktor{E}{K}} \cong \Z_2 \]
		\item $ \Gal{x^3 -2} \cong \S_3 $
		\item $ \Gal{\Q(\zeta_7)/\Q} \cong \Z_6 $.
		\item $ \Gal{\Q(\zeta_7+ \zeta_7^{-1})/\Q} \cong \Z_3 $
	\end{enumerate}
	
	Come si comporta la proprietà di Galois sulle torri? Consideriamo
	\[\begin{tikzcd}[row sep=0.25cm]
	L \arrow [d,-] \\
	F \arrow [d,-] \\
	K
	\end{tikzcd} \]
	
	e chiediamoci:
	\begin{itemize}
		\item Se L/K è di Galois, lo è anche L/F?
		
		 Sì! La separabilità passa, perché un polinomio minimo in $ F[\,x\,] $ divide il corrispondente in  $ K[\,x\,] $. La normalità passa, perché se $ \varphi $ è un $ F $-omomorfismo è anche un $ K $-omomorfismo, che dunque deve fissare $ \varphi(L)=L $.
		
		\item Se L/K è di Galois, lo è anche F/K? No!
		\[\begin{tikzcd}[column sep=0.25cm]
		\Q \arrow [r,-] &
		\Q(\sqrt[4]{2}) \arrow [r,-] &
		\Q(\sqrt[4]{2}, i)
		\end{tikzcd} \]
		
		\item Se L/F e F/K sono di Galois, lo è anche L/K? No!
		\[\begin{tikzcd}[column sep=0.25cm]
		\Q \arrow [r,-] &
		\Q(\sqrt{2}) \arrow [r,-] &
		\Q(\sqrt[4]{2})
		\end{tikzcd} \]
	\end{itemize}
	
	\begin{theorem}[di corrispondenza]
		Sia $ \faktor{L}{K} $ un'estensione di Galois finita. La mappa
		\fun{f}
		{\{F \text{ campo} \mid K \subseteq F \subseteq L \}}
		{\left\{ H < \Gal{\faktor{L}{K}} \right\}}
		{F}
		{\Gal{\faktor{L}{F}}}
		è bigettiva con inversa
		\[ H \mapsto L^H = \{ \alpha \in L \mid \sigma\alpha = \alpha \;\forall \sigma \in H \} \]
		Inoltre
		\[ H \lhd G \LR \faktor{L^H}{K} \text{ è di Galois} \]
		e in questo caso
		\[ \Gal{\faktor{L^H}{K}} \cong \frac{\Gal{\faktor{L}{K}}}{\Gal{\faktor{L}{L^H}}} \]
	\end{theorem}
	
	
	
\end{multicols}

\newpage
\listoftheorems[ignoreall,show={theorem}]
\end{document}